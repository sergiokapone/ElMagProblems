\protect \section *{\nameref *{AlternatingCurrent}}
\begin{Solution}{7.{1}}
	$I = \frac{\EMF}{R}\left( 1-e^{-\frac{R}{2L}t}\right) $.
\end{Solution}
\begin{Solution}{7.{2}}
	$I_1 = \frac{\EMF}{R} \frac{L_2}{L_1 + L_2}$, $I_2 = \frac{\EMF}{R} \frac{L_1}{L_1 + L_2}$.
\end{Solution}
\begin{Solution}{7.{3}}
	$Q = \frac{C_1\EMF^2}{2} = 0.3$~Дж.
\end{Solution}
\begin{Solution}{7.{4}}
	$Q = \frac{L_2\EMF^2}{2r^2}$.
\end{Solution}
\begin{Solution}{7.{5}}
	$I_{\max} = \EMF\sqrt{\nfrac{C}{L}}$, $U_{C\max} =2 \EMF$.
\end{Solution}
\begin{Solution}{7.{7}}
	$\omega = \frac{i}{2RC} \pm \sqrt{\frac{1}{LC} - \frac{1}{4R^2C^2}}$, комплексному $\omega$ відповідають згасаючі коливання в контурі.
\end{Solution}
\begin{Solution}{7.{8}}
	$M = -L_2$.
\end{Solution}
\begin{Solution}{7.{9}}
	$\omega_{1,2}^2 = \frac{L_1C_1 + L_2C_2 \pm \sqrt{(L_1C_1 - L_2C_2)^2 + 4C_1C_2L_{12}^2}}{2C_1C_2(L_1L_2 - L_{12}^2)}$.
\end{Solution}
\begin{Solution}{7.{13}}
    $Z = 43.71$~Ом, $\phi = 46.7^\circ$.
\end{Solution}
\begin{Solution}{7.{14}}
    $24$~Ом.
\end{Solution}
\begin{Solution}{7.{15}}
	$I = \frac{\EMF}{i\omega L}$, тобто струм через амперметр не залежить від опору реостата.
\end{Solution}
\begin{Solution}{7.{16}}
	$I = \EMF_0\frac{1 - \omega^2CL}{\omega L(2 - \omega^2CL)}\sin\omega t$, $I = \EMF_0\frac{1 - \omega^2CL}{\omega(2 - \omega^2CL)}\sin\omega t$.
\end{Solution}
\begin{Solution}{7.{17}}
	$RC = \nfrac{L}{R}$, $\tg\phi = \frac{\omega L}{R}$.
\end{Solution}
\begin{Solution}{7.{18}}
	$\omega^2LC = 1$.
\end{Solution}
\begin{Solution}{7.{19}}
	$I_1 = \frac{\EMF_0}{R}\cos 2\omega t$, $I_1 = \frac{\EMF_0}{2(R + R_1)}$.
\end{Solution}
\begin{Solution}{7.{21}}
    $R = 10$~Ом.
\end{Solution}
\begin{Solution}{7.{22}}
    $I_0 = 13$~мА, $P = 160$~мВт, $\delta\phi = 0$.
\end{Solution}
\begin{Solution}{7.{23}}
    $I_{\eff} = 10$~А, $P = 1.52$~кВт.
\end{Solution}
\begin{Solution}{7.{24}}
    $0.15$~мВт.
\end{Solution}
\begin{Solution}{7.{25}}
    $Q = \sqrt{\eta^2 - \frac14}$.
\end{Solution}
\begin{Solution}{7.{26}}
    $5$~мкВт.
\end{Solution}
\begin{Solution}{7.{27}}
    $Q = \frac{I_0^2}{2P}\sqrt{\frac{L}{C}}$.
\end{Solution}
\begin{Solution}{7.{28}}
    $Q = \frac{V_0^2}{2P}\sqrt{\frac{C}{L}}$.
\end{Solution}
\begin{Solution}{7.{29}}
    $V_R = 5$~В, $V_L = 40$~В, $V_C = 50$~В.
\end{Solution}
\begin{Solution}{7.{30}}
	$\omega L = R$, $V_{out} = V_{in} \left( \frac13 - \frac{R_1}{R_1 + R_2}\right) $.
\end{Solution}
\begin{Solution}{7.{31}}
	$R_1C = RC_1$.
\end{Solution}
\begin{Solution}{7.{32}}
	$P = 160$~Вт, $\phi \approx 37^\circ$.
\end{Solution}
\begin{Solution}{7.{33}}
	Імпеданс кола $Z = \sqrt{\left( \frac{Z_1}{2}\right)^2 + Z_1Z_2}$.
	$Z = \sqrt{\frac{L}{C} - \frac{\omega^2L^2}{4}}$. При $\omega > \nfrac{2}{\sqrt{LC}}$.
\end{Solution}
\begin{Solution}{7.{34}}
	$F = \frac{2I_1I_2}{2cr}\cos\phi$.
\end{Solution}
\begin{Solution}{7.{35}}
	\begin{enumerate*}[label=\alph*)]
		\item $\omega \le \sqrt{\frac{2}{LC}}$,
		\item $\omega \ge \sqrt{\frac{1}{2LC}}$,
		\item $\Omega_2 < \omega < \Omega_1$, де $\Omega_1^2 = \nfrac{1}{LC} $, $\Omega_2^2 = \frac15 \left( \Omega_1^2 + 4\Omega^2\right)  $, $\Omega = \nfrac{1}{4LC}$.
	\end{enumerate*}
\end{Solution}
