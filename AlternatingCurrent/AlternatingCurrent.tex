% !TeX program = lualatex
% !TeX encoding = utf8
% !TeX spellcheck = uk_UA
% !TeX root =../EMProblems.tex

%=========================================================
\Opensolutionfile{answer}[\currfilebase/\currfilebase-Answers]
\Writetofile{answer}{\protect\section*{\nameref*{\currfilebase}}}
\chapter{Квазістаціонарні струми в електричних колах}\label{\currfilebase}
%=========================================================
\def\eff{e\kern-2.2pt f\kern-3pt f}
\section{Перехідні процеси в електричних колах}

%=========================================================
\begin{problem}\label{prb:RL}% Іродов 2.349
Знайти закон зміни в часі струму, який тече через індуктивність $L$ в схемою (рис.~\ref{RL}) після замикання ключа  в момент $t = 0$.
\begin{solution}
	$I = \frac{\EMF}{R}\left( 1-e^{-\frac{R}{2L}t}\right) $.
\end{solution}
\end{problem}

%=========================================================
\begin{problem}\label{prb:RLL}% Іродов 2.350
У схемі (рис.~\ref{RLL}) відомі ЕРС $\EMF$ джерела, опір $R$ і індуктивності котушок $L_1$ і $L_2$. Внутрішнім опором джерела і опором котушок можна знехтувати. Знайти струми, які встановляться в котушках після замикання ключа.
\begin{solution}
	$I_1 = \frac{\EMF}{R} \frac{L_2}{L_1 + L_2}$, $I_2 = \frac{\EMF}{R} \frac{L_1}{L_1 + L_2}$.
\end{solution}
\end{problem}

%=========================================================
\begin{figure}[h!]\centering
	\begin{minipage}[b]{0.45\linewidth}\centering
		\begin{tikzpicture}
			\draw (0,1) to [battery={info={$\EMF$}}] (0,-1);
			\draw (0,1) to [make contact] +(2,0) node (A) [contact] {} to [resistor={info'={$R$}}] +(2,-2) node (B) [contact]  {} to [resistor={info'={$R$}}] + (0,-2);
			\draw (A) -- ++(1,0) to [inductor={info'={$L$}}] ++(0,-2) coordinate (S)
			-- (B);
			\draw [white] (S) to [battery={white}] +(0.2,0); % jus shifting
		\end{tikzpicture}
		\caption{До задачі~\ref{prb:RL}}
		\label{RL}
	\end{minipage}
	%---------------------------------------------------------
	\begin{minipage}[b]{0.45\linewidth}\centering
		\begin{tikzpicture}
			\draw (0,0) coordinate (A) to [battery={info={$\EMF$}}] +(1,0) to [resistor={info={$R$}}] +(3,0) coordinate (B);
			\draw (A) -- +(0,1) node [contact] (C) {} -- ++(0,2) to [inductor={info={$L_1$}}] ++(3,0) -- +(0,-1) node [contact] (D) {} to [make contact] (B)
			(C) to [inductor={info={$L_2$}}] (D);
		\end{tikzpicture}
		\caption{До задачі~\ref{prb:RLL}}
		\label{RLL}
	\end{minipage}
\end{figure}
%=========================================================

%=========================================================
\begin{problem}\label{prb:c4}
У колі, зображеному на рис.~\ref{c4}, ЕРС батареї $\EMF = 100$~В, опори резісторів $R_1 = 10$~Ом і $R~2 = 6$~Ом, а ємності конденсаторів $C_1 = 60$~мкФ і $C_1 = 100$~мкФ. У початковому стані ключ розімкнений, а конденсатори не заряджені. Через деякий час після замикання ключа в системі встановиться рівновага. Яка кількість теплоти виділиться в колі за час встановлення рівноваги?
\begin{solution}
	$Q = \frac{C_1\EMF^2}{2} = 0.3$~Дж.
\end{solution}
\end{problem}

%=========================================================
\begin{problem}\label{prb:fizportal}
У схемі на рис.~\ref{fizportal} після встановлення струмів миттєво перекидають ключ з положення $1$ в положення $2$. Вважаючи котушки $L_1$ і $L_2$ ідеальними, визначте кількість теплоти, що виділиться на резисторі $R$. ЕРС джерела струму $\EMF$, а внутрішній опір $r$.
\begin{solution}
	$Q = \frac{L_2\EMF^2}{2r^2}$.
\end{solution}
\end{problem}

%=========================================================
\begin{figure}[h!]\centering
	%---------------------------------------------------------
	\begin{minipage}[t]{0.45\linewidth}\centering
		\begin{tikzpicture}
			\draw (-2,1) to [battery={info'={$\EMF$}}] (-2,-1);
			\draw (-2,1) to [capacitor={info'={$C_1$}}] ++(1,0) to [make contact] ++(1,0) coordinate (A) node [contact] {} to[resistor={info={$R_1$}}] +(0,-2) coordinate (B) node [contact] {} -- (-2,-1)
			(A) to[resistor={info={$R_2$}}] ++(2,0) to [capacitor={info'={$C_2$}}] ++(0,-2) -- (B);
		\end{tikzpicture}
		\caption{До задачі~\ref{prb:c4}}
		\label{c4}
	\end{minipage}
	%---------------------------------------------------------
	\begin{minipage}[t]{0.45\linewidth}\centering
		\begin{tikzpicture}
			\draw (1,-1) to [resistor={info'={$R$}}] (1,1) coordinate (A) to [make contact={yscale={-1}}] +(-0.6,0) coordinate (one) -- (-1,1) to [battery={info={$r$}, info'={$\EMF$}}] (-1,-1) -- (1,-1);
			\draw (A) to [inductor={info'={$L_1$}}] ++(2,0) coordinate (B) to [inductor={info={$L_2$}}] +(0,-2) -- (1,-1);
			\draw (1,1.3) node[contact] {} node[above] {$2$} -- ({1,1.3}-|B) -- (B);
			\node[contact] at ([xshift=0.1cm]one) {} ;
			\node[below] at ([xshift=0.1cm]one) {$1$};
		\end{tikzpicture}
		\caption{До задачі~\ref{prb:fizportal}}
		\label{fizportal}
	\end{minipage}
	%---------------------------------------------------------
\end{figure}
%=========================================================

\section{Вільні коливання в електричних колах}

%=========================================================
\begin{problem}\label{prb:Irodov3.118}
Знайти максимальний струм в колі (рис~\ref{Irodov3.118}) і максимальну напруга на
конденсаторі після замикання ключа. Активним опором кола можна знехтувати.
\begin{solution}
	$I_{\max} = \EMF\sqrt{\nfrac{C}{L}}$, $U_{C\max} =2 \EMF$.
\end{solution}
\end{problem}
%---------------------------------------------------------
\begin{figure}[h!]\centering
	\begin{tikzpicture}
		\draw
		(0,0) to [make contact] (2,0) to [capacitor={info'={$C$}}] (2,2) to [inductor={rotate=180, info={$L$}}]  (0,2) to [battery={info'={$\EMF$}}] (0,0);
	\end{tikzpicture}
	\caption{До задачі~\ref{prb:Irodov3.118}}
	\label{Irodov3.118}
\end{figure}
%---------------------------------------------------------

%=========================================================
\begin{problem}% КРС 9.5
Коливальний контур містить індуктивність і ємність. В деякий момент часу з конденсатора швидко витягують пластину з діелектричною проникністю $\epsilon$. Як зміниться частота коливань контуру? У скільки разів зміняться максимальні величини заряду на конденсаторі і струму в котушці, якщо пластину витягують в момент, коли заряд на конденсаторі
\begin{enumerate*}[label=\alph*)]
	\item відсутній
	\item максимальний?
\end{enumerate*}
\end{problem}

%=========================================================
\begin{problem}\label{prb:Irodov3.140}
Знайти частоти згасаючих коливань контуру, який показаний показаного на рис~\ref{Irodov3.140}. Ємність $C$, індуктивність $L$ і активний опір $R$ вважати відомими.
\begin{solution}
	$\omega = \frac{i}{2RC} \pm \sqrt{\frac{1}{LC} - \frac{1}{4R^2C^2}}$, комплексному $\omega$ відповідають згасаючі коливання в контурі.
\end{solution}
\end{problem}

%=========================================================
\begin{problem}\label{prb:krs10.41} % КРС 10.41
Високодобротний коливальний контур (рис.~\ref{krs10.41}) включає дві послідовно з'єднані котушки з індуктивностями $L_1$ і $L_2$. Після того, як котушку $L_2$ замикають накоротко, частота власних коливань контуру не змінюється. Визначити коефіцієнт взаємної індукції $М$.
\begin{solution}
	$M = -L_2$.
\end{solution}
\end{problem}

%=========================================================
\begin{problem}\label{prb:Bugaenko234}
Визначити частоти вільних коливань у двох індуктивно зв'язаних контурах, схема яких подана на рис.~\ref{Bugaenko234}. Коефіцієнт взаємної індукції контурів~$L_{12}$.
\begin{solution}
	$\omega_{1,2}^2 = \frac{L_1C_1 + L_2C_2 \pm \sqrt{(L_1C_1 - L_2C_2)^2 + 4C_1C_2L_{12}^2}}{2C_1C_2(L_1L_2 - L_{12}^2)}$.
\end{solution}
\end{problem}

%=========================================================
\begin{figure}[h!]\centering
	%---------------------------------------------------------
	\begin{minipage}[t]{0.3\linewidth}\centering
		\begin{tikzpicture}
			\draw
			(0,0) to [inductor={info={$L$}}] (0,2) -- +(1,0) node [contact] (A) {} -- +(2,0) to [capacitor={info={$C$}}] (2,0) -- +(-1,0) node [contact] (B) {} -- +(-2,0)
			(A) to [resistor={info'={$R$}}] (B);
		\end{tikzpicture}
		\caption{До задачі~\ref{prb:Irodov3.140}}
		\label{Irodov3.140}
	\end{minipage}
	%---------------------------------------------------------
	\begin{minipage}[t]{0.3\linewidth}\centering
		\begin{tikzpicture}[rotate=90]
			\newcommand{\shift}{0.8}
			\draw (2,0) to [capacitor={info'={$C$}}] (0,0) -- ++(0,1) -- ++(\shift,0) to [inductor={rotate=180, info'={$L_1$}}] ++(0,1) -- ++(-\shift,0) -- ++(0,1) -- ++(2,0) -- ++(0,-1) coordinate (A) -- ++(-\shift,0) to [inductor={rotate=180, info'={$L_2$}}] ++(0,-1) -- ++(\shift,0) coordinate (B) -- (2,0)
			(A) to [make contact] (B);
		\end{tikzpicture}
		\caption{До задачі~\ref{prb:krs10.41}}
		\label{krs10.41}
	\end{minipage}
	%---------------------------------------------------------
	\begin{minipage}[t]{0.3\linewidth}\centering
		\begin{tikzpicture}
			\coordinate (A) at (0,2);
			\coordinate (B) at (0,0);
			\coordinate (C) at (2,2);
			\coordinate (D) at (2,0);
			\draw
			(B) to [capacitor={info'={$C_1$}}] (A) -- (C) to [inductor={info'={$L_1$}}] (D) -- (B);
			\coordinate (A1) at (2.5,2);
			\coordinate (B1) at (2.5,0);
			\coordinate (C1) at (4.5,2);
			\coordinate (D1) at (4.5,0);
			\draw (B1) to [inductor={info'={$L_2$}}] (A1) -- (C1) to [capacitor={info'={$C_2$}}] (D1) -- (B1);
		\end{tikzpicture}
		\caption{До задачі~\ref{prb:Bugaenko234}}
		\label{Bugaenko234}
	\end{minipage}
	%---------------------------------------------------------
\end{figure}
%=========================================================

%%=========================================================
%\begin{problem}\label{prb:Bugaenko}
%Визначити моди вільних незгасаючих коливань у колі, схема якого подана на
%\begin{enumerate*}[label=\alph*)]
%\item рис.~\ref{Bugaenko235} та
%\item рис.~\ref{Bugaenko236}.
%\end{enumerate*}
%\begin{solution}
%\begin{enumerate}[label=\alph*)]
%\item $\omega_{1,2}^2 = \dfrac{[L_1(C+C_1) - L_2(C + C_2)] \pm \sqrt{[L_1(C + C_1) - L_2(C + C_2)]^2+4L_1L_2C^2}}{2L_1L_2(C_1C_2 + CC_1 + CC_2)}$.
%\item \begin{multline*}
%	\omega_{1,2}^2 = \frac12\left( \frac{1}{LC_1} + \frac{1}{LC_2} +\frac{1}{L_1C_1} +\frac{1}{L_2C_2} \right)\pm \\ \pm\frac12\sqrt{\left[\frac{1}{C_1}\left( \frac{1}{L} + \frac{1}{L_1}\right) - \frac{1}{C_2} \left( \frac{1}{L} + \frac{1}{L_2}\right) \right]^2 + \frac{4}{L^2C_1C_2}}.
%\end{multline*}
%\end{enumerate}
%\end{solution}
%\end{problem}
%
%%%=========================================================
%\begin{figure}[h!]\centering
%	%---------------------------------------------------------
%	\subfloat[\label{Bugaenko235}]{%
%		\begin{tikzpicture}
%			\coordinate (A) at (0,2);
%			\coordinate (B) at (0,0);
%			\node [contact] (C) at (2,2) {};
%			\node [contact](D) at (2,0) {};
%			\node [contact](E) at (4,2) {};
%			\node [contact](F) at (4,0) {};
%			\coordinate (G) at (6,2) {};
%			\coordinate (H) at (6,0) {};
%			\draw (B) to [inductor={info'={$L_1$}}] (A)
%			(D) to [capacitor={info'={$C_1$}}] (C)
%			(E) to [capacitor={info={$C_2$}}] (F)
%			(G) to [inductor={info'={$L_2$}}] (H)
%			(C) to [capacitor={info={$C$}}] (E)
%			(A) -- (C) (E) -- (G)
%			(B) -- (D) -- (F) -- (H)
%			;
%		\end{tikzpicture}
%	}
%	\quad
%	%---------------------------------------------------------
%	\subfloat[\label{Bugaenko236}]{
%			\begin{tikzpicture}
%				\coordinate (A) at (0,2);
%				\coordinate (B) at (0,0);
%				\node [contact] (C) at (2,2) {};
%				\node [contact](D) at (2,0) {};
%				\node [contact](E) at (4,2) {};
%				\node [contact](F) at (4,0) {};
%				\coordinate (G) at (6,2) {};
%				\coordinate (H) at (6,0) {};
%				\draw (B) to [inductor={info'={$L_1$}}] (A)
%				(D) to [capacitor={info'={$C_1$}}] (C)
%				(E) to [capacitor={info={$C_2$}}] (F)
%				(G) to [inductor={info'={$L_2$}}] (H)
%				(C) to [inductor={info={$L$}}] (E)
%				(A) -- (C) (E) -- (G)
%				(B) -- (D) -- (F) -- (H)
%				;
%			\end{tikzpicture}
%	}
%	%---------------------------------------------------------
%	\caption{До задачі~\ref{prb:Bugaenko}}
%\end{figure}

%% --------------------------------------------------------
\section{Вимушені коливання в електричних колах. Резонанс}
%% --------------------------------------------------------

\begin{Theory}

	У змінному струмі різні елементи кола впливають на амплітуду й фазу струму. Фаза --- це зсув у часі між коливаннями струму і напруги.

	Синусоїдальний струм:

	\begin{equation*}
		i(t) = I_0 \sin(\omega t),\quad \text{де}\ \omega = 2\pi f.
	\end{equation*}

	%% --------------------------------------------------------
	\subsection*{Поведінка елементів кола}
	%% --------------------------------------------------------

	\begin{itemize}

		\item Резистор (активний опір)

		      \begin{center}
			      \begin{tikzpicture}
				      \draw (-1,0) node [contact] {} to[resistor={info={$R$}}] (1,0)  node [contact] {};
			      \end{tikzpicture}
		      \end{center}

		      Напруга синхронна зі струмом:
		      \begin{equation}
			      u_R(t) = I_0 R \sin(\omega t)
		      \end{equation}
		      Зсув фаз: $\phi = 0^\circ$

		\item Конденсатор (ємнісний опір)

		      \begin{center}
			      \begin{tikzpicture}
				      \draw (-1,0) node [contact] {} to[capacitor={info={$C$}}] (1,0)  node [contact] {};
			      \end{tikzpicture}
		      \end{center}

		      Напруга відстає від струму на $90^\circ$:
		      \begin{equation}
			      u_C(t) = \frac{I_0}{\omega C} \sin\left(\omega t - \frac{\pi}{2}\right)
		      \end{equation}
		      Ємнісний опір: $X_C = \frac{1}{\omega C}$ (зменшується зі збільшенням частоти)

		\item Котушка індуктивності (індуктивний опір)

		      \begin{center}
			      \begin{tikzpicture}
				      \draw (-1,0) node [contact] {} to[inductor={info={$L$}}] (1,0)  node [contact] {};
			      \end{tikzpicture}
		      \end{center}

		      Напруга випереджає струм на $90^\circ$:
		      \begin{equation}
			      u_L(t) = I_0 \omega L \sin\left(\omega t + \frac{\pi}{2}\right)
		      \end{equation}
		      Індуктивний опір: $X_L = \omega L$ (збільшується зі збільшенням частоти)

	\end{itemize}

	%% --------------------------------------------------------
	\subsection*{Послідовне RLC-коло}
	%% --------------------------------------------------------

	\begin{center}
		\begin{tikzpicture}
			\draw (0,0) node [contact] {} to[resistor={info={$R$}}] ++(2,0)
			node [contact] {}
			to[capacitor={info={$C$}}]  ++(2,0)
			node [contact] {}
			to[inductor={info={$L$}}]  ++(2,0)
			node [contact] {} ;
		\end{tikzpicture}
	\end{center}

	Зсув фаз:
	\begin{equation}
		\tg\phi = \frac{X_L - X_C}{R} = \frac{\omega L - 1/(\omega C)}{R}
	\end{equation}

	Амплітуда напруги:
	\begin{equation}
		U_0 = I_0 \sqrt{R^2 + (X_L - X_C)^2}
	\end{equation}

	\textbf{Характер кола:}
	\begin{itemize}
		\item $X_L > X_C$ --- індуктивний характер ($\varphi > 0$)
		\item $X_L < X_C$ --- ємнісний характер ($\varphi < 0$)
		\item $X_L = X_C$ --- резонанс ($\varphi = 0$)
	\end{itemize}

	%% --------------------------------------------------------
	\subsection*{Ефективні (діючі) значення}
	%% --------------------------------------------------------

	\begin{align}
		I_{e\kern-2.2pt f\kern-3pt f} & = \frac{I_0}{\sqrt{2}} \\
		U_{e\kern-2.2pt f\kern-3pt f} & = \frac{U_0}{\sqrt{2}}
	\end{align}

	Приклад: побутова напруга $220$ В --- це ефективне значення, амплітуда становить $311$ В.

	%% --------------------------------------------------------
	\subsection*{Потужність змінного струму}
	%% --------------------------------------------------------

	Середня потужність:
	\begin{equation}
		\left\langle P\right\rangle = I_{e\kern-2.2pt f\kern-3pt f} U_{e\kern-2.2pt f\kern-3pt f} \cos\varphi = \frac{1}{2} I_0 U_0 \cos\varphi
	\end{equation}

	Коефіцієнт потужності $\cos\phi$ показує частку корисної енергії.

	%% --------------------------------------------------------
	\subsection*{Метод комплексних амплітуд}
	%% --------------------------------------------------------

	Струм і напруга у комплексній формі:
	\begin{align}
		\tilde{I} & = I_0 e^{i\omega t}          \\
		\tilde{U} & = U_0 e^{i(\omega t + \phi)}
	\end{align}

	\begin{itemize}

		\item Комплексні імпеданси

		      \begin{center}
			      \begin{tblr}{cccc}
				      \toprule
				      \textbf{Елемент} & \textbf{Імпеданс}     & \textbf{Характер} & \textbf{Фаза} \\
				      \midrule
				      Резистор         & $R$                   & дійсний           & $0°$          \\
				      Конденсатор      & $\frac{1}{i\omega C}$ & уявний            & $-90°$        \\
				      Котушка          & $i\omega L$           & уявний            & $+90°$        \\
				      \bottomrule
			      \end{tblr}
		      \end{center}

		\item Правила Кірхгофа для змінного струму

		      \begin{itemize}
			      \item Перше правило (струми у вузлі):
			            \begin{equation}
				            \sum_{i} \tilde{I}_i = 0
			            \end{equation}

			      \item Друге правило (напруги в контурі):
			            \begin{equation}
				            \sum_{i} \tilde{I}_i \tilde{Z}_i = \sum_{j} \tilde{\mathcal{E}}_j
			            \end{equation}
		      \end{itemize}

		\item Розрахунок кіл змінного струму

		      \begin{itemize}
			      \item Закон Ома:
			            \begin{equation}
				            I_0 = \frac{U_0}{|\tilde{Z}|}
			            \end{equation}

			      \item  Модуль імпедансу:
			            \begin{equation}
				            |\tilde{Z}| = \sqrt{[\operatorname{Re}(\tilde{Z})]^2 + [\operatorname{Im}(\tilde{Z})]^2}
			            \end{equation}

			      \item  Зсув фаз:
			            \begin{equation}
				            \tan\varphi = \frac{\operatorname{Im}(\tilde{Z})}{\operatorname{Re}(\tilde{Z})}
			            \end{equation}
		      \end{itemize}

	\end{itemize}

	%% --------------------------------------------------------
	\subsection*{Потужність через імпеданс}
	%% --------------------------------------------------------

	Середня потужність:
	\begin{equation}
	\langle P \rangle = \frac{1}{2} U_0 I_0 \cos\varphi = U_{\eff} I_{{e\kern-2.2pt f\kern-3pt f}} \cos\phi
	\end{equation}

	Через імпеданс:
	\begin{equation}
	\langle P \rangle = \frac{1}{2} I_0^2 \operatorname{Re}(\tilde{Z}) = I_{{e\kern-2.2pt f\kern-3pt f}}^2 \operatorname{Re}(\tilde{Z})
	\end{equation}

	Потужність виділяється лише на активному опорі.

	%% --------------------------------------------------------
	\subsection*{Резонанс}
	%% --------------------------------------------------------

	Умова резонансу:
	\begin{equation}
	\operatorname{Im}(\tilde{Z}) = 0 \quad \Rightarrow \quad X_L = X_C
	\end{equation}

	Резонансна частота для RLC-кола:
	\begin{equation}
	\omega_0 = \frac{1}{\sqrt{LC}}
	\end{equation}

	\textbf{При резонансі:}
	\begin{itemize}
	\item Імпеданс мінімальний: $|\tilde{Z}| = R$
	\item Струм максимальний: $I_0 = U_0/R$
	\item Зсув фаз: $\varphi = 0°$
	\item Потужність максимальна
	\end{itemize}

\end{Theory}

%% --------------------------------------------------------
\subsection*{Резонанс. Правила Кірхгофа для змінного струму}
%% --------------------------------------------------------

%=========================================================
\begin{problem}%
Показати, що при резонансі в контура з малим загасанням амплітуда
напруги на конденсаторі визначається формулою:
\begin{equation*}
	V_{0_C} = QV_0,
\end{equation*}
де $Q$ --- добротність контура і $V_0$ --- амплітуда напруги генератора підключеного до контура.
\end{problem}

%=========================================================
\begin{problem}%
Послідовне $RLС$-коло під'єднано до генератора синусоїдальної напруги, частоту якої можна змінювати при незмінній амплітуді. Знайти частоту $\omega$, при якій амплітуда сили струму в контурі буде максимальною.
\end{problem}

%=========================================================
\begin{problem}%
Доведіть, що в послідовному $RLC$-колі напруга на $LC$-ділянці при резонансі дорівнює нулю. Котушку та конденсатор вважати ідеальними.
\end{problem}

%=========================================================
\begin{problem}%
Конденсатор $100$~мкФ та резистор $30$~Ом з'єднані послідовно й увімкнені в освітлювальну мережу. Знайти імпеданс кола $Z$ та зсув фаз $\phi$ між струмом у колі та напругою мережі. Випереджає чи відстає за фазою струм від напруги в мережі? Зобразити приблизну векторну діаграму кола і показати на ній фазовий кут.
\begin{solution}
	$Z = 43.71$~Ом, $\phi = 46.7^\circ$.
\end{solution}
\end{problem}

%=========================================================
\begin{problem}%
В освітлювальну мережу з діючою напругою $220$~В і частотою $50$~Гц
паралельно увімкнули котушку з індуктивністю $95.5$~мГн і резистор $40$~Ом. Побудувати векторну діаграму та знайти імпеданс кола.
\begin{solution}
	$24$~Ом.
\end{solution}
\end{problem}

%=========================================================
\begin{problem}\label{prb:current_ampermeter}
Частота змінної напруги у контурі pjбраженому на рис.~\ref{current_ampermeter} дорівнює $\omega = \frac{1}{\sqrt{LC}}$. Як залежить струм, що тече через амперметр, від опору реостата $R$?
\begin{solution}
	$I = \frac{\EMF}{i\omega L}$, тобто струм через амперметр не залежить від опору реостата.
\end{solution}
\end{problem}
%---------------------------------------------------------
\begin{figure}[h!]\centering
	\begin{tikzpicture}
		\draw (0,-1) coordinate (START) to [ac source={rotate=-90,info={left:$\mathcal{E}$}}] (0,1) to [inductor={info={$L$}}] ++(3,0) -- ++(0,-2) coordinate(D) to [capacitor={info'={$C$}}] (START) -- cycle;
		\draw (START) -- ++(0,-1.5) -- ++(1,0) coordinate (A);
		\draw[-latex] (A) -- ++(0,0.5) coordinate (B);
		\draw ([xshift=-0.3cm]B) node[above right = 4pt] {$R$} rectangle ([shift={(0.3,0.2)}]B) coordinate (C);
		\coordinate (R) at ([yshift=-0.1cm]C);
		\draw (R)  to[ampermeter] ({D|-R}) -- (D);
	\end{tikzpicture}
	\caption{До задачі~\ref{prb:current_ampermeter}}
	\label{current_ampermeter}
\end{figure}
%---------------------------------------------------------

%=========================================================
\begin{problem}\label{prb:LCL}% ФЛФ 22.2
Знайдіть струм в колі, яке зображене на рис.~\ref{LCL}, якщо ЕРС джерела змінюється за законом $\EMF = \EMF_0\cos\omega t$. Чому буде дорівнювати струм, якщо котушки матимуть взаємну індуктивність $M$?
\begin{solution}
	$I = \EMF_0\frac{1 - \omega^2CL}{\omega L(2 - \omega^2CL)}\sin\omega t$, $I = \EMF_0\frac{1 - \omega^2CL}{\omega(2 - \omega^2CL)}\sin\omega t$.
\end{solution}
\end{problem}

%=========================================================
\begin{problem}\label{prb:RCRL}% ФЛФ 22.8
В колі, зображеному на рис.~\ref{RCRL}, діє джерело змінної ЕРС  $\EMF = \EMF_0\cos\omega t$. При яких умовах струм в колі не залежить від частоти? Знайдіть різницю фаз між ЕРС та напругою на кінцях $RC$-пари за цих умов.
\begin{solution}
	$RC = \nfrac{L}{R}$, $\tg\phi = \frac{\omega L}{R}$.
\end{solution}
\end{problem}

%=========================================================
\begin{problem}\label{prb:LCLcircuit}
Знайти умову, при якій в схемі рис.~\ref{LCLcircuit} струм, що протікає через деяке навантаження $Z$, не буде залежати від нього.
\begin{solution}
	$\omega^2LC = 1$.
\end{solution}
\end{problem}

%=========================================================
\begin{figure}[h!]\centering
	%---------------------------------------------------------
	\begin{minipage}[t]{0.3\linewidth}\centering
		\begin{tikzpicture}
			\draw (0,-2) coordinate (START) to [ac source={rotate=-90,info={left:$\EMF$}}] (0,2) -- ++(2,0) to [inductor={info={$L$}}] ++(0,-2) node [contact] (A) {};
			\draw (A) -- ++(-0.5,0) to [capacitor={info'={$C$}}] ++(0,-1) -- ++(0.5,0) node [contact] (B) {}
			(A) -- ++(0.5,0)  to [inductor={info={$L$}}] ++(0,-1) -- (B) -- +(0,-1) -- (START);
		\end{tikzpicture}
		\caption{До задачі~\ref{prb:LCL}}
		\label{LCL}
	\end{minipage}
	%---------------------------------------------------------
	\begin{minipage}[t]{0.3\linewidth}\centering
		\begin{tikzpicture}
			\draw (0,-2) coordinate (START) to [ac source={rotate=-90,info={left:$\EMF$}}] (0,2) -- ++(2,0) -- +(0,-0.3) node [contact] (C) {};
			\draw (C) -- ++(-0.5,0) to [resistor={info'={$R$}}] ++(0,-1.5) -- ++(0.5,0) node [contact] (D) {}
			(C) -- ++(0.5,0)  to [capacitor={info={$C$}}] ++(0,-1.5) -- (D) -- +(0,-0.4) node [contact] (E) {} ;
			\draw (E) -- ++(-0.5,0) to [resistor={info'={$R$}}] ++(0,-1.5) -- ++(0.5,0) node [contact] (B) {}
			(E) -- ++(0.5,0)  to [inductor={info={$L$}}]  ++(0,-1.5) -- (B) -- ({B|-START}) -- (START);
		\end{tikzpicture}
		\caption{До задачі~\ref{prb:RCRL}}
		\label{RCRL}
	\end{minipage}
	%---------------------------------------------------------
	\begin{minipage}[t]{0.3\linewidth}\centering
		\begin{tikzpicture}
			\draw (0,-2) coordinate (START) to [ac source={rotate=-90,info={left:$\EMF$}}] (0,2) -- ++(2,0) to [inductor={info={$L$}}] ++(0,-2) node [contact] (A) {};
			\draw (A) -- ++(-0.5,0) to [capacitor={info'={$C$}}] ++(0,-1.5) -- ++(0.5,0) node [contact] (B) {}
			(A) -- ++(0.5,0)  to [resistor={info={$Z$}}] ++(0,-1.5) -- (B) -- +(0,-0.5) -- (START);
		\end{tikzpicture}
		\caption{До задачі~\ref{prb:LCLcircuit}}
		\label{LCLcircuit}
	\end{minipage}
	%---------------------------------------------------------
\end{figure}
%=========================================================

%=========================================================
\begin{problem}\label{prb:KRS3.249}
В колі, зображеному на рис.~\ref{KRS3.249}, діє змінна ЕРС $\EMF = \EMF_0\cos^2\omega t$. Визначити струми $I_1$ та $I$, якщо параметри кола задовольняють співвідношенню $\omega^2 = \frac{1}{4LC}$.
\begin{solution}
	$I_1 = \frac{\EMF_0}{R}\cos 2\omega t$, $I_1 = \frac{\EMF_0}{2(R + R_1)}$.
\end{solution}
\end{problem}

\begin{problem}\label{prb:filcontr}
Параметри $R$ і $C$ схеми (рис.~~\ref{filcontr}) задані. При якій частоті $\omega$ вихідна напруга $V_\text{out}$ буде знаходиться в фазі із вхідною напругою $V_\text{in}$? Яким при цьому буде відношення амплітуди напруг $V_\text{out}$ та $V_\text{in}$?
\end{problem}

%=========================================================
\begin{figure}[h!]\centering
	%---------------------------------------------------------
	\begin{minipage}[t]{0.45\linewidth}\centering
		\begin{tikzpicture}
			\draw
			(0,0) node (A) [ocontact] {} to [capacitor={info={$C$}}] (0.7,0) to [resistor={info={$R$}}] (2.5,0) node (B) [contact] {} -- (4,0) node [ocontact] {}
			(0,-2) node (A1) [ocontact] {} -- +(2.5,0) node (B1) [contact] {} -- +(4,0)  node [ocontact] {}
			(B) to [resistor={info'={$R$}}] (B1)
			(3.5,0) node (D1) [contact] {} to [capacitor={info'={$C$}}] +(0,-2) node (D2) [contact] {}
			;
			\node at (0,-1) {$V_\text{in}$};
			\node at (4.5,-1) {$V_\text{out}$};
		\end{tikzpicture}
		\caption{До задачі~\ref{prb:filcontr}}
		\label{filcontr}
	\end{minipage}
	%---------------------------------------------------------
	\begin{minipage}[t]{0.45\linewidth}\centering
		\begin{tikzpicture}
			\draw (0,-1.5) coordinate (START) to [ac source={rotate=-90,info={left:$\EMF$}}] (0,1.5) to [resistor={info'={$R$}}] ++(2,0) node [contact] (A) {}  to [resistor={info={$R_1$}}] ++(0,-1.5) to [inductor={info={$L_1$}}] ++(0,-1.5) node [contact] (B) {} -- (START)
			(A) -- ++(2,0) to [inductor={info'={$L$}}] ++(0,-1.5) to [resistor={info'={$R$}}] ++(0,-1.5) -- (B)
			;
		\end{tikzpicture}
		\caption{До задачі~\ref{prb:KRS3.249}}
		\label{KRS3.249}
	\end{minipage}
	%---------------------------------------------------------
\end{figure}
%=========================================================

%% --------------------------------------------------------
\subsection*{Енергія та потужність в колах змінного струму}
%% --------------------------------------------------------

%=========================================================
\begin{problem}%
Сполучені послідовно котушка з індуктивністю $100$~мГн і резистор
$R_0 = 20$~Ом підключені до генератора з напругою $100$~В і частотою $400$~рад/с. Знайти активний опір котушки, якщо діюча сила струму в колі дорівнює $2$~А.
\begin{solution}
	$R = 10$~Ом.
\end{solution}
\end{problem}

%=========================================================
\begin{problem}%
На послідовне коло, що складається з конденсатора $40$~мкФ, котушки
індуктивності $1.0$~мГн та резистора $25$~Ом, подано від генератора змінну напругу із діючим значенням $2.0$~В і коловою частотою $5\cdot 10^3$~c${}^{-1}$. Знайти амплітуду струму, споживану колом потужність, і зсув фаз між коливаннями струму та     напруги генератора.
\begin{solution}
	$I_0 = 13$~мА, $P = 160$~мВт, $\delta\phi = 0$.
\end{solution}
\end{problem}

%=========================================================
\begin{problem}%
На з'єднані послідовно конденсатор $200$~мкФ та резистор $15.2$~Ом подано діючу напругу $220$~В промислової частоти $50$~Гц. Знайти діюче значення
струму в колі та споживану ним потужність.
\begin{solution}
	$I_{\eff} = 10$~А, $P = 1.52$~кВт.
\end{solution}
\end{problem}

%=========================================================
\begin{problem}%
Активний опір коливального контура $R = 0.33$~Ом. Яку потужність
споживає контур, якщо в ньому відбуваються незгасаючі коливання з амплітудою сили струму $І_0 = 30$~мА.
\begin{solution}
	$0.15$~мВт.
\end{solution}
\end{problem}

%=========================================================
\begin{problem}%
В коливальний контур послідовно включена змінна ЕРС. Обчислити
добротність контура, якщо при резонансі напруга на конденсаторі в $\eta$ разів більша, ніж на джерелі.
\begin{solution}
	$Q = \sqrt{\eta^2 - \frac14}$.
\end{solution}
\end{problem}

\begin{problem}%
Коливальний контур складається з конденсатора ємності $100$~пФ і котушки з індуктивністю $80$~мкГн та активним опором 0,5 Ом. Визначити потужність, яку споживає контур, якщо в ньому підтримуються власні незгасаючі коливання з амплітудою напруги на конденсаторі $V_0 = 4$~В.
\begin{solution}
	$5$~мкВт.
\end{solution}
\end{problem}

%=========================================================
\begin{problem}%
Коливальний контур з малим загасанням має індуктивність $L$ і ємність
$C$. Для підтримання в ньому незагасаючих коливань з амплітудою струму $I_0$ витрачається потужність $P$. Знайти добротність контура.
\begin{solution}
	$Q = \frac{I_0^2}{2P}\sqrt{\frac{L}{C}}$.
\end{solution}
\end{problem}

%=========================================================
\begin{problem}%
Коливальний контур з малим опором складається з котушки індуктивності $L$ і конденсатора $C$. Для підтримання в ньому незагасаючих коливань з
амплітудою напруги на конденсаторі $V_0$ витрачається потужність $P$. Знайти добротність контура.
\begin{solution}
	$Q = \frac{V_0^2}{2P}\sqrt{\frac{C}{L}}$.
\end{solution}
\end{problem}

%=========================================================
\begin{problem}%
На з'єднані послідовно резистор $R = 0.5$~Ом, котушку індуктивності
$L = 4.0$~мГн і конденсатор $C = 200$~мкФ подано змінну напругу з діючим значенням $11.2$~В і частотою $1000$~рад/с. Знайти діючу напругу на кожному елементі кола.
\begin{solution}
	$V_R = 5$~В, $V_L = 40$~В, $V_C = 50$~В.
\end{solution}
\end{problem}

%% --------------------------------------------------------
\subsection*{Мостові схеми}
%% --------------------------------------------------------

%=========================================================
\begin{problem}\label{prb:KRS3.242}
При якому співвідношенні між параметрами моста, зображеного на рис.~\ref{KRS3.242}, напруга на його виході знаходиться в фазі з вхідною напругою $V = V_0\cos\omega t$. Визначити при цьому амплітуду напруги на виході.
\begin{solution}
	$\omega L = R$, $V_{out} = V_{in} \left( \frac13 - \frac{R_1}{R_1 + R_2}\right) $.
\end{solution}
\end{problem}

%=========================================================
\begin{problem}\label{prb:KRS3.244}
Знайти умови,  при яких міст зображений на рис.~\ref{KRS3.244}, буде збалансований (тобто $V_{\mathrm{out}} = 0$ при подачі на його
вхід періодичної напруги $V_{\mathrm{in}}(t)$ будь-якої форми.
\begin{solution}
	$R_1C = RC_1$.
\end{solution}
\end{problem}

%=========================================================
\begin{figure}[h!]\centering
	%---------------------------------------------------------
	\begin{minipage}[t]{0.45\linewidth}\centering
		\begin{tikzpicture}
			\draw (0,2) node [ocontact]  {}  -- (4,2) node [contact] (A) {}
			to [resistor={info'={$R$}}, label distance = -5pt] ++(-1,-1) to [inductor={info={$L$}, rotate=180}, label distance = -5pt] ++(-1,-1) coordinate (B)
			-- ++(0.5,-0.5) coordinate (A1) {} -- ++(45:0.5) --  ++(-45:0) to [resistor={info'={$R$}}] ++ (-45:1.5) -- ++ (-135:0.5)  coordinate (B1) -- (4,-2) node [contact] (C) {}

			(A1) -- ++(-135:0.5) to [inductor={info'={$L$}}] ++(-45:1.5) -- (B1)

			(C) to [resistor={info'={$R_2$}}] ++(2,2) coordinate (D)
			to [resistor={info'={$R_1$}}] ++(-2,2)
			(C) -- +(-4,0) node [ocontact] {}
			(B) -- +(1.5,0) node [ocontact] {}
			(D) -- +(-1.5,0) node [ocontact] {}
			;
			\draw (-0.25,0) sin +(.1,.1) cos ++(.1,-0.1) sin ++(.1,-0.1) cos ++(.1,0.1);
			\draw ([xshift=1.8cm]B) sin +(.1,.1) cos ++(.1,-0.1) sin ++(.1,-0.1) cos ++(.1,0.1);
		\end{tikzpicture}
		\caption{До задачі~\ref{prb:KRS3.242}}
		\label{KRS3.242}
	\end{minipage}
	%---------------------------------------------------------
	\begin{minipage}[t]{0.45\linewidth}\centering
		\begin{tikzpicture}
			\node [contact] (A) at (4,2) {};
			\node [contact] (C) at (4,-2) {};
			\coordinate (B) at (2,0);
			\coordinate (D) at (6,0);
			\draw (0,2) node [ocontact]  {}  -- (A) {}
			to [resistor={info'={$R$}}, label distance = -5pt] ++(-1,-1) to [capacitor={info={$C$}, rotate=180}, label distance = -5pt] (B)

			(B) to [resistor={info'={$R_1$}}] (C)

			(C) -- ++(45:0.6) coordinate (A1) -- ++(135:0.5)  to [capacitor={info'={$C$}}] ++(45:1.5) -- ++(-45:0.5) coordinate (B1) -- (D)

			(A1) --  ++(-45:0.5) to [resistor={info'={$R$}}] ++(45:1.5) -- (B1)

			(D) to [capacitor={info'={$C_1$}}] ++(-2,2)
			(C) -- +(-4,0) node [ocontact] {}
			(B) -- +(1.5,0) node [ocontact] {}
			(D) -- +(-1.5,0) node [ocontact] {}
			;
			\draw (-0.25,0) sin +(.1,.1) cos ++(.1,-0.1) sin ++(.1,-0.1) cos ++(.1,0.1);
		\end{tikzpicture}
		\caption{До задачі~\ref{prb:KRS3.244}}
		\label{KRS3.244}
	\end{minipage}
	%---------------------------------------------------------
\end{figure}
%=========================================================

%=========================================================
\begin{problem}
До джерела із діючим значенням напруги $U = 100$~В приєднали реальну котушку, імпеданс якої $Z = 50$~Ом, а індуктивний опір $X_L = 30$~Ом. Яка потужність виділятиметься в колі. Знайдіть зсув фаз між струмом та напругою.
\begin{solution}
	$P = 160$~Вт, $\phi \approx 37^\circ$.
\end{solution}
\end{problem}

%=========================================================
\begin{problem}\label{prb:Infinite_chain_impedance}
Знайдіть імпеданс нескінченного кола, показаного на рис.~\ref{Infinite_chain_impedance}. Розгляньте випадок коли $Z_1 = i\omega L$, $Z_2 = \frac{1}{i\omega C}$, за яких частот коло не буде споживати потужність від джерела?
\begin{solution}
	Імпеданс кола $Z = \sqrt{\left( \frac{Z_1}{2}\right)^2 + Z_1Z_2}$.
	$Z = \sqrt{\frac{L}{C} - \frac{\omega^2L^2}{4}}$. При $\omega > \nfrac{2}{\sqrt{LC}}$.
\end{solution}
\end{problem}
%---------------------------------------------------------
\begin{figure}[h!]\centering
	\begin{tikzpicture}
		\draw (0,0) node[ocontact] {} to [resistor={info={$Z_1/2$}}] (2,0) node[contact] (A1) {} to [resistor={info={$Z_1$}}] (4,0) node[contact] (A2) {}  to [resistor={info={$Z_1$}}] (6,0) node[contact] (A3) {};
		\draw (0,-2) node[ocontact] {} -- (2,-2) node[contact] (B1) {} -- (4,-2) node[contact] (B2) {} -- (6,-2) node[contact] (B3) {};
		\draw (A1) to [resistor={info={$Z_2$}}] (B1) (A2) to [resistor={info={$Z_2$}}] (B2) (A3) to [resistor={info={$Z_2$}}] (B3);
		\draw [dashed] (A3) -- (7.5,0)
		(B3) -- (7.5,-2);
		\node[left] at (0,0) {$A$};
		\node[left] at (0,-2) {$B$};
		\node[] at (7.5,-1) {$\infty$};
	\end{tikzpicture}
	\caption{До задачі~\ref{prb:Infinite_chain_impedance}}
	\label{Infinite_chain_impedance}
\end{figure}
%---------------------------------------------------------

%=========================================================
\begin{problem}
Знайти силу взаємодії на одиницю довжини між паралельними провідниками, по яким течуть гармонічні струми  амплітудами $I_1$ та $I_2$ зі зсувом фаз $\phi$, що знаходяться на відстані $r$ один від одного.
\begin{solution}
	$F = \frac{2I_1I_2}{2cr}\cos\phi$.
\end{solution}
\end{problem}

\subsection*{Електричні фільтри}

%=========================================================
\begin{problem}\label{prb:AxFilters}
Знайти частоти пропускання фільтрів, зображених на рис.~\ref{Ax1}, \ref{Ax2} та ~\ref{Ax3}.
\begin{solution}
	\begin{enumerate*}[label=\alph*)]
		\item $\omega \le \sqrt{\frac{2}{LC}}$,
		\item $\omega \ge \sqrt{\frac{1}{2LC}}$,
		\item $\Omega_2 < \omega < \Omega_1$, де $\Omega_1^2 = \nfrac{1}{LC} $, $\Omega_2^2 = \frac15 \left( \Omega_1^2 + 4\Omega^2\right)  $, $\Omega = \nfrac{1}{4LC}$.
	\end{enumerate*}
\end{solution}
\end{problem}
%---------------------------------------------------------
\begin{figure}[h!]\centering
	\subfloat[\label{Ax1}]{
		\begin{tikzpicture}
			\draw
			(0,0) node (A) [ocontact] {} to [inductor={info={$L$}}] (2,0) node (B) [contact] {} to [inductor={info={$L$}}] (4,0) node [ocontact] {}
			(0,-2) node (A1) [ocontact] {} -- +(2,0) node (B1) [contact] {} -- +(4,0)  node [ocontact] {}
			(B) to [capacitor={info={$C$}}] (B1);
		\end{tikzpicture}
	}
	\qquad
	\subfloat[\label{Ax2}]{
		\begin{tikzpicture}
			\draw
			(0,0) node (A) [ocontact] {} to [capacitor={info={$C$}}] (2,0) node (B) [contact] {} to [capacitor={info={$C$}}] (4,0) node [ocontact] {}
			(0,-2) node (A1) [ocontact] {} -- +(2,0) node (B1) [contact] {} -- +(4,0)  node [ocontact] {}
			(B) to [inductor={info={$L$}}] (B1);
		\end{tikzpicture}
	}
	\qquad
	\subfloat[\label{Ax3}]{
		\begin{tikzpicture}
			\draw
			(0,0) node (A) [ocontact] {} to [inductor={info={$L$}}] (1,0) to [capacitor={info={$C$}}] (2,0) node (B) [contact] {} to [capacitor={info={$C$}}] (3,0) to [inductor={info={$L$}}] (4,0)node [ocontact] {}
			(0,-2) node (A1) [ocontact] {} -- +(2,0) node (B1) [contact] {} -- +(4,0)  node [ocontact] {}
			(B) to [inductor={info={$2L$}}] ([yshift=-1cm]B) to [capacitor={info={$2C$}}] (B1);
		\end{tikzpicture}
	}
	\caption{До задачі~\ref{prb:AxFilters}}
\end{figure}
%---------------------------------------------------------

%=========================================================
%\begin{problem}\label{prb:Irodov3.147}
%    На рис.~\ref{Irodov3.147} зображено найпростішу схему фільтра. На вхід подається змінна напруга $V_{\mathrm{in}} = V_0(1 + \cos\omega t)$. Знайти
%    \begin{enumerate*}[label=\alph*)]
%        \item вихідну напругу;
%        \item значення $RC$, при якому амплітуда змінної складової напруги на виході буде в $7$ разів менша за сталу складову, якщо частота $314$~с$^{-1}$.
%    \end{enumerate*}
%    Які частоти відсікає цей фільтр: низькі чи високі?
%    %---------------------------------------------------------
%    \begin{figure}[h!]\centering
%        \begin{tikzpicture}
%            \draw
%                 (0,0) node (A) [ocontact] {} to [resistor={info={$R$}}] +(2,0) node (B) [contact] {} -- +(4,0) node [ocontact] {}
%                 (0,-2) node (A1) [ocontact] {} -- +(2,0) node (B1) [contact] {} -- +(4,0)  node [ocontact] {}
%                 (B) to [capacitor={info={$C$}}] (B1);
%                 \node at (0,-1){$V$};
%        \end{tikzpicture}
%    \caption{До задачі~\ref{prb:Irodov3.147}}
%    \label{Irodov3.147}
%    \end{figure}
%    %---------------------------------------------------------
%\begin{solution}
%\begin{enumerate*}[label=\alph*)]
%    \item $V_{\mathrm{out}} = V_0\left( 1 + \frac{1}{\sqrt{1 + (\omega RC)^2}}\right)\cos(\omega t - \phi) $, $\tg\phi = \omega RC$,
%    \item $RC \approx 22$~мс.
%\end{enumerate*}
%\end{solution}
%\end{problem}

\Closesolutionfile{answer}

