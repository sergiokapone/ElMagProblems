% !TeX program = lualatex
% !TeX encoding = utf8
% !TeX spellcheck = uk_UA
% !BIB program = biber

\documentclass[biblatex]{ProblemBook}
\usepackage{lipsum}
\title{Електрика та магнетизм}
\def\subtitle{збірник задач}
\def\authors{С.~М.~Пономаренко}
%========================================================================================================
%
\input{CoverPage}%					      Титульна сторінка
%
%========================================================================================================

\begin{document}
\CoverPage
\maketitle
\makeinfopage
\Opensolutionfile{answer}[Answers]

%
%
%
%\chapter{Chapter}
%
%%=========================================================
%\begin{problem}\label{prb:potential_6_2013-030-036_1}
%    У просторі між обкладками незаміщений плоский конденсатор вносять металеву пластину, що має заряд $Q$, так що між пластиною і обкладками конденсатора залишаються зазори $d_1$ та $d_2$ (рис.~\ref{potential_6_2013-030-036_1}). Площі пластини і обкладок конденсатора однакові і дорівнюють $S$. Визначити різницю потенціалів між обкладинками конденсатора. Крайовими ефектами нехтувати.
%\begin{solution}
%	$\Delta\phi = \frac{2\pi Q}{S} (d_2 - d_1)$.
%\end{solution}
%\end{problem}
%
%%=========================================================
%\begin{problem}\label{prb:potential_6_2013-030-036_4}
%    У плоскому конденсаторі на ліву обкладку поміщають заряд $+Q_1$, а на праву $+Q_2$. Всередину конденсатора паралельно обкладкам поміщають незаряджену металеву пластину (рис. 10). Які заряди будуть індуковані на лівій і правій поверхнях пластини, якщо $Q_2>Q_1$?
%\begin{solution}
%	$Q_{L} = \frac{Q_2 - Q_1}{2}$, $Q_{R} = -Q_{L}$.
%\end{solution}
%\end{problem}
%
%%=========================================================
%\begin{figure}[h!]\centering
%%---------------------------------------------------------
%\begin{minipage}[t]{0.45\linewidth}\centering
%	\begin{tikzpicture}
%		\draw[ultra thick] (0,0) -- +(0,2);
%		\draw[ultra thick] (3,0) -- +(0,2);
%		\draw[fill=gray!50, thick] (0.75,0) rectangle (1.75,2) ;
%		\node[above] at (1.5,2) {$+Q$};
%		\draw[latex-latex] (0,0.5) --node[above] {$d_1$} +(0.75,0);
%		\draw[latex-latex] (1.75,0.5) --node[above] {$d_2$} +(1.25,0);
%	\end{tikzpicture}
%\caption{До задачі~\ref{prb:potential_6_2013-030-036_1}}
%\label{potential_6_2013-030-036_1}
%\end{minipage}
%%---------------------------------------------------------
%\begin{minipage}[t]{0.45\linewidth}\centering
%	\begin{tikzpicture}
%		\draw[ultra thick] (0,0) -- +(0,2);
%		\draw[ultra thick] (3,0) -- +(0,2);
%		\draw[fill=gray!50, thick] (0.75,0) rectangle (1.75,2) ;
%		\node[above] at (0,2) {$+Q_1$};
%		\node[above] at (3,2) {$+Q_2$};
%		\draw[latex-latex] (0,0.5) --node[above] {$d_1$} +(0.75,0);
%		\draw[latex-latex] (1.75,0.5) --node[above] {$d_2$} +(1.25,0);
%	\end{tikzpicture}
%\caption{До задачі~\ref{prb:potential_6_2013-030-036_4}}
%\label{potential_6_2013-030-036_4}
%\end{minipage}
%%---------------------------------------------------------
%\end{figure}
%%=========================================================
%
%
%%=========================================================
%\begin{problem}\label{prb:potential_6_2013-030-036_2}
%    Три однакові нерухомі металеві пластини розташовані в повітрі на відстані $d_1$ і $d_2$ ($d_2> d_1$) одна від одної. Площа кожної з пластин дорівнює $S$ (рис.~\ref{potential_6_2013-030-036_2}). На середню пластину поміщають позитивний заряд $Q$. Пластини $1$ і $3$ не заряджені і підключені через ключ до резистору з невідомим опором, відмінним від нуля. Які заряди встановляться на пластинах $1$ і $3$ після довгого часу після замикання ключа? Крайовими ефектами нехтувати.
%\begin{solution}
%	$Q_1 = -\frac{Q}{2}\frac{d_2 - d_1}{d_2 + d_1}$, $Q_2 = - Q_1$.
%\end{solution}
%\end{problem}
%
%%=========================================================
%\begin{problem}\label{prb:potential_6_2013-030-036_5}
%    Три однакові металеві пластини розташовані в повітрі на рівних відстанях $d$ одна від одної. Площа кожної пластини дорівнює $S$ (рис.~\ref{potential_6_2013-030-036_5}). На пластині $1$ знаходиться позитивний заряд $Q$. Пластини $2$ і $3$ не заряджені і підключені через ключ до резистору з невідомим відмінним від нуля опором. Які набої встановляться на пластинах $2$ і $3$ після великого часу після замикання ключа?
%\begin{solution}
%	$Q_2 = -\frac{Q}{2}$, $Q_3 = \frac{Q}{2}$.
%\end{solution}
%\end{problem}
%
%%=========================================================
%\begin{figure}[h!]\centering
%%---------------------------------------------------------
%\begin{minipage}[t]{0.45\linewidth}\centering
%	\begin{tikzpicture}
%		\draw[ultra thick] (0,0) -- +(0,2) node[above] {$1$};
%		\draw[ultra thick] (1,0) -- node[pos=0.8, right] {$+Q$} +(0,2) node[above] {$2$};
%		\draw[ultra thick] (3,0) -- +(0,2) node[above] {$3$};
%		\draw[latex-latex] (0,0.5) --node[above] {$d_1$} +(1,0);
%		\draw[latex-latex] (1,0.5) --node[above] {$d_2$} +(2 ,0);
%		\draw (0,1) -- ++(-0.5,0) -- ++(0,-2) to[make contact] ++(1,0) to [resistor] ++(3,0) -- ++(0,2) -- ++(-0.5,0);
%	\end{tikzpicture}
%\caption{До задачі~\ref{prb:potential_6_2013-030-036_2}}
%\label{potential_6_2013-030-036_2}
%\end{minipage}
%%---------------------------------------------------------
%\begin{minipage}[t]{0.45\linewidth}\centering
%	\begin{tikzpicture}
%		\draw[ultra thick] (0,0) -- node[left] {$+Q$}+(0,2) node[above] {$1$};
%		\draw[ultra thick] (1,0) -- +(0,2) node[above] {$2$};
%		\draw[ultra thick] (2,0) -- +(0,2) node[above] {$3$};
%		\draw[latex-latex] (0,0.5) --node[above] {$d$} +(1,0);
%		\draw[latex-latex] (1,0.5) --node[above] {$d$} +(1 ,0);
%		\draw (1,0) -- ++(0,-0.25) -- ++(-1.5,0) -- ++(0,-0.75) to[make contact] ++(1,0) to [resistor] ++(2,0) -- ++(0,2) -- ++(-0.5,0);
%	\end{tikzpicture}
%\caption{До задачі~\ref{prb:potential_6_2013-030-036_5}}
%\label{potential_6_2013-030-036_5}
%\end{minipage}
%%---------------------------------------------------------
%\end{figure}
%%=========================================================
%
%%=========================================================
%\begin{problem}\label{prb:potential_6_2013-030-036_3}
%    Три тонкі незаряджені металеві пластини площею $S$ кожна розташовані на відстанях $d$ одна від одної, причому $d$ багато менше розмірів пластин. До пластин $2$ і $3$ під'єднали батарею з ЕРС $\EMF$ (рис.~\ref{potential_6_2013-030-036_3}). Пластині $1$ надали позитивний заряд $q_0$. Визначити заряд, який встановився на пластинах $2$ і $3$.
%\begin{solution}
%	$Q_2 = -\frac{q_0}{2} -\frac{\EMF S}{4\pi d}$, $Q_3 = \frac{q_0}{2} + \frac{\EMF S}{4\pi d}$.
%\end{solution}
%\end{problem}
%
%%=========================================================
%\begin{problem}\label{prb:potential_6_2013-030-036_6}
%    Три однакові нерухомі металеві пластини розташовані в повітрі на різних відстанях $d_1$ і $d_2$ ($d_2> d_1$) один від одного (рис.~\ref{potential_6_2013-030-036_6}). На середній пластині $2$ знаходиться позитивний заряд $Q$. спочатку не заряджені пластини $1$ і $3$ підключають через ключ до батареї з ЕРС $\EMF$ (рис. 12). Визначити встановилися заряди пластин $1$ і $3$ після замикання ключа.
%\begin{solution}
%	$Q_1 = -\frac{2\EMF S + 4\pi Q(d_2 - d_1)}{8\pi(d_1 + d_2)}$, $Q_3 = -Q_1$.
%\end{solution}
%\end{problem}
%%=========================================================
%\begin{figure}[h!]\centering
%%---------------------------------------------------------
%\begin{minipage}[t]{0.45\linewidth}\centering
%	\begin{tikzpicture}
%		\draw[ultra thick] (0,0) -- node[left] {$q_0$}+(0,2) node[above] {$1$};
%		\draw[ultra thick] (1,0) -- +(0,2) node[above] {$2$};
%		\draw[ultra thick] (2,0) -- +(0,2) node[above] {$3$};
%		\draw[latex-latex] (0,0.5) --node[above] {$d$} +(1,0);
%		\draw[latex-latex] (1,0.5) --node[above] {$d$} +(1 ,0);
%		\draw (1,0) -- ++(0,-1) to [battery={info'={$\EMF$},rotate=180}] ++(2,0) -- ++(0,2) -- ++(-1,0);
%	\end{tikzpicture}
%\caption{До задачі~\ref{prb:potential_6_2013-030-036_3}}
%\label{potential_6_2013-030-036_3}
%\end{minipage}
%%---------------------------------------------------------
%\begin{minipage}[t]{0.45\linewidth}\centering
%	\begin{tikzpicture}
%		\draw[ultra thick] (0,0) -- +(0,2) node[above] {$1$};
%		\draw[ultra thick] (1,0) -- node[pos=0.8, right] {$+Q$} +(0,2) node[above] {$2$};
%		\draw[ultra thick] (3,0) -- +(0,2) node[above] {$3$};
%		\draw[latex-latex] (0,0.5) --node[above] {$d_1$} +(1,0);
%		\draw[latex-latex] (1,0.5) --node[above] {$d_2$} +(2 ,0);
%		\draw (0,1) -- ++(-0.5,0) -- ++(0,-2) to[make contact] ++(1,0) to [battery={info'={$\EMF$}, rotate=180}] ++(3,0) -- ++(0,2) -- ++(-0.5,0);
%	\end{tikzpicture}
%\caption{До задачі~\ref{prb:potential_6_2013-030-036_6}}
%\label{potential_6_2013-030-036_6}
%\end{minipage}
%%---------------------------------------------------------
%\end{figure}
%%=========================================================

	\begin{center}
		\begin{tikzpicture}[scale=0.7]
			\begin{axis}[   axis y line = left,
					axis x line = bottom,
					grid = both,
					ylabel={$B$, кГс},
					xlabel={$H$, Е},
					xtick = {0,5,...,25},
					ytick = {0,5,...,25},
				]

				\addplot[thick, red, domain={0:25}, samples=100, name path = curve] {25*sqrt(1-((x-25)^2/25^2))};
				\addplot[blue, domain={0:20}, name path = line] {20 - x};
				\path [name intersections={of=line and curve, by=P}];
				\fill[red] (P) circle (0.05);
			\end{axis}
		\end{tikzpicture}
	\end{center}
%---------------------------------------------------------


\Closesolutionfile{answer}
\newpage
\input{Answers}
\end{document}






