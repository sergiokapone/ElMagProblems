% !TeX program = lualatex
% !TeX encoding = utf8
% !TeX spellcheck = uk_UA
% !TeX root =../EMProblems.tex

\introtrue
\chapter*{Передмова}

Дисципліна <<Електрика та магнетизм>> курсу <<Загальна фізика>>, який вивчається студентами Навчально-наукового ф-технічного інституту КПІ імені Ігоря Сікорського на другому курсі, входить до циклу базової підготовки студентів, що навчаються за спеціальністю E6 <<Прикладна фізика та наноматеріали>>. На вивчення дисципліни відведено один семестр, практичні заняття проводяться один раз на тиждень. Велика частина програмного матеріалу, пов'язана з умінням розв'язування конкретних задач. Вироблення умінь, навичок і методів розв'язку величезного числа задач, звичайно, не може бути реалізована тільки за рахунок годин, відведених на практичні заняття, і має на увазі велику самостійну роботу студента.

Назви розділів та підрозділів посібника відповідають робочій програмі курсу. До кожного з них подано короткий теоретичний матеріал, який містить основні формули, необхідні для розв'язування задач. В кінці збірника міститься довідковий матеріал та перелік використаної літератури.

Задачі збірника мають в основному підвищений рівень складності і розташовано у такій послідовності, що для розв'язання наступних задач можна використовувати розв'язок попередніх. Практично до кожної задачі  в кінці збірника наведена відповідь, деякі задачі містять детальних розв'язок.

Всі формули в основному тексті і відповідях наведені в гаусовій системі одиниць (СГС). Система одиниць СГС була вибрана з огляду на традиції викладання фізичних дисциплін в фізико-технічному інституті, крім того, при вивчені саме теорії електромагнітного поля вона має безперечні переваги (див. наприклад~\cite[\S85]{Siv3}). За необхідності, студенту буде не складно перевести формули із системи СГС в СІ і навпаки, крім того, в Додатку~\ref{SItoGauss} наведено алгоритм таких дій. Зрозуміло, що задачі на освоєння законів постійного та змінного струмів, все ж таки зручніше розв'язувати в системі СІ, тому у відповідних розділах використовується саме ця система одиниць. Хоч більшість задач передбачають отримання розв'язку у вигляді формули, однак є й такі задачі, які вимагають числової відповіді, а тому вихідні дані і числові відповіді надані з урахуванням точності значень відповідних величин і правил дій над наближеними числами. В кінці збірника дана таблиця основних фізичних констант та інші довідкові таблиці.

Задачі даного збірника щорічно пропонуються студентам при складанні письмового екзамену, який передбачено навчальним планом. Зрозуміло, що багато задач було взято з джерел, які перелічені в списку літератури, оскільки методика викладання дисципліни <<Електрика та магнетизм>> на протязі століть вже є здебільшого завершеною і багато вже винайдених іншими вченими задач є ключовими для розуміння предмета. Однак в збірнику є й велика частина авторських задач, ідеї яких сформувались у автора на основі власного досвіду викладання предмету.

Дана версія посібника є електронним виданням, тому для зручності користування ним передбачена система навігації у вигляді гіперпосилань та бічної панелі змісту.

\vspace*{4em}

\begin{flushright}\Annabelle
	С.~М.~Пономаренко
\end{flushright}
\introfalse





