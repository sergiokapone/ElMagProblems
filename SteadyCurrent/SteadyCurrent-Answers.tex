\protect \section *{\nameref *{SteadyCurrent}}
\begin{Solution}{2.{4}}
	$j_{1n} = j_{2n}$, $\frac{j_{1\tau}}{\lambda_1} = \frac{j_{2\tau}}{\lambda_2}$,
	$\frac{\tg{\alpha_1}}{\tg{\alpha_2}} = \frac{\lambda_1}{\lambda_2}$.
\end{Solution}
\begin{Solution}{2.{5}}
	Для постійних струмів:
	\begin{equation}
		\divg\vect{j} = 0,
	\end{equation}
	використовуючи закон Ома в диференціальній формі:
	\begin{equation}
		\divg\vect{j} = \divg(\lambda\Efield) = \lambda\divg\Efield + \Efield\cdot\vect{\nabla}\lambda = 0.
	\end{equation}
	З теореми Гауса в диференціальній формі $\divg\Efield = 4\pi\rho$, бачимо, що густина об'ємних зарядів в провіднику дорівнює:
	\begin{equation}
		\rho = - \frac{\vect{j}\cdot\vect{\nabla}\lambda}{4\pi\lambda^2}.
	\end{equation}

	Якщо провідник однорідний ($\lambda = \const$), то об'ємного заряду не виникає.
\end{Solution}
\begin{Solution}{2.{8}}
	$ I = \frac{4\pi VC}{\rho \epsilon} $
\end{Solution}
\begin{Solution}{2.{9}}
	\begin{enumerate*}[label=\alph*)]
		\item  $j = V\left( \frac{d_1}{\lambda_1} + \frac{d_2}{\lambda_2}\right)^{-1}$;
		\item $\sigma_1  = \frac{V}{4\pi} \frac{\epsilon_1}{\lambda_1} \left( \frac{d_1}{\lambda_1} + \frac{d_2}{\lambda_2}\right)^{-1}$, $\sigma_2  = \frac{\epsilon_2\lambda_1}{\epsilon_1\lambda_2} \sigma_1$.
		\item $\sigma = \frac{V}{4\pi} \frac{\epsilon_2\lambda_1 - \epsilon_1\lambda_2}{d_1\lambda_2 + d_2\lambda_1}$;
		\item $\sigma' = - \frac{V}{4\pi} \frac{(\epsilon_2 - 1)\lambda_1 - (\epsilon_1 - 1)\lambda_2}{d_1\lambda_2 + d_2\lambda_1}$;
		\item $R = \frac1S\left( \frac{d_1}{\lambda_1} + \frac{d_2}{\lambda_2}\right) $.
	\end{enumerate*}
\end{Solution}
\begin{Solution}{2.{10}}
	$q  = \frac{I}{4\pi} \left( \frac{\epsilon_2}{\lambda_2} - \frac{\epsilon_1}{\lambda_1}\right) $.
\end{Solution}
\begin{Solution}{2.{11}}
	$\rho = \frac{4\pi\tau}{\epsilon\ln\frac32}$
\end{Solution}
\begin{Solution}{2.{12}}
	$R = \frac{\ln\left( \frac{R_2}{R_1} \right)}{2\pi l\lambda}$, $C = \frac{\epsilon}{\lambda R}$.
\end{Solution}
\begin{Solution}{2.{13}}
	$R = \frac{\epsilon\rho\left( r_2 - r_1 \right)}{4\pi r_1r_2}$.
\end{Solution}
\begin{Solution}{2.{14}}
	Згідно теореми Гауса, заряд однієї із оболонок дорівнює:
	\[
		Q = \frac{1}{4\pi} \oiint\limits_S \Efield d\vect{S}.
	\]

	З іншого боку, із означення ємності системи двох провідників:
	\[
		Q = CV,
	\]
	де $V$~-- різниця потенціалів між провідниками, $C$~-- їх взаємна ємність.

	Звідси,

	\[
		V = \frac{1}{4\pi C} \oiint\limits_S \Efield d\vect{S}.
	\]

	Струм між цими провідниками можна знайти як:
	\[
		I =  \oiint\limits_S \vect{j} d\vect{S} = \lambda \oiint\limits_S \Efield d\vect{S}.
	\]

	Отже, опір середовища знаходимо з закону Ома:

	\[
		R = \frac{V}{I} = \frac{1}{4\pi\lambda C}.
	\]
	В системі SI остання формула матиме вигляд:
	\[
		R = \frac{\epsilon_0}{\lambda C}.
	\]
\end{Solution}
\begin{Solution}{2.{15}}
	$R = \frac{\rho}{4\pi}\left( \frac{1}{r_1} + \frac{1}{r_2} \right)$, $C  = \frac{r_1r_2}{r_1 + r_2}$.
\end{Solution}
\begin{Solution}{2.{16}}
	$q_1 = \frac{\EMF r_2}{r_2/r_1 + \lambda_1/\lambda_2}$, $q_2 = \frac{\EMF r_1}{r_1/r_2 + \lambda_2/\lambda_1}$.
\end{Solution}
\begin{Solution}{2.{17}}
%---------------------------------------------------------
\begin{center}
    \begin{tikzpicture}[every node/.style={scale=0.8}]

			\node  [gray!50, ground, minimum width=7cm,yshift=-0.17cm] (floor1) at (-1,0) {};
			\draw (floor1.north west) -- (floor1.north east);
			\node  [gray!50, ground, minimum width=7cm,yshift=-0.17cm] (floor2) at (1,0) {};
			\draw (floor2.north west) -- (floor2.north east);
			\draw [ball color=gray!20] (180:1) arc (180:360:1) -- cycle;
			\draw [-latex] (0,0) -- node [right] {$\nfrac{D}{2}$}(240:1);
			% ----------------- Рисування ЛЕП -----------------
			\draw [gray!50, thick] (-0.5,0) -- (0,  4) foreach \x in  {0,...,10}  {coordinate[pos=0.1*\x]      ({A\x})}
			-- (0.5,0) foreach \x in  {0,...,10}  {coordinate[pos=1 - 0.1*\x]  ({B\x})};
			\draw[gray!50] (B0) foreach \x in {1,...,10} {to (\ifodd\x A\x\else B\x\fi)}
			(A0) foreach \x in {1,...,10} {to (\ifodd\x B\x\else A\x\fi)};
			% ----------------- Рисування графіка -----------------
	        \draw[-latex] (0,0) -- (0,5) node[right] {$\phi$};
			\draw[-latex] (-4,0) -- (4,0) node[below] {$r$};
			\draw [blue] (180:2) arc (180:360:2);
			\draw [blue] (180:2.4) arc (180:360:2.4);
			\draw[ultra thick, blue] (0,{3.5*exp(-0.6*1)+0.5}) -- +(1,0);
			\draw[ultra thick, domain=1:4,smooth,variable=\x,blue, name path = curve] plot ({\x},{3.5*exp(-0.6*\x)+0.5});

			\draw[ultra thick, gray!50] (0,{3.5*exp(-0.6*1)+0.5}) -- +(-1,0);
			\draw[ultra thick, domain=-4:-1,smooth,variable=\x,gray!50] plot ({\x},{3.5*exp(0.6*\x)+0.5});

			\path[name path global= line] (2.4,0) -- ++(0,4);
			\draw[dashed, name intersections = {of = curve and line, by = A}] (2.4,0) -- (A);
			\draw[dashed] (A) -- (A-|0,0) node[below left=-5pt] {$\phi_2$};

			\path[name path global= line] (2,0) -- ++(0,4);и
			\draw[dashed, name intersections = {of = curve and line, by = B}] (2,0) -- (B);
			\draw[dashed] (B) -- (B-|0,0) node[above left=-5pt] {$\phi_1$};

			\draw (B-|0,0) -- +(-2,0) coordinate (P1);
			\draw (A-|0,0) -- +(-2,0) coordinate (P2);
			\draw[latex-] ([xshift=0.2cm]P1) -- ([shift={(0.2,0.5)}]P1) node[above] {$U_\text{крокова} = \phi_1 - \phi_2$};
			\draw[latex-] ([xshift=0.2cm]P2) -- ([shift={(0.2,-0.5)}]P2) ;


			\draw (2,0) -- +(0,-2);
			\draw (2.4,0) -- +(0,-2);
			\draw[latex-latex] (2,-1.9) -- node[below] {$\ell$} (2.4, -1.9);
			% ----------------- Рисування людини -----------------
			\coordinate (M) at (2.2,0.25);
			\draw[red, thick] (2,0) -- (M) (2.4,0) -- (M);
			\draw[red, thick] (M) -- ([yshift=0.3cm]M) coordinate (HEAD);
			\draw[red, thick] ([yshift=0.1cm]HEAD) circle (0.1);
			\draw[red, thick] (HEAD) -- +(-45:0.3);
			\draw[red, thick] (HEAD) -- +(-135:0.3);
    \end{tikzpicture}
\end{center}
%---------------------------------------------------------
Опір заземлення $R = \frac{1}{\pi\lambda D} \approx 8$~Ом.

Потенціал заземленої кулі спадає обернено пропорційно відстані:
\[
	\phi = \frac{\phi_0 D}{2}\frac{1}{r},
\]
а крокова напруга визначається за формулою (див. рис.) $U_\text{крокова} = \phi_1 - \phi_2$, тобто
\[
	U_\text{крокова} = \frac{\phi_0D}{2}\frac{\ell}{r^2} \approx 246\, \text{В},
\]
де було враховано, що $r = \frac{r + (r + \ell)}{2}$.
\end{Solution}
\begin{Solution}{2.{18}}
	$\vect{p} = \frac{\lambda_i - \lambda_e}{\lambda_i + 2\lambda_e}R^3\Efield_0,$

	$\vect{j} =
		\begin{cases}
			\frac{3\lambda_e}{\lambda_i + 2\lambda_e} \vect{j}_0,                                                                                                    & r \le R \\
			\vect{j}_0  + \frac{\lambda_i - \lambda_e}{\lambda_i + 2\lambda_e} \left( \frac{3R^3}{r^5}(\vect{j}_0\cdot\vect{r}) - \frac{R^3}{r^3}\vect{j}_0\right) , & r > R,
		\end{cases}
	$, де $r$~-- відстань від центру порожнини, $\sigma = \frac{3}{4\pi} \frac{\lambda_i - \lambda_e}{\lambda_i + 2\lambda_e} \frac{\Efield_0\vect{r}}{R}.$
\end{Solution}
\begin{Solution}{2.{19}}
	Задача аналогічна до \ref{sphere:current_in_media}, де треба покласти $\lambda_i = 0$.
	%
	%	$\vect{j} =
	%	\begin{cases}
	%		\vect{j}_0\left( 1 + \frac{R^3}{2r^3}\right)  - \frac{3R^3(\vect{j}_0\cdot\vect{R})\vect{R}}{2r^5}, & \quad r>R, \\
	%		0, & \quad r<R
	%	\end{cases}
	%	$, де $r$~-- відстань від центру порожнини.
\end{Solution}
\begin{Solution}{2.{20}}
	$U = \frac{\EMF_1+\EMF_2+\EMF_3}{r_1 + r_2 + r_3} r_1 - \EMF_1$.
\end{Solution}
\begin{Solution}{2.{21}}
	$R = \frac{R_1R_2}{R_1 + R_2}$, $I = 0$ при $\frac{\EMF_1}{\EMF_2} = \frac{R_1}{R_2}$.
\end{Solution}
\begin{Solution}{2.{22}}
	$\phi_A - \phi_B = \frac{\EMF_1 - \EMF_2}{R_1 + R_2} R_2 - \EMF_1 =-4$~В.
\end{Solution}
\begin{Solution}{2.{23}}
	Відносна похибка
	\[
		\frac{\Delta R_x}{R_x} = \frac{(R_x + R_3)(R_1 + R_2)}{R_2 R_x}\frac{\Delta V}{V}
	\]
	де $V$~-- напруга на містку і $\Delta V$~-- різниця потенціалів на клемах гальванометра при зміні $R_x$ на $R_x + \Delta R_x$. У стані рівноваги $R_x / R_3 = R_1 / R_2 = b$, тому множник при $\frac{\Delta V}{V}$ дорівнюватиме $(1+b)^2/b$, він має мінімум при $b = 1$.
\end{Solution}
\begin{Solution}{2.{25}}
	$R_{AB} = 2R (1 + \sqrt{3})$.
\end{Solution}
\begin{Solution}{2.{26}}
	$I = \frac{V}{2}\frac{R_2 - R_1}{R_1R_2} = 1.0$~А. Струм тече від $1$ до $2$.
\end{Solution}
\begin{Solution}{2.{27}}
	$R_{AB}  = R_2 \frac{R_2 + 3R_1}{R_1 + 3R_2}$.
\end{Solution}
\begin{Solution}{2.{31}}
	$Q = \frac{4\pi\lambda}{\epsilon}\sum\limits_i^n q_i\phi_i$.
\end{Solution}
\begin{Solution}{2.{32}}
	$P = \frac{2\pi l U}{\rho\ln\frac{R_2}{R_1}}$.
\end{Solution}
\begin{Solution}{2.{33}}
	$Q = \frac{q^2}{2\epsilon}\left( \frac{1}{R_1} - \frac{1}{R_2}\right) $.
\end{Solution}
\begin{Solution}{2.{34}}
	$P_1  = \frac{2V^2}{d} \frac{\rho_1}{(\rho_1 + \rho_2)^2}S$, $P_2  = \frac{2V^2}{d} \frac{\rho_2}{(\rho_1 + \rho_2)^2}S$.
\end{Solution}
\begin{Solution}{2.{35}}
	Напруженість поля в середині кулі визначається формулою
	\[
		\Efield_i = \frac{3}{\lambda_i + 2\lambda_e} \vect{j}_0,
	\]
	де $\vect{j}_0$~-- густина струму далеко від кулі.
	Потужність, що виділяється в одиниці об'єму згідно закону Джоуля-Ленца $w = \vect{j}_i\cdot\Efield_i$, отже, використовуючи закон Ома $ \vect{j}_i = \lambda_i\Efield_i$:
	\[
		w = \frac{9\lambda_i}{\left( \lambda_i + 2\lambda_e\right)^2} \vect{j}_i^2.
	\]
	Звідки випливає, що максимум потужності буде при $\lambda_i = 2\lambda_e$.
\end{Solution}
\begin{Solution}{2.{36}}
	$d = \frac{\rho I_0^2}{6\pi Q} \tau \ln\frac{R_2}{R_1}$.
\end{Solution}
\begin{Solution}{2.{37}}
	$P_{\max} = \frac{\pi\EMF^2 d}{\rho \left( 1 + \ln\left( \frac{b}{d}\right)  \right) }.$
\end{Solution}
\begin{Solution}{2.{38}}
	На кожному з опорів виділиться $Q = \frac{CV_0^2}{6}$.
\end{Solution}
\begin{Solution}{2.{39}}
	$Q = \frac25 C\EMF^2$.
\end{Solution}
\begin{Solution}{2.{40}}
	$P_2 = \frac{(\sqrt13 - 1)^3}{8}P_1 \approx 13.3$~Вт.
\end{Solution}
