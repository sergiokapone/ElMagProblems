% !TeX program = lualatex
% !TeX encoding = utf8
% !TeX spellcheck = uk_UA
% !TeX root =../EMProblems.tex
\clearpage
\section{Сфери в елетричному та магнітному полях}

\begin{center}
	Позначення
\end{center}
\begin{enumerate}\small
	\item $\Efield_0$ та $\Bfield_0$~-- поля на далекій відстані від куль.
	\item $R$~-- радіус кулі.
	\item Індекс $e$ відноситься до оточуючого середовища, $i$~-- до матеріалу кулі.
	\item Формули для діелектриків можна замінити на формули для металів, якщо покласти $\frac{1}{\epsilon} \to 0$.
	\item Формули для магнетиків можна замінити на формули для надпровідників, якщо покласти $\mu = 0$.
\end{enumerate}

\begin{center}\small
	\begin{longtable}{p{0.5\textwidth}p{0.45\textwidth}}
		%============================================================================================
		\multicolumn{2}{c}{\cellcolor{themecolorlight}\bfseries\color{white} Електростатичне поле однорідно поляризованої кулі. Задача~\ref{prb:Field_of_Dielectric_Sphere}}                                                       \\
		\begin{center}
			Поле вектора $\Efield$
		\end{center}

		\begin{center}
			\begin{tikzpicture}[scale=0.8, rotate=180]
				\clip (-3.5,-8.1) rectangle (3.5,8.1);
				\foreach \i [evaluate=\i as \j using abs(\i)] in {-40,-12,-8,-6,...,8,12,40} {
						\ifnum\i<0\def\domain{1:179}\else\def\domain{179:1}\fi
						%\ifnum\j>12\def\position{0.05}\else\def\position{0.1}\fi
						\draw [color=red,
							samples=200,
							domain=\domain,
							decoration={markings, mark=at position 0.02 with {\arrow{latex'}}},
							decoration={markings, mark=at position 0.1 with {\arrow{latex'}}},
							decoration={markings, mark=at position 0.5 with {\arrow{latex'}}},
							decoration={markings, mark=at position 0.9 with {\arrow{latex'}}},
							decoration={markings, mark=at position 0.98 with {\arrow{latex'}}},
							postaction={decorate}
						] plot (xy polar cs:angle=\x,radius= {\i*(sin(\x))^2});
					}
				\fill[gray!10, draw=blue] (0,0) circle (1);

				\foreach \i in {-1.8,-1.4,...,2} {
						\pgfmathparse{abs(\i/(1+1))}
						\ifdim\pgfmathresult cm < 1 cm
							\draw[red, decoration={markings, mark=at position 0.5 with {\arrow{latex'}}}, postaction={decorate}]
							({180-asin(\i/(1+1))}:1) -- ({asin(\i/(1+1))}:1);
						\fi
					}
				% ============================ знаки зарядів =================================
				\foreach \i in {70,30,15,0}{
						\node at (\i:0.9) {\tiny $-$};
						\node at (-\i:0.9) {\tiny $-$};
						\node at ({180-\i}:0.9) {\tiny $+$};
						\node at ({180+\i}:0.9) {\tiny $+$};
					}
			\end{tikzpicture}
		\end{center}
		 &
		Вектор поляризації в середині кулі:
		\[\vect{P} = \const.\]

		Дипольний момент кулі:
		\[
			\vect{p} = \frac{4\pi}{3}R^3\vect{P}.
		\]

		Напруженість електричного поля:
		\[
			\Efield =
			\begin{cases}
				- \frac{4\pi}{3}\vect{P},                                                                                   & r \le R \\
				\frac{4\pi}{3}R^3\left( \frac{\vect{P}}{r^3} + \frac{3\left(\vect{P}\vect{r}\right)\vect{r} }{r^5}\right) , & r > R.
			\end{cases}
		\]

		Зв'язані заряди на поверхні:

		\[\sigma' = \vect{P}\cdot\vect{r}.\]

		\\
		\pagebreak
		\multicolumn{2}{c}{\cellcolor{themecolorlight}\bfseries\color{white} Діелектрична куля в однорідному електричному полі. Задача~\ref{sphere:Dielectric_in_Dielectric}}         \\


		\begin{center}
			Поле вектора $\Efield$ ($\epsilon_i > \epsilon_e$)
		\end{center}

		\begin{center}
			\input{Spheres/Efield0.tikz}
		\end{center}

		\begin{center}
			Поле вектора $\Dfield$ ($\epsilon_i > \epsilon_e$)
		\end{center}

		\begin{center}
			\input{Spheres/Dfield0.tikz}
		\end{center}

		\begin{center}
			Поле вектора $\Efield$ ($\epsilon_i < \epsilon_e$)
		\end{center}

		\begin{center}
			\begin{tikzpicture}
% ============================ ��������� ===================================
\pgfmathsetmacro{\step}{0.4}
\pgfmathsetmacro{\ea}{1}
\pgfmathsetmacro{\eb}{5}
\pgfmathsetmacro{\shape}{(\ea-\eb)/(\ea+2*\eb)}
% ============================== ������� ===================================
\draw [
    raw gnuplot, red,
    ] plot[id=curve, raw gnuplot] function {
            set isosamples 55, 55;
            set contour base;
            set cntrparam levels incremental -1.6,\step,1.6;
            %set style data lines;
            unset  surface;
            splot [-4:4] [-2.2:2.2] (y*(1+\shape/(x**2 + y**2))) ;
            };
% ================================ ���� ======================================
\fill[gray!20, draw=blue, thick] (0,0) circle (1.01);
% ======================= ������ �� ����� ==================================
\foreach \i in {-1.6,-1.2,...,1.8} {
\draw[red, -latex', rotate around = {{-asin(\i/(3 +\shape/3))}:({asin(\i/(3 +\shape/3))}:3)}] ({asin(\i/(3 +\shape/3))}:3) -- ({asin(\i/(3 +\shape/3))}:3.1);
\draw[red, -latex',rotate around = {{180-asin(\i/(3 +\shape/3))}:({180+asin(\i/(3 +\shape/3))}:3)} ] ({180+asin(\i/(3 +\shape/3))}:3) -- ({180+asin(\i/(3 +\shape/3))}:3.1);
}
% ======================= ���� � �������� ��� ==============================
\foreach \i in {-1.6,-1.4,...,1.8} {
\pgfmathparse{abs(\i/(1+\shape))}
\ifdim\pgfmathresult cm < 1 cm
\draw[red, decoration={markings, mark=at position 0.5 with {\arrow{latex'}}}, postaction={decorate}] 
({180-asin(\i/(1+\shape))}:1) -- ({asin(\i/(1+\shape))}:1);
\fi
}
% ============================ ����� ������ =================================
\foreach \i in {70,30,15,0}{
\node at (\i:0.9) {\tiny $-$};
\node at (-\i:0.9) {\tiny $-$};
\node at ({180-\i}:0.9) {\tiny $+$};
\node at ({180+\i}:0.9) {\tiny $+$};
%==================================
\node at (\i:1.1) {\tiny $+$};
\node at (-\i:1.1) {\tiny $+$};
\node at ({180-\i}:1.1) {\tiny $-$};
\node at ({180+\i}:1.1) {\tiny $-$};
}
\end{tikzpicture}
		\end{center}

		\begin{center}
			Поле вектора $\Dfield$ ($\epsilon_i < \epsilon_e$)
		\end{center}

		\begin{center}
			\input{Spheres/Dfield1.tikz}
		\end{center}

		 &
		Дипольний момент кулі:

		\[\vect{p} = \frac{\epsilon_i - \epsilon_e}{\epsilon_i + 2\epsilon_e}R^3\Efield_0,\]

		{\footnotesize Напрямок дипольного моменту визначається різницею $\epsilon_e - \epsilon_e$.}

		Потенціал кулі:

		\[
			\phi =
			\begin{cases}
				-\frac{3\epsilon_e}{\epsilon_i + 2\epsilon_e}\left( \Efield_0\cdot \vect{r}\right), & r \le R \\
				-\left( \Efield_0\cdot \vect{r}\right) + \frac{\vect{p} \vect{r}}{r^3},             & r > R
			\end{cases}
			,\]

		Поле кулі:

		\[
			\Efield =
			\begin{cases}
				\frac{3\epsilon_e}{\epsilon_i + 2\epsilon_e} \Efield_0,                                 & r \le R \\
				\Efield_0 - \frac{\vect{p}}{r^3} + \frac{3\left(\vect{p}\vect{r}\right)\vect{r} }{r^5}, & r > R,
			\end{cases}
		\]

		Зв'язані заряди на поверхні:

		\[\sigma' = \frac{3}{4\pi} \frac{\epsilon_i - \epsilon_e}{\epsilon_i + 2\epsilon_e} \frac{\Efield_0\vect{r}}{R}.\]


		\\
		%============================================================================================
		\pagebreak
		\multicolumn{2}{c}{\cellcolor{themecolorlight}\bfseries\color{white} Металева сфера в однорідному електричному полі}                                                          \\



		\begin{center}
			Поле вектора $\Efield$
		\end{center}

		\begin{center}
			\begin{tikzpicture}
% ============================ ��������� ===================================
\pgfmathsetmacro{\step}{0.4}
\pgfmathsetmacro{\ea}{5}
\pgfmathsetmacro{\eb}{1}
\pgfmathsetmacro{\shape}{1}
% ============================== ������� ===================================
\draw [
    raw gnuplot, red,
    ] plot[id=curve, raw gnuplot] function {
            set isosamples 55, 55;
            set contour base;
            set cntrparam levels incremental -2.2,\step,2.2;
            %set style data lines;
            unset  surface;
            splot [-4:4] [-2.2:2.2] (y*(1+\shape/(x**2 + y**2))) ;
            };
% ================================ ���� ======================================
\fill[gray!20, draw=blue, thick] (0,0) circle (1.01);
% ======================= ������ �� ����� ==================================
\foreach \i in {-2.2,-1.8,...,2.2} {
\draw[red, -latex', rotate around = {{-asin(\i/(3 +\shape/3))}:({asin(\i/(3 +\shape/3))}:3)}] ({asin(\i/(3 +\shape/3))}:3) -- ({asin(\i/(3 +\shape/3))}:3.1);
\draw[red, -latex',rotate around = {{180-asin(\i/(3 +\shape/3))}:({180+asin(\i/(3 +\shape/3))}:3)} ] ({180+asin(\i/(3 +\shape/3))}:3) -- ({180+asin(\i/(3 +\shape/3))}:3.1);
}
% ============================ ����� ������ =================================
\foreach \i in {70,30,15,0}{
%==================================
\node at (\i:1.1) {\tiny $+$};
\node at (-\i:1.1) {\tiny $+$};
\node at ({180-\i}:1.1) {\tiny $-$};
\node at ({180+\i}:1.1) {\tiny $-$};
}
\end{tikzpicture}
		\end{center}

		 &

		Дипольний момент кулі:

		\[\vect{p} = R^3\Efield_0,\]

		Потенціал кулі:

		\[
			\phi =
			\begin{cases}
				0,                                                                      & r \le R \\
				-\left( \Efield_0\cdot \vect{r}\right) + \frac{\vect{p} \vect{r}}{r^3}, & r > R
			\end{cases}
			,\]

		Поле кулі:

		\[
			\Efield =
			\begin{cases}
				0,                                                                                      & r \le R \\
				\Efield_0 - \frac{\vect{p}}{r^3} + \frac{3\left(\vect{p}\vect{r}\right)\vect{r} }{r^5}, & r > R,
			\end{cases}
		\]

		Вільні заряди на поверхні:

		\[\sigma = \frac{3}{4\pi} \frac{\Efield_0\vect{r}}{R}.\]

		\\
		\multicolumn{2}{c}{\cellcolor{themecolorlight}\bfseries\color{white} Провідна куля в середовищі по якому тече струм. Задача~\ref{sphere:current_in_media}}                    \\


		\begin{center}
			Поле вектора $\vect{j}$ ($\lambda_i > \lambda_e$)
		\end{center}

		\begin{center}
			\input{Spheres/Dfield0.tikz}
		\end{center}

		\begin{center}
			Поле вектора $\vect{j}$ ($\lambda_i < \lambda_e$)
		\end{center}

		\begin{center}
			\input{Spheres/Dfield1.tikz}
		\end{center}

		 &

		Дипольний момент кулі:

		\[\vect{p} = \frac{\lambda_i - \lambda_e}{\lambda_i + 2\lambda_e}R^3\Efield_0,\]

		Поле:
		\[
			\Efield =
			\begin{cases}
				\frac{3\lambda_e}{\lambda_i + 2\lambda_e} \Efield_0,                                    & r \le R \\
				\Efield_0 - \frac{\vect{p}}{r^3} + \frac{3\left(\vect{p}\vect{r}\right)\vect{r} }{r^5}, & r > R,
			\end{cases}
		\]

		Густина струму:

		\[
			\vect{j} =
			\begin{cases}
				\frac{3\lambda_e}{\lambda_i + 2\lambda_e} \vect{j}_0,                                                       & r \le R \\
				\vect{j}_0  + \lambda_e\frac{3\left(\vect{p}\vect{r}\right)\vect{r} }{r^5} - \lambda_e\frac{\vect{p}}{r^3}, & r > R,
			\end{cases}
		\]

		Заряди на поверхні:

		\[\sigma = \frac{3}{4\pi} \frac{\lambda_i - \lambda_e}{\lambda_i + 2\lambda_e} \frac{\Efield_0\vect{r}}{R}.\]
		\\
		\pagebreak
		\multicolumn{2}{c}{\cellcolor{themecolorlight}\bfseries\color{white} Куля з магнетика в однорідному магнітному полі. Задача~\ref{sphere:Magnetic_in_magnetic}}                \\

		\begin{center}
			Поле вектора $\Hfield$ ($\mu_e > \mu_i$)
		\end{center}

		\begin{center}
			\input{Spheres/Hfield.tikz}
		\end{center}

		\begin{center}
			Поле вектора $\Bfield$ ($\mu_e > \mu_i$)
		\end{center}

		\begin{center}
			\begin{tikzpicture}
	% ============================ ��������� ===================================
	\pgfmathsetmacro{\step}{0.4}
	\pgfmathsetmacro{\ea}{5}
	\pgfmathsetmacro{\eb}{1}
	\pgfmathsetmacro{\shape}{(\ea-\eb)/(\ea+2*\eb)}
	% ============================== ������� ===================================
	\draw [
	    raw gnuplot, red,
	    ] plot[id=curve, raw gnuplot] function {
	            set isosamples 55, 55;
	            set contour base;
	            set cntrparam levels incremental -2,\step,2;
	            %set style data lines;
	            unset  surface;
	            splot [-4:4] [-2.2:2.2] (y*(1+\shape/(x**2 + y**2))) ;
	            };
	% ================================ ���� ======================================
	\fill[gray!20, draw=blue, thick] (0,0) circle (1.01);
	% ======================= ������ �� ����� ==================================
	\foreach \i in {-2,-1.6,...,2} {
	\draw[red, -latex', rotate around = {{-asin(\i/(3 +\shape/3))}:({asin(\i/(3 +\shape/3))}:3)}] ({asin(\i/(3 +\shape/3))}:3) -- ({asin(\i/(3 +\shape/3))}:3.1);
	\draw[red, -latex',rotate around = {{180-asin(\i/(3 +\shape/3))}:({180+asin(\i/(3 +\shape/3))}:3)} ] ({180+asin(\i/(3 +\shape/3))}:3) -- ({180+asin(\i/(3 +\shape/3))}:3.1);
	}
	% ======================= ���� � �������� ��� ==============================
	\foreach \i in {-1.6,-1.2,...,1.8} {
	\pgfmathparse{abs(\i/(1+\shape))}
	\ifdim\pgfmathresult cm < 1 cm
	\draw[red, decoration={markings, mark=at position 0.5 with {\arrow{latex'}}}, postaction={decorate}] 
	({180-asin(\i/(1+\shape))}:1) -- ({asin(\i/(1+\shape))}:1);
	\fi
	}
	\end{tikzpicture}
		\end{center}

		 &
		Дипольний момент кулі:

		\[\vect{p}_m = \frac{\mu_i - \mu_e}{\mu_i + 2\mu_e}R^3\Bfield_0,\]

		Поле кулі:

		\[
			\Bfield =
			\begin{cases}
				\frac{3\mu_i}{\mu_i + 2\mu_e} \Bfield_0,                                                    & r \le R \\
				\Bfield_0 - \frac{\vect{p}_m}{r^3} + \frac{3\left(\vect{p}_m\vect{r}\right)\vect{r} }{r^5}, & r > R,
			\end{cases}
		\]

		Густина об'ємних струмів намагнічування $\vect{j}' = 0$.

		Поверхнева густина струмів намагнічування
		\[i = \frac{3c}{4\pi} \frac{\mu_i - \mu_e}{\mu_i + 2\mu_e} \frac{\Bfield_0\vect{r}}{R},\]
		де $\vect{r}$~-- радіус-вектор поверхні провідника.

		\\
		\multicolumn{2}{c}{\cellcolor{themecolorlight}\bfseries\color{white} Надпровідна куля в однорідному магнітному полі $\Bfield_0$. Задача~\ref{sphere:Superconductor_in_field}} \\

		\begin{center}
			Поле вектора $\Bfield$
		\end{center}

		\begin{center}
			\begin{tikzpicture}
\clip (-4,-2) rectangle (4,2);
% ======================= стрілки на лініях ==================================
\foreach \i [evaluate=\i as \x using \i*0.1]in {-12,-10,...,12} {
\ifnum\i=0\relax\else%
\draw[red, -latex', rotate around = {{-asin(\x/(3 -1/3))}:({asin(\x/(3 -1/3))}:3)}] ({asin(\x/(3 -1/3))}:3) -- ({asin(\x/(3 -1/3))}:3.1);

\draw[red, -latex',rotate around = {{180-asin(\x/(3 -1/3))}:({180+asin(\x/(3 -1/3))}:3)} ] ({180+asin(\x/(3 -1/3))}:3) -- ({180+asin(\x/(3 -1/3))}:3.1);
\fi
}
% ================================ куля ======================================
\fill[gray!20, draw=blue, thick] (0,0) circle (1.01);
\foreach \i  in {0.1,0.2,...,0.6} {
\foreach \j in {-1,1} {
\draw [color=red,
		domain=179:1,
		samples=200,
		smooth,
		] plot (xy polar cs:angle=\x,radius= {\j*\i/sin(\x) +\j * sqrt( (\i)^2/(sin(\x)^2) + 1) });
		}
		}
\end{tikzpicture}
		\end{center}
		 &

		Дипольний момент кулі:

		\[\vect{p}_m = - \frac{1}{2}R^3\Bfield_0,\]

		Поле кулі:

		\[
			\Bfield =
			\begin{cases}
				0,                                                                                          & r \le R \\
				\Bfield_0 - \frac{\vect{p}_m}{r^3} + \frac{3\left(\vect{p}_m\vect{r}\right)\vect{r} }{r^5}, & r > R,
			\end{cases}
		\]

		Густина об'ємних струмів намагнічування $\vect{j}' = 0$.

		Поверхнева густина струмів намагнічування
		\[i = -\frac{3c}{8\pi} \frac{\Bfield_0\vect{r}}{R},\]
		де $\vect{r}$~-- радіус-вектор поверхні провідника.

		\\
	\end{longtable}
\end{center}