%arara

%%============================ Compiler Directives =======================%%
%%                                                                        %%
% !TeX program = pdflatex
% !TeX encoding = utf8
% !TeX spellcheck = uk_UA
% !BIB program = biber
%%                                                                        %%
%%============================== Клас документа ==========================%%
%%                                                                        %%
\documentclass[12pt, oneside]{extarticle}
%%                                                                        %%
%%========================== Мови, шрифти та кодування ===================%%
%%  
\usepackage{ifluatex}  
\ifluatex                                                                    
\usepackage{fontspec}
\setsansfont{CMU Sans Serif}%{Arial}
\setmainfont{CMU Serif}%{Times New Roman}
\setmonofont{CMU Typewriter Text}%{Consolas}
\defaultfontfeatures{Ligatures={TeX}}
\usepackage{amsmath}
\usepackage[math-style=TeX]{unicode-math}
\else
\usepackage[utf8]{inputenc}
\usepackage[T2A,T1]{fontenc}
\usepackage{amsmath}
\fi
\usepackage[english, russian, ukrainian]{babel}
%%                                                                        %%
%%============================= Геометрія сторінки =======================%%
%%                                                                        %%
\usepackage[%
a4paper,%
footskip=1cm,%
headsep=0.3cm,% 
top=2cm, %поле сверху
bottom=2cm, %поле снизу
left=1cm, %поле ліворуч
right=1cm, %поле праворуч
]{geometry}                          
%%                                                                        %%
%%========================== Оформлення бібліографії =====================%%
%%                                                                        %%
\usepackage[%
backend=biber, 
style=draft,
babel=other, 
isbn=false, 
url=false,
doi=false]{biblatex}
\newbibmacro{string+doiurlisbn}[1]{%
  \iffieldundef{doi}{%
    \iffieldundef{url}{%
      \iffieldundef{isbn}{%
        \iffieldundef{issn}{%
          #1%
        }{%
          \href{http://books.google.com/books?vid=ISSN\thefield{issn}}{#1}%
        }%
      }{%
        \href{http://books.google.com/books?vid=ISBN\thefield{isbn}}{#1}%
      }%
    }{%
      \href{\thefield{url}}{#1}%
    }%
  }{%
    \href{http://dx.doi.org/\thefield{doi}}{#1}%
  }%
}
\AtEveryBibitem{%
  \clearlist{language}%
}
\DeclareFieldFormat{title}{\usebibmacro{string+doiurlisbn}{#1}}                                                                    
%%============================== Інтерліньяж  ============================%%
%%                                                                        %%
\renewcommand{\baselinestretch}{1}
%-------------------------  Подавление висячих строк  --------------------%%
\clubpenalty =10000
\widowpenalty=10000
%---------------------------------Інтервали-------------------------------%%
\setlength{\parskip}{0.5ex}%
\setlength{\parindent}{2.5em}%
%%                                                                        %%
%%=========================== Математичні пакети і графіка ===============%%
%%                                                                        %%
\usepackage{amsmath}
\usepackage{graphicx}
\usepackage{floatflt}
%%                                                                        %%
%%=========================== Таблиці та кольори =========================%%
%%                                                                        %%
\usepackage{xcolor, colortbl}
\usepackage{longtable}
\usepackage{array,ragged2e}
\setlength\arrayrulewidth{1pt}
\newcounter{magicrownumbers}
\newcommand\rownumber{\stepcounter{magicrownumbers}\arabic{magicrownumbers}}
\usepackage{tikz,pgf}
\usetikzlibrary{external}
\tikzexternalize[prefix=pict/] 
\usetikzlibrary{decorations.text, decorations.markings}
\usetikzlibrary{arrows}  

\def\EMF{\ifmmode\mathcal{E}\else$\mathcal{E}$\fi}
\ifluatex
\newcommand{\vect}[1]{\symbf{#1}}
\else
\newcommand{\vect}[1]{\mathbf{#1}}
\fi
\AtBeginDocument{%
\def\Efield{\vect{E}}
\def\Dfield{\vect{D}}
\def\Bfield{\vect{B}}
\def\Hfield{\vect{H}}
\let\phi\varphi
\let\epsilon\varepsilon}
%%                                                                        %%
%%========================== Гіперпосилення (href) =======================%%
%%                                                                        %%
\usepackage[colorlinks=true,
urlcolor = blue, %Colour for external hyperlinks
linkcolor  = red!60, %Colour of internal links
citecolor  = green!70!black, %Colour of citations
%bookmarks = true,
bookmarksnumbered=true,
unicode,
linktoc = all,
hypertexnames=false,
pdftoolbar=false,
%pdfpagelayout=TwoPageRight,
pdfauthor={Ponomarenko S.M.},
pdfdisplaydoctitle=true,
pdfencoding=auto
]%
{hyperref}
\makeatletter
\AtBeginDocument{
\hypersetup{
pdfinfo={
Title={\@title},
}
}
}
\makeatother
%\addbibresource{\jobname.bib} 
%%                                                                        %%
%%============================ Заголовок та автори =======================%%
%%                                                                        %%
\title{\large\bfseries Сфери в електричному та магнітному полях}   
\date{}                            
%%                                                                        %%
%%========================================================================%%


\begin{document}

\maketitle\thispagestyle{empty}

\begin{center}
	Позначення
\end{center}
\begin{enumerate}
	\item $\Efield_0$ та $\Bfield_0$~-- поля на далекій відстані від куль.
	\item $R$~-- радіус кулі.
	\item Індекс $e$ відноситься до оточуючого середовища, $i$~-- до матеріалу кулі.
	\item Формули для діелектриків можна замінити на формули для металів, якщо покласти $\frac{1}{\epsilon} \to 0$.
	\item Формули для магнетиків можна замінити на формули для надпровідників, якщо покласти $\mu = 0$.
\end{enumerate}

\newpage
\renewcommand{\arraystretch}{1.5}% 

\begin{center}
	\begin{longtable}{|p{0.45\textwidth}|p{0.45\textwidth}|}
		\hline
		\multicolumn{1}{|>{\centering\arraybackslash}m{0.45\textwidth}|}{\cellcolor{cyan!50}\textbf{Вигляд силових ліній}} & \multicolumn{1}{>{\centering\arraybackslash}m{0.45\textwidth}|}{\cellcolor{cyan!50}\textbf{Формули}} \\
		\hline
		\endhead
		%============================================================================================ 
		\multicolumn{2}{|c|}{\cellcolor{cyan!10}Діелектрична куля в однорідному електричному полі}                                                                                                                                \\
		\hline

		\begin{center}
			Поле вектора $\Efield$ ($\epsilon_i > \epsilon_e$)

			\input{Efield0.tikz}

			Поле вектора $\Dfield$ ($\epsilon_i > \epsilon_e$)

			\input{Dfield0.tikz}

			Поле вектора $\Efield$ ($\epsilon_i < \epsilon_e$)

			\begin{tikzpicture}
% ============================ ��������� ===================================
\pgfmathsetmacro{\step}{0.4}
\pgfmathsetmacro{\ea}{1}
\pgfmathsetmacro{\eb}{5}
\pgfmathsetmacro{\shape}{(\ea-\eb)/(\ea+2*\eb)}
% ============================== ������� ===================================
\draw [
    raw gnuplot, red,
    ] plot[id=curve, raw gnuplot] function {
            set isosamples 55, 55;
            set contour base;
            set cntrparam levels incremental -1.6,\step,1.6;
            %set style data lines;
            unset  surface;
            splot [-4:4] [-2.2:2.2] (y*(1+\shape/(x**2 + y**2))) ;
            };
% ================================ ���� ======================================
\fill[gray!20, draw=blue, thick] (0,0) circle (1.01);
% ======================= ������ �� ����� ==================================
\foreach \i in {-1.6,-1.2,...,1.8} {
\draw[red, -latex', rotate around = {{-asin(\i/(3 +\shape/3))}:({asin(\i/(3 +\shape/3))}:3)}] ({asin(\i/(3 +\shape/3))}:3) -- ({asin(\i/(3 +\shape/3))}:3.1);
\draw[red, -latex',rotate around = {{180-asin(\i/(3 +\shape/3))}:({180+asin(\i/(3 +\shape/3))}:3)} ] ({180+asin(\i/(3 +\shape/3))}:3) -- ({180+asin(\i/(3 +\shape/3))}:3.1);
}
% ======================= ���� � �������� ��� ==============================
\foreach \i in {-1.6,-1.4,...,1.8} {
\pgfmathparse{abs(\i/(1+\shape))}
\ifdim\pgfmathresult cm < 1 cm
\draw[red, decoration={markings, mark=at position 0.5 with {\arrow{latex'}}}, postaction={decorate}] 
({180-asin(\i/(1+\shape))}:1) -- ({asin(\i/(1+\shape))}:1);
\fi
}
% ============================ ����� ������ =================================
\foreach \i in {70,30,15,0}{
\node at (\i:0.9) {\tiny $-$};
\node at (-\i:0.9) {\tiny $-$};
\node at ({180-\i}:0.9) {\tiny $+$};
\node at ({180+\i}:0.9) {\tiny $+$};
%==================================
\node at (\i:1.1) {\tiny $+$};
\node at (-\i:1.1) {\tiny $+$};
\node at ({180-\i}:1.1) {\tiny $-$};
\node at ({180+\i}:1.1) {\tiny $-$};
}
\end{tikzpicture}

			Поле вектора $\Dfield$ ($\epsilon_i < \epsilon_e$)

			\input{Dfield1.tikz}

		\end{center}

		                                                                                                                   &
		Дипольний момент кулі:

		\[\vect{p} = \frac{\epsilon_i - \epsilon_e}{\epsilon_i + 2\epsilon_e}R^3\Efield_0,\]

		{\scriptsize Напрямок дипольного моменту визначається різницею $\epsilon_e - \epsilon_e$.}

		Потенціал кулі:

		\[
			\phi =
			\begin{cases}
				-\frac{3\epsilon_e}{\epsilon_i + 2\epsilon_e}\left( \Efield_0\cdot \vect{r}\right), & r \le R \\
				-\left( \Efield_0\cdot \vect{r}\right) + \frac{\vect{p} \vect{r}}{r^3},             & r > R
			\end{cases}
			,\]

		Поле кулі:

		\[
			\Efield =
			\begin{cases}
				\frac{3\epsilon_e}{\epsilon_i + 2\epsilon_e} \Efield_0,                                 & r \le R \\
				\Efield_0 - \frac{\vect{p}}{r^3} + \frac{3\left(\vect{p}\vect{r}\right)\vect{r} }{r^5}, & r > R,
			\end{cases}
		\]

		Зв'язані заряди на поверхні:

		\[\sigma' = \frac{3}{4\pi} \frac{\epsilon_i - \epsilon_e}{\epsilon_i + 2\epsilon_e} \frac{\Efield_0\vect{r}}{R}.\]


		\\
		%============================================================================================  
		\hline\pagebreak
		\multicolumn{2}{|c|}{\cellcolor{cyan!10}Металева сфера в однорідному електричному полі}                                                                                                                                   \\
		\hline

		\begin{center}
			Поле вектора $\Efield$

			\begin{tikzpicture}
% ============================ ��������� ===================================
\pgfmathsetmacro{\step}{0.4}
\pgfmathsetmacro{\ea}{5}
\pgfmathsetmacro{\eb}{1}
\pgfmathsetmacro{\shape}{1}
% ============================== ������� ===================================
\draw [
    raw gnuplot, red,
    ] plot[id=curve, raw gnuplot] function {
            set isosamples 55, 55;
            set contour base;
            set cntrparam levels incremental -2.2,\step,2.2;
            %set style data lines;
            unset  surface;
            splot [-4:4] [-2.2:2.2] (y*(1+\shape/(x**2 + y**2))) ;
            };
% ================================ ���� ======================================
\fill[gray!20, draw=blue, thick] (0,0) circle (1.01);
% ======================= ������ �� ����� ==================================
\foreach \i in {-2.2,-1.8,...,2.2} {
\draw[red, -latex', rotate around = {{-asin(\i/(3 +\shape/3))}:({asin(\i/(3 +\shape/3))}:3)}] ({asin(\i/(3 +\shape/3))}:3) -- ({asin(\i/(3 +\shape/3))}:3.1);
\draw[red, -latex',rotate around = {{180-asin(\i/(3 +\shape/3))}:({180+asin(\i/(3 +\shape/3))}:3)} ] ({180+asin(\i/(3 +\shape/3))}:3) -- ({180+asin(\i/(3 +\shape/3))}:3.1);
}
% ============================ ����� ������ =================================
\foreach \i in {70,30,15,0}{
%==================================
\node at (\i:1.1) {\tiny $+$};
\node at (-\i:1.1) {\tiny $+$};
\node at ({180-\i}:1.1) {\tiny $-$};
\node at ({180+\i}:1.1) {\tiny $-$};
}
\end{tikzpicture}

		\end{center}

		                                                                                                                   &

		Дипольний момент кулі:

		\[\vect{p} = R^3\Efield_0,\]

		Потенціал кулі:

		\[
			\phi =
			\begin{cases}
				0,                                                                      & r \le R \\
				-\left( \Efield_0\cdot \vect{r}\right) + \frac{\vect{p} \vect{r}}{r^3}, & r > R
			\end{cases}
			,\]

		Поле кулі:

		\[
			\Efield =
			\begin{cases}
				0,                                                                                      & r \le R \\
				\Efield_0 - \frac{\vect{p}}{r^3} + \frac{3\left(\vect{p}\vect{r}\right)\vect{r} }{r^5}, & r > R,
			\end{cases}
		\]

		Вільні заряди на поверхні:

		\[\sigma = \frac{3}{4\pi} \frac{\Efield_0\vect{r}}{R}.\]

		\\
		\hline
		\multicolumn{2}{|c|}{\cellcolor{cyan!10}Провідна куля в середовищі з провідністю по якому тече струм}                                                                                                                     \\
		\hline

		\begin{center}

			Поле вектора $\vect{j}$ ($\lambda_i > \lambda_e$)

			\input{Dfield0.tikz}

			Поле вектора $\vect{j}$ ($\lambda_i < \lambda_e$)

			\input{Dfield1.tikz}

		\end{center}


		                                                                                                                   &

		Дипольний момент кулі:

		\[\vect{p} = \frac{\lambda_i - \lambda_e}{\lambda_i + 2\lambda_e}R^3\Efield_0,\]

		Поле:
		\[
			\Efield =
			\begin{cases}
				\frac{3\lambda_e}{\lambda_i + 2\lambda_e} \Efield_0,                                    & r \le R \\
				\Efield_0 - \frac{\vect{p}}{r^3} + \frac{3\left(\vect{p}\vect{r}\right)\vect{r} }{r^5}, & r > R,
			\end{cases}
		\]

		Густина струму:

		\[
			\vect{j} =
			\begin{cases}
				\frac{3\lambda_e}{\lambda_i + 2\lambda_e} \vect{j}_0,                                                       & r \le R \\
				\vect{j}_0  + \lambda_e\frac{3\left(\vect{p}\vect{r}\right)\vect{r} }{r^5} - \lambda_e\frac{\vect{p}}{r^3}, & r > R,
			\end{cases}
		\]

		Заряди на поверхні:

		\[\sigma = \frac{3}{4\pi} \frac{\lambda_i - \lambda_e}{\lambda_i + 2\lambda_e} \frac{\Efield_0\vect{r}}{R}.\]
		\\
		\hline\pagebreak
		\multicolumn{2}{|c|}{\cellcolor{cyan!10}Куля з магнетика в однорідному магнітному полі}                                                                                                                                   \\
		\hline


		\begin{center}
			Поле вектора $\Hfield$ ($\mu_e > \mu_e$)

			\input{Hfield.tikz}

			Поле вектора $\Bfield$ ($\mu_e > \mu_e$)

			\begin{tikzpicture}
	% ============================ ��������� ===================================
	\pgfmathsetmacro{\step}{0.4}
	\pgfmathsetmacro{\ea}{5}
	\pgfmathsetmacro{\eb}{1}
	\pgfmathsetmacro{\shape}{(\ea-\eb)/(\ea+2*\eb)}
	% ============================== ������� ===================================
	\draw [
	    raw gnuplot, red,
	    ] plot[id=curve, raw gnuplot] function {
	            set isosamples 55, 55;
	            set contour base;
	            set cntrparam levels incremental -2,\step,2;
	            %set style data lines;
	            unset  surface;
	            splot [-4:4] [-2.2:2.2] (y*(1+\shape/(x**2 + y**2))) ;
	            };
	% ================================ ���� ======================================
	\fill[gray!20, draw=blue, thick] (0,0) circle (1.01);
	% ======================= ������ �� ����� ==================================
	\foreach \i in {-2,-1.6,...,2} {
	\draw[red, -latex', rotate around = {{-asin(\i/(3 +\shape/3))}:({asin(\i/(3 +\shape/3))}:3)}] ({asin(\i/(3 +\shape/3))}:3) -- ({asin(\i/(3 +\shape/3))}:3.1);
	\draw[red, -latex',rotate around = {{180-asin(\i/(3 +\shape/3))}:({180+asin(\i/(3 +\shape/3))}:3)} ] ({180+asin(\i/(3 +\shape/3))}:3) -- ({180+asin(\i/(3 +\shape/3))}:3.1);
	}
	% ======================= ���� � �������� ��� ==============================
	\foreach \i in {-1.6,-1.2,...,1.8} {
	\pgfmathparse{abs(\i/(1+\shape))}
	\ifdim\pgfmathresult cm < 1 cm
	\draw[red, decoration={markings, mark=at position 0.5 with {\arrow{latex'}}}, postaction={decorate}] 
	({180-asin(\i/(1+\shape))}:1) -- ({asin(\i/(1+\shape))}:1);
	\fi
	}
	\end{tikzpicture}
		\end{center}

		                                                                                                                   &
		Дипольний момент кулі:

		\[\vect{p}_m = \frac{\mu_i - \mu_e}{\mu_i + 2\mu_e}R^3\Bfield_0,\]

		Поле кулі:

		\[
			\Bfield =
			\begin{cases}
				\frac{3\mu_i}{\mu_i + 2\mu_e} \Bfield_0,                                                    & r \le R \\
				\Bfield_0 - \frac{\vect{p}_m}{r^3} + \frac{3\left(\vect{p}_m\vect{r}\right)\vect{r} }{r^5}, & r > R,
			\end{cases}
		\]

		Густина об'ємних струмів намагнічування $\vect{j}' = 0$.

		Поверхнева густина струмів намагнічування
		\[i = \frac{3c}{4\pi} \frac{\mu_i - \mu_e}{\mu_i + 2\mu_e} \frac{\Bfield_0\vect{r}}{R},\]
		де $\vect{r}$~-- радіус-вектор поверхні провідника.

		\\
		\hline
		\multicolumn{2}{|c|}{\cellcolor{cyan!10}Надпровідна куля в однорідному магнітному полі $\Bfield_0$}                                                                                                                       \\
		\hline

		\begin{center}
			Поле вектора $\Bfield$
			\begin{tikzpicture}
\clip (-4,-2) rectangle (4,2);
% ======================= стрілки на лініях ==================================
\foreach \i [evaluate=\i as \x using \i*0.1]in {-12,-10,...,12} {
\ifnum\i=0\relax\else%
\draw[red, -latex', rotate around = {{-asin(\x/(3 -1/3))}:({asin(\x/(3 -1/3))}:3)}] ({asin(\x/(3 -1/3))}:3) -- ({asin(\x/(3 -1/3))}:3.1);

\draw[red, -latex',rotate around = {{180-asin(\x/(3 -1/3))}:({180+asin(\x/(3 -1/3))}:3)} ] ({180+asin(\x/(3 -1/3))}:3) -- ({180+asin(\x/(3 -1/3))}:3.1);
\fi
}
% ================================ куля ======================================
\fill[gray!20, draw=blue, thick] (0,0) circle (1.01);
\foreach \i  in {0.1,0.2,...,0.6} {
\foreach \j in {-1,1} {
\draw [color=red,
		domain=179:1,
		samples=200,
		smooth,
		] plot (xy polar cs:angle=\x,radius= {\j*\i/sin(\x) +\j * sqrt( (\i)^2/(sin(\x)^2) + 1) });
		}
		}
\end{tikzpicture}
		\end{center}
		                                                                                                                   &

		Дипольний момент кулі:

		\[\vect{p}_m = - \frac{1}{2}R^3\Bfield_0,\]

		Поле кулі:

		\[
			\Bfield =
			\begin{cases}
				0,                                                                                          & r \le R \\
				\Bfield_0 - \frac{\vect{p}_m}{r^3} + \frac{3\left(\vect{p}_m\vect{r}\right)\vect{r} }{r^5}, & r > R,
			\end{cases}
		\]

		Густина об'ємних струмів намагнічування $\vect{j}' = 0$.

		Поверхнева густина струмів намагнічування
		\[i = -\frac{3c}{8\pi} \frac{\Bfield_0\vect{r}}{R},\]
		де $\vect{r}$~-- радіус-вектор поверхні провідника.

		\\
		\hline
	\end{longtable}
\end{center}




\end{document}


