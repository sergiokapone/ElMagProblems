% !TeX program = lualatex
% !TeX encoding = utf8
% !TeX spellcheck = uk_UA
% !TeX root =../EMProblems.tex

%=========================================================
\Opensolutionfile{answer}[\currfilebase/\currfilebase-Answers]
\Writetofile{answer}{\protect\section*{\nameref*{\currfilebase}}}
 \chapter{Магнітостатика}\label{\currfilebase}
%=========================================================

\epigraph{\Annabelle  Так как я уже давно рассматривал силы, проявляющиеся в электрических явлениях, всеобщими природными силами, то я должен был отсюда вывести и магнитные действия}{Х.-К.~Эрстед}

\section{Закон Біо-Савара-Лапласа. Теорема про циркуляцію}
\begin{Theory}
Для випадку магнітостатики у вакуумі, рівняння Максвелла приймають вигляд:
Інтегральна форма:
\begin{align}
	&\oiint\limits_S \Bfield\cdot d\vect{S} = 0 && \text{\small Теорема Гауса для магнітного поля} \label{Int II_magstat}\\
	&\oint\limits_L \Bfield\cdot d\vect{r} =\dfrac{4\pi}{c} \iint\limits_S \vect{j}\cdot d\vect{S}  &&\text{\small Теорема про циркуляцію вектора індукції} \label{Int IV_magstat}
\end{align}
або у диференціальній формі:
\begin{flalign}
	\divg\Bfield &= 0 \label{Diff II_magstat}\\
	\rot\Bfield &= \dfrac{4\pi}{c} \vect{j} \label{Diff IV_magstat}
\end{flalign}

Розв'язком цих рівнянь є закон Біо-Савара-Лапласа:
\begin{equation}
	\Bfield = \frac{1}{c}\iiint\limits_{V} \frac{\vect{j}\times \vect{r}  dV}{\left| \vect{r}\right|^3 },
\end{equation}
де $\vect{r}$~--- радіус-вектором проведений від елемента струму $\vect{j}dV$ до точки спостереження.

%Індукція магнітного поля точкової зарядженої частинки, що несе заряд $q$ і рухається  з швидкістю $\vect{v}$ визначається формулою:
%\begin{equation}
%	\Bfield = \frac{q}{r^3} \left[ \frac{\vect{v}}{c}\times \vect{r}\right].
%\end{equation}
\end{Theory}

\subsection*{Визначення характеристик магнітного поля розподілів струмів, які мають певну симетрію}

%=========================================================
\begin{problem}\label{prb:inf_wire}
    Визначити магнітне поле на відстані $r$ від нескінченно довгого провідника зі струмом (рис.~\ref{inf_wire}) за допомогою закону Біо-Савара-Лапласа та за допомогою теореми про циркуляцію. Порівняйте результати.
\end{problem}

%=========================================================
\begin{problem}\label{prb:short_wire}
    Визначте магнітне поле в точці $P$ на відстані $r$ від короткого провідника зі струмом. Положення точки $P$ визначається кутами $\alpha_1$ та $\alpha_2$ (рис.~\ref{short_wire}). Поясніть, чому в  цьому випадку не можна скористатись теоремою про циркуляцію.
\end{problem}
%=========================================================
\begin{figure}[h!]\centering
%---------------------------------------------------------
\begin{minipage}[t]{0.45\linewidth}\centering
	\begin{tikzpicture}
        \draw (0,0)  to [current direction={info={$I$}}] +(0,5) coordinate (B);
		\draw[dashed] (B) -- +(0,1) (0,0) -- +(0,-1);
		\coordinate (P) at (1.5,2.5);
		\fill  (P) circle (0.05) node[right] {$P$};
		\draw let \p1=(P) in (0,\y1) -- node[below] {$r$} (\x1,\y1);
		\draw[themecolordark] let \p1 = (P) in (0,\y1) ellipse(\x1 and 0.3cm);
		\draw[-latex, themecolordark] let \p1 = (P) in (0,\y1) +(45:\x1 and 0.3cm) arc(45:135:\x1 and 0.3cm) node[black, above] {$\Bfield$};
\end{tikzpicture}
\caption{До задачі~\ref{prb:inf_wire}}
\label{inf_wire}
\end{minipage}
%---------------------------------------------------------
\begin{minipage}[t]{0.45\linewidth}\centering
    \begin{tikzpicture}
        \draw (0,0) node[contact] {}  to [current direction={info={$I$}}] +(0,5) coordinate (B) node[contact] {};
		\draw[dashed] (B) -- +(0,1) (0,0) -- +(0,-1);
		\coordinate (P) at (1.5,3);
		\fill  (P) circle (0.05) node[right] {$P$};
		\draw let \p1=(P) in (0,\y1) -- node[below] {$r$} (\x1,\y1);
		\draw (P) -- (0,0) (P) -- (B);
		\draw let \p1=(P) in (0,0.5) arc(90:{90-atan(\x1/\y1)}:0.5) node[pos=1, right] {$\alpha_1$};
		\draw let \p1=(P), \p2=(B) in ([yshift=0.5cm]B) arc(90:{atan((\y1 - \y2)/(\x1 - \x2))}:0.5) node[pos=0.5, right] {$\alpha_2$};
		\draw[themecolordark] let \p1 = (P) in (0,\y1) ellipse(\x1 and 0.3cm);
		\draw[-latex, themecolordark] let \p1 = (P) in (0,\y1) +(45:\x1 and 0.3cm) arc(45:135:\x1 and 0.3cm) node[black, above] {$\Bfield$};
    \end{tikzpicture}
\caption{До задачі~\ref{prb:short_wire}}
\label{short_wire}
\end{minipage}
%---------------------------------------------------------
\end{figure}
%=========================================================

%=========================================================
\begin{problem}\label{prb:bsl_contour}
    Знайдіть індукцію магнітного поля в точці $O$ контуру зі струмом $I$, який показано
	\begin{enumerate*}[label = \alph*)]
		\item на \ref{bsl_contour1}, при чому радіуси $a$ та $b$ і кут $\phi$ відомі;
		\item на \ref{bsl_contour2}, при чому радіуси $a$ та сторона $b$ відомі.
	\end{enumerate*}
\begin{solution}
	\begin{enumerate*}[label = \alph*)]
		\item $\frac{I}{c} \left( \frac{2\pi - \phi}{a}  + \frac{\phi}{b}\right) $;
		\item $\frac{I}{c} \left( \frac{3\pi}{a}  + \frac{\sqrt2}{b}\right) $.
	\end{enumerate*}
\end{solution}
\end{problem}
%=========================================================
\begin{figure}[h!]\centering
%---------------------------------------------------------
\begin{minipage}[t]{0.45\linewidth}\centering
	\begin{tikzpicture}
		\def\startangle{80}
		\def\stoptangle{10}
		\fill (0,0) circle (0.05) node[below left] {$O$};
		\draw[-latex] (0,0) -- (-45:1) node[above, pos=0.7] {$a$};
		\draw[-latex] (0,0) ++(\stoptangle:1) coordinate (B) arc(\stoptangle:-135:1);
		\draw (0,0) +(-134:1) arc(-134:-270-\stoptangle:1) coordinate (A);
		%\draw (0,1) -- +(75:1) arc (90:0:2) -- ++(-1,0);
		\draw[-latex] (0,0) -- (45:2) node[above, pos=0.6] {$b$};
		\draw (0,0) ++(\startangle:2) coordinate (A1) arc(\startangle:\stoptangle:2) coordinate (B1);
		\draw (A1) -- (A) (B1) -- (B);
		\draw[dashed] (0,0) -- (A) (0,0) -- (B);
		\draw (0,0) +(\stoptangle:0.3) arc (\stoptangle:\startangle:0.3) node[above, pos=0.5] {$\phi$};
	\end{tikzpicture}
\caption{До задачі~\ref{prb:bsl_contour}}
\label{bsl_contour1}
\end{minipage}
%---------------------------------------------------------
\begin{minipage}[t]{0.45\linewidth}\centering
	\begin{tikzpicture}
		\fill (0,0) circle (0.05) node[below left] {$O$};
		\draw[-latex] (0,0) -- (-45:1) node[above, pos=0.7] {$a$};
		\draw[-latex] (0,0) +(1,0) arc(0:-135:1);
		\draw (0,0) +(-134:1) arc(-134:-270:1);
		\draw (0,1) -- ++(0,1) -- node[above] {$b$} ++(2,0) -- node[right] {$b$} ++(0,-2)  -- ++ (-1,0);
		\draw[dashed] (0,0) -- +(0,1) (0,0) -- +(1,0);
	\end{tikzpicture}
\caption{До задачі~\ref{prb:bsl_contour}}
\label{bsl_contour2}
\end{minipage}
%---------------------------------------------------------
\end{figure}
%=========================================================

%=========================================================
\begin{problem}
    Вздовж нескінченної металевої пластини тече струм рівномірно розподілений по її ширині з густиною $i$. Знайти магнітне поле в просторі навколо такої пластини.
\begin{solution}
	$\Bfield = \frac{2\pi i }{c} \left[ \vect{\tau}\times\vect{n}\right] $, де $\vect{\tau}$~-- тангенціальний вектор, напрямлений в бік протікання струму, $\vect{n}$~-- вектор нормалі до поверхні.
\end{solution}
\end{problem}

%=========================================================
\begin{problem}
    Знайдіть магнітне поле в середині нескінченно довгого соленоїда, по поверхні якого по колу тече струм, рівномірно розподілений за шириною з густиною $i$.
	\begin{solution}
			$\Bfield = - \frac{4\pi i }{c} \left[ \vect{\tau}\times\vect{n}\right] $, де $\vect{\tau}$~-- тангенціальний вектор, напрямлений в бік протікання струму, $\vect{n}$~-- вектор зовнішньої нормалі до поверхні соленоїда.
	\end{solution}
\end{problem}


%=========================================================
%\begin{problem} %Батигін 2.53 4е вид
%Прямолінійна нескінченно довга смуга має ширину $ a $. Вздовж смуги тече струм $I$ рівномірно розподілений по її ширині. Знайти магнітне поле.
%\end{problem}

%=========================================================
\begin{problem}
    Знайти магнітне поле, що створюється тонким півкільцем радіусом $R$  в його центрі.
\begin{solution}
	$B = \frac{4I}{c\pi R}$.
\end{solution}
\end{problem}



%=========================================================
\begin{problem}\label{Bfield_of_electric_dipole}
    Визначити індукцію магнітного поля в дипольному наближенні електричного диполя з моментом $\vect{p}_e$ що рухається зі швидкістю $\vect{v}$.
\begin{solution}
	$\Bfield = \left[ \frac{\vect{v}}{c}\times\Efield\right]  $, де $\Efield$~--- напруженість електричного поля диполя.
\end{solution}
\end{problem}


%%=========================================================
%\begin{problem}
%Виходячи з закону Біо-Савара-Лапласа показати, що для довільного замкненого контуру, по якому тече струм $I$ індукція магнітного поля у вакуумі в деякій точці виражається формулою $\Bfield = - \frac{I}{c} \vect{\nabla}\Omega$, де $\Omega$~-- тілесний кут, під яким контур видно з цієї точки.
%\end{problem}

%=========================================================
\begin{problem}\label{prb:fielinellipsefocus}
По провіднику, який має форму еліпса, тече струм $I$. Велика і мала півоісі еліпса дорівнюють $a$ та $b$ відповідно. Визначити індукцію магнітного поля в фокусі еліпса.
\begin{solution}
	$B = \frac{2\pi I}{cp}$, де $p = \frac{b^2}{a}$~-- фокальний параметр еліпса.
\end{solution}
\end{problem}

%=========================================================
\begin{problem}
Знайти поле на осі і в центрі колового витка радіусом $R$, по якому тече струмом $I$. Використовуючи отриманий результат, знайти:
\begin{enumerate*}[label=\alph*)]
	\item поле на осі круглого соленоїда в точці, з якої його краї видно під кутами $\alpha_1$ та $\alpha_2$,
	\item поле на кінці напівнескінченного соленоїда,
	\item поле в середині нескінченного соленоїда. Число витків на одиницю довжини соленоїда дорівнює $n$.
\end{enumerate*}
\begin{solution}
	$B_z = \frac{2\pi}{c}\frac{I R^2}{(R^2 + z^2)^{3/2}}$,
	\begin{enumerate*}[label=\alph*)]
		\item $B_z = \frac{2\pi}{c} In (\cos\alpha_1 - \cos\alpha_2)$,
		\item $B_z(0) = \frac{2\pi}{c} In$,
		\item $B_z(0) = \frac{4\pi}{c} In$.
	\end{enumerate*}
\end{solution}
\end{problem}

%=========================================================
\begin{problem}\label{prb:MF_rotated_disk} %http://www.physicspages.com/2013/04/17/magnetic-fields-of-spinning-disk-and-sphere/
Знайти магнітне поле на осі рівномірно зарядженого диска радіусом $R$ (повний заряд диска дорівнює $q$, що обертається навколо осі з кутовою швидкістю $\omega$ на відстані $z$ від диска.
\begin{solution}
	$B_z(z) = \frac{2q\omega}{cR^2}\left(\frac{2z^2 + R^2}{\sqrt{R^2 + z^2}} - 2z\right)$.
\end{solution}
\end{problem}

%=========================================================
\begin{problem}
Нескінченно довгий циліндричний дріт складається з двох коаксіальних циліндрів. Внутрішній суцільний циліндр, виготовлений з немагнітного матеріалу  і має радіус $R_1$. Радіус зовнішнього пустотілого циліндра дорівнює $R_2$. Уздовж циліндрів течуть постійні струми однакової величини $I$, але спрямовані протилежно. Визначити залежність величини індукції магнітного поля від відстані до осі дроту.
\begin{solution}
	$B(r) = %
		\begin{cases}
			\frac{2Ir}{cR_1^2}, & \quad r < R_1           \\
			\frac{2I}{cr},      & \quad R_1 \le r \le R_2 \\
			0,                  & \quad r \ge R_2
		\end{cases}
	$.
\end{solution}
\end{problem}

%=========================================================
\begin{problem}
Нескінченно довгий циліндричний циліндричний дріт складається з двох коаксіальних циліндрів. Внутрішній суцільний циліндр, виготовлений з немагнітного матеріалу  і має радіус $R_1$ та питомий опір $\rho_1$. Радіус зовнішнього циліндра дорівнює $R_2$ та питомий опір $\rho_2$. Зовнішня поверхня внутрішнього циліндра знаходиться в тісному контакті із внутрішньою поверхнею зовнішнього. Уздовж циліндрів тече постійний струм $I$. Визначити залежність величини індукції магнітного поля від відстані до осі дроту.
\begin{solution}
	$B(r) = %
		\begin{cases}
			\frac{2I\rho_1r}{c(\rho_1(R_2^2 - R_1^2) + \rho_2R_1^2)}, & \quad r < R_1           \\
			\frac{2I(\rho_1(r^2 - R_1^2) + \rho_2R_1^2)}{cr(\rho_1(R_2^2 - R_1^2) + \rho_2R_1^2)},      & \quad R_1 \le r \le R_2 \\
			\frac{2I}{cr},                  & \quad r \ge R_2.
		\end{cases}
	$
\end{solution}
\end{problem}

%=========================================================
\begin{problem}\label{prb:hole_in_wire}
В середині довгого прямого дроту круглого перерізу є довга кругла циліндрична порожнина, вісь якої паралельна осі проводу і зміщена щодо неї на відстань $d$ (рис.~\ref{hole_in_wire}). По дроту тече постійний струм густиною $j$, рівномірно розподілений за перерізом. Нехтуючи впливом речовини дроту, визначити величину індукції магнітного поля всередині порожнини.
\begin{solution}
	$B = \frac{2\pi}{c} jd$.
\end{solution}
\end{problem}

%=========================================================
\begin{problem}\label{prb:cut_in_cylinder_wire}
Уздовж довгої тонкостінної циліндричної трубки радіуса $R$ тече постійний струм $I$. У стінці трубки є тонка щілина ширини $d$, паралельна осі трубки (рис.~\ref{cut_in_cylinder_wire}). Визначити величину індукції магнітного поля в точці, що лежить всередині трубки на її радіусі, якщо відстань від середини щілини до даної точки дорівнює $r$ ($R \gg  d$).
\begin{solution}
	$B(r) = \frac{1}{c}\frac{Id}{rR}$.
\end{solution}
\end{problem}

%=========================================================
\begin{figure}[htbp!]\centering
	%---------------------------------------------------------
	\begin{minipage}{0.45\linewidth}\centering
		\begin{tikzpicture}
			\draw [fill=gray!30] (0,0) coordinate (A) circle (1.75);
			\draw [fill=white] (-1,0) coordinate (B) circle (0.5);
			\draw [dash dot] (A) -- ([yshift=-2.7cm]A) coordinate (C)
			(B) -- ([yshift=-2.7cm]B) coordinate (D);
			\draw[latex'-latex'] ([yshift=0.2cm]C) -- node[below] {$d$}([yshift=0.2cm]D);
		\end{tikzpicture}
		\caption{До задачі~\ref{prb:hole_in_wire}}
		\label{hole_in_wire}
	\end{minipage}
	%---------------------------------------------------------
	\begin{minipage}{0.45\linewidth}\centering
		\begin{tikzpicture}%[tdplot_main_coords]
			%----------- Data ------------------
			\pgfmathsetmacro{\R}{1.5}
			\pgfmathsetmacro{\PHI}{5}
			\pgfmathsetmacro{\distance}{0.25}
			%--------- Calculations ------------
			\pgfmathsetmacro\PHII{%
				atan(\R*sin(\PHI)/(\distance + \R*cos(\PHI)))}
			\pgfmathsetmacro\RR{%
				\R*sin(\PHI)/sin(\PHII)}
			%------------ Drawing --------------
			\draw[gray, fill=gray!50] (\PHI:\R) arc (\PHI:360-\PHI:\R) --
			(360-\PHII:\RR) arc (360-\PHII:\PHII:\RR) -- cycle;

			\draw[-latex'] (0,0) -- node[above=10pt, pos=0.5] {$R$} (45:\R);

			\draw[-latex'] (30:\RR+0.2) arc (30:\PHII:\RR+0.2);
			\draw[-latex'] (-30:\RR+0.2) arc (-30:-\PHII:\RR+0.2);
			\node at (0:2.5) {$d$};

			\node[below] at (0,-2.7) {}; % remove if alone
		\end{tikzpicture}
		\caption{До задачі~\ref{prb:cut_in_cylinder_wire}}
		\label{cut_in_cylinder_wire}
	\end{minipage}
	%---------------------------------------------------------
\end{figure}

%=========================================================
\begin{problem}% Lim 2021
Нескінченно довгий діелектричний циліндр радіусом $R$ радіально поляризований таким чином, що поляризованість визначається законом $\vect{P}~=~a \vect{r}$, де $a$~-- позитивна константа. Циліндр обертається з постійною кутовою швидкістю $\omega$ навколо своєї осі. Визначити магнітне поле в залежності від відстані до осі в середині та зовні циліндра.
\begin{solution}
	$\Bfield = %
		\begin{cases}
			\frac{4\pi}{c} a \omega r^2 \vect{e}_r, & \quad r < R \\
			0,                                      & \quad r > R
		\end{cases}
	$.
\end{solution}
\end{problem}

\section{Векторний потенціал. Магнітний момент}

\begin{Theory}\small
  Зв'язок вектора індукції магнітного поля з вектор-потенціалом:
  \begin{equation}\label{BrotA}
	  \Bfield = \rot\vect{A}.
  \end{equation}

  Рівняння Пуассона для вектор-потенціалу:
  \begin{equation}
	  \nabla^2\vect{A} = - \frac{4\pi}{c}\vect{j}.
  \end{equation}

  Розв'язком цього рівняння є вектор-потенціал:
  \begin{equation}\label{Supperposition_principle}
	  \vect{A}  = \iiint\limits_{V'} \frac{\vect{j}' dV'}{\left| \vect{r} - \vect{r'} \right|}.
  \end{equation}
  де $\vect{r}'$~--- радіус-вектор елементів заряду $\rho(\vect{r'}) dV'$, а $\vect{r}$~--- радіус-вектор точки спостереження.

	Циркуляція вектор-потенціалу вздовж довільного контуру $L$ дорівнює потоку вектора індукції магнітного поля крізь поверхню $S$, на яку натягнуто даний контур:

  \begin{equation}\label{Bflux_is_circulating_potential}
	  \iint\limits_S \Bfield d\vect{S} = \oint\limits_L \vect{A} d\vect{r}.
  \end{equation}

  Означення вектора магнітного моменту:
  \begin{equation}\label{mag_momentum}
	  \vect{p}_m = \frac{1}{2c} \iiint\limits_V \vect{r}\times\vect{j}\,dV,
  \end{equation}

  Вектор-потенціал поля магнітного диполя:
  \begin{equation}\label{dipole_vect_potential}
	  \vect{A} = \frac{\vect{p}_m\times \vect{r} }{r^3}.
  \end{equation}

  Магнітне поле диполя:
  \begin{equation}\label{dipole_Bfield}
	  \Bfield = \frac{3(\vect{p}_m\cdot\vect{r})\vect{r}}{r^5} - \frac{\vect{p}_m}{r^3}.
  \end{equation}
\end{Theory}

\subsection*{Поняття ротора}

%=========================================================
\begin{problem}
    Знайдіть $\rot\vect{r}$, $\vect{r}$~-- радіус-вектор довільної точки.
\end{problem}

\begin{problem}
    Знайдіть $\rot\frac{\vect{r}}{r^3}$, $\vect{r}$~-- радіус-вектор довільної точки.
\end{problem}

\begin{problem}
    Знайдіть $\rot\left[ \vect{a}\times\frac{\vect{r}}{r^3}\right] $, де $\vect{a}$~-- довільний постійний вектор, $\vect{r}$~-- радіус-вектор довільної точки.
\end{problem}


\subsection*{Розрахунок магнітного моменту струмів}

%=========================================================
\begin{problem}
Показати, що з означення вектора магнітного моменту~\ref{mag_momentum} для випадку колового витка зі струмом $I$ площею $S$ випливає формула:
\[
	\vect{p}_m = \frac1c IS\vect{n},
\]
де $\vect{n}$~--- вектор нормалі до поверхні $S$.
\end{problem}

%=========================================================
\begin{problem}
    Згідно теорії Бора, електрон в основному стані атома водню обертається навколо ядра по коловій орбіті на відстані, що дорівнює борівському радіусу $a$. Знайти магнітний момент електрона, що пов'язаний з таким орбітальним рухом.
\end{problem}


%=========================================================
\begin{problem}
    Знайдіть магнітний момент короткого соленоїда довжиною $b$ і радіусом $R$ по поверхні якого тече поверхневий коловий струм густиною~$i$.
	\begin{solution}
		$p = \frac1c ib \pi R^2$.
	\end{solution}
 \end{problem}

%=========================================================
\begin{problem}
    Знайти магнітний момент котушки, яка представляє собою витки, щільно намотані намотані на половину тора. Сила струму яка тече по виткам $I$, діаметр перерізу тора $D$, кількість витків $N = 500$.
\begin{solution}
	$p = \frac{IND^2}{2c}$.
\end{solution}
\end{problem}

%=========================================================
\begin{problem}
Для умов задачі~\ref{prb:MF_rotated_disk} знайти магнітне поле за умови $z \gg R$ та виразити його через магнітний момент диска.
\begin{solution}
	$B = \frac{q\omega}{2c}\frac{R^2}{z^3}$, $B = \frac{2p_m}{z^3}$.
\end{solution}
\end{problem}

%=========================================================
\begin{problem}\label{mag_momentum_of_sphere}
Знайти магнітний момент однорідно зарядженої сфери (кулі) радіусом $R$, яка має повний заряд $q$ і обертається навколо одного зі своїх діаметрів з кутовою швидкістю $\vect{\omega}$. Визначити гіромагнітне відношення для обох випадків, якщо маса сфери (кулі) дорівнює $m$.
\begin{solution}
	Для сфери $\vect{p}_m = \frac{qR^2}{3c}\vect{\omega}$,
	для кулі $\vect{p}_m = \frac{qR^2}{5c}\vect{\omega}$.
	В обох випадках $\frac{p_m}{L} = \frac{q}{2mc}$.
\end{solution}
\end{problem}

%=========================================================
\begin{problem}
    Гіромагнітним співвідношенням (магнітомеханічним співвідношенням) називається відношення дипольного магнітного моменту тіла до його моменту імпульсу. Знайдіть гіромагнітне співвідношення для сфери (кулі). Скористайтеся відповідями до задачі~\ref{mag_momentum_of_sphere}.
\end{problem}

%=========================================================
\begin{problem}
    Користуючись поняттям гіромагнітного співвідношення та моменту інерції, запропонуйте спосіб знаходження магнітних моментів довільних тіл, повний заряд яких дорівнює $q$, що обертаються з кутовою швидкістю $\omega$.
\end{problem}


%%=========================================================
%\begin{problem}
%	Куля (сфера) радіусом $R$ заряджена зарядом $q$ рівномірно по об'єму (поверхні) і обертається навколо одного зі своїх діаметрів з кутовою швидкістю~$\omega$. Знайти магнітне поле всередині і зовні кулі (сфери). Виразити її індукцію через магнітний момент.
%\begin{solution}
%	$\Bfield =
%	\begin{cases}
%		 \frac{3(\vect{p}_m\vect{r})\vect{r}}{r^5} -\frac{\vect{p}_m}{r^3}, & r \ge R \\
%		 \frac{5\vect{p}_m}{R^3} + \frac{3(\vect{p}_m'\vect{r})\vect{r}}{r^5}  - \frac{6\vect{p}_m'}{r^3}, & r \le R
%	\end{cases},$
%	 де $\vect{p}_m'$~-- магнітний момент кулі радіусом $r$.
%\end{solution}
%\end{problem}

\subsection*{Задачі на знаходження та використання поняття векторного потенціалу}

%=========================================================
\begin{problem}
    Доведіть формулу~\ref{Bflux_is_circulating_potential}.
	\emph{Вказівка}: скористайтесь формулами~\eqref{BrotA} та теоремою Стокса~\eqref{Stoksheorem}.
\end{problem}

%=========================================================
\begin{problem}
    Використовуючи вираз для вектор-потенціалу~\eqref{Supperposition_principle} та зв'язок~~\eqref{BrotA} виведіть закон Біо-Савара-Лапласа.
\end{problem}



%=========================================================
\begin{problem}
Знайти вектор-потенціал нескінченного провідника по якому тече струм $I$.
\begin{solution}
	$\vect{A} =-  \frac{2I}{c}\ln r \vect{k}$, $\vect{k}$~--- орт, напрямлений вздовж провідника зі струмом.
	% Вказівка, скористайтесь теоремою Стокса і обчисліть потік вектора \Bfield
\end{solution}
\end{problem}

%=========================================================
\begin{problem}
Знайти вектор-потенціал всередині і поза межами нескінченно довгого  соленоїда радіусом $R$, по якому протікає струм, який створює  однорідне магнітне поле з індукцією $\Bfield$. Покласти $\vect{A}$ = 0 на осі соленоїда.
\begin{solution}
	$\vect{A} = \frac12 \left[ \Bfield\times \vect{r} \right] $.
\end{solution}
\end{problem}

%=========================================================
\begin{problem}
    Знайти вектор потенціал колового провідника радіусом $R$ зі струмом $I$ в довільній точці простору в дипольному наближенні.
\begin{solution}
	$\vect{A} = \frac{\vect{p}_m\times\vect{r}}{r^3}$, де $\vect{p}_m = \frac{I\pi R^2}{c}\vect{n}$~-- дипольний момент колового витка, $\vect{n}$~-- вектор нормалі до поверхні витка.
\end{solution}
\end{problem}


%=========================================================
\begin{problem}
Використовуючи вираз для вектор-потенціалу~\eqref{dipole_vect_potential}, що створюється магнітним диполем $\vect{p}_m$, показати, що магнітне поле диполя виражається формулою~\eqref{dipole_Bfield}.
\emph{Вказівка}: скористайтесь формулою векторного аналізу~\eqref{rotvect}.
\begin{solution}
	З рівняння~\eqref{rotvect} випливає, що $\rot\vect{A} = \rot\left( \frac{\vect{p}_m\times\vect{r}}{r^3}\right)  = -(\vect{p}_m\cdot\vect{\nabla}) \frac{\vect{r}}{r^3}$.
	Розписуючи останній вираз  в декартовій системі координат, отримуємо шукану формулу.
\end{solution}
\end{problem}


%=========================================================
\begin{problem}
Уздовж осі довгого кругового циліндра радіусом $R$ тече струм з постійною густиною $\vect{j}$. Знайти вектор-потенціал магнітного поля в залежності від відстані $r$ до осі циліндра.
\begin{solution}
	При калібровці $\vect{A}(0) = 0$,

	$
		\vect{A}(r) = %
		\begin{cases}
			-\frac{1}{c}\pi r^2\vect{j},                                          & \quad r \le R \\
			-\frac{1}{c}2\pi R^2\left( \frac12 + \ln\frac{r}{R}\right) \vect{j} , & \quad r \ge R
		\end{cases}
	$.
\end{solution}
\end{problem}

%=========================================================
%\begin{problem}% Алексеев 172
%У сферичних координатах компоненти вектора $\vect{j}$ середньої об'ємної густини орбітального струму, що тече в збудженому атомі водню, дорівнюють:
%\[
%	j_r = j_{\theta} = 0, \quad
%	j_{\phi} = \frac{1}{2\cdot 3^8}\frac{e\hbar r^3}{\pi m a^7}e^{-\frac{2r}{3a}}\sin^3\theta.
%\]
%де $a$~-- борівський радіус, $\hbar$~-- постійна Планка, $m$ і $e$~-- маса і заряд електрона, а $r$~-- відстань до протона. Орбітальний струм створює в просторі магнітне поле. Знайти індукцію цього магнітного поля в центрі атома.
%\begin{solution}
%	$B = \frac{2e\hbar}{405mca^3}$.
%\end{solution}
%\end{problem}

%=========================================================
\begin{problem} %Алексеев 173
Середня густина заряду електронної хмари в атомі водню дорівнює $\rho = \frac{e}{\pi a^3}e^{-\frac{2r}{a}}$, де $ a $~-- борівський радіус, $ r $~-- відстань до протона, а $e$~-- заряд електрона. Якщо помістити атом у зовнішнє однорідне магнітне поле з індукцією $\Bfield_0$, то електронне хмара почне обертатися, що в свою чергу утворить в просторі струм з об'ємною густиною $\vect{j} = \frac{e\rho}{2mc} \vect{r}\times \Bfield_0$, де $m$~-- маса електрона. На яку величину $\Delta\Bfield$ зміниться індукція магнітного поля в центрі атома внаслідок обертання електронної хмари?
\begin{solution}
	$\Delta\Bfield = -\frac{e^2 \Bfield_0}{3amc^2}$.
\end{solution}
\end{problem}

%=========================================================
\begin{problem}
Сфера радіусом $R$, що заряджена рівномірно з поверхневою густиною $\sigma$ обертається з кутовою швидкістю $\omega$ навколо одного зі своїх діаметрів. Знайдіть вектор-потенціал та магнітне поле в середині та зовні сфери як функцію $\vect{r}$. Запишіть отримані вирази через магнітний момент.
\begin{solution}
	$\vect{A} =
		\begin{cases}
			\frac{4\pi}{3}R\omega\sigma r\sin\theta \vect{e}_{\phi} = \frac{\vect{p}_m\times\vect{r}}{R^3},              & r \le R \\
			\frac{4\pi}{3}R^4\omega\sigma \frac{\sin\theta}{r^2} \vect{e}_{\phi} = \frac{\vect{p}_m\times\vect{r}}{r^3}. & r \ge R \\
		\end{cases}$,

	$\Bfield =
		\begin{cases}
			\frac{8\pi}{3c}R\sigma\vect{\omega} = \frac{2\vect{p}_m}{R^3},     & r \le R \\
			\frac{3(\vect{p}_m\vect{r})\vect{r}}{r^5} -\frac{\vect{p}_m}{r^3}. & r \ge R \\
		\end{cases}$.
\end{solution}
\end{problem}

%=========================================================
\begin{problem} %Меледин 4.20
Вздовж осі провідника товщиною $d$, вигнутого в формі напівциліндра радіусом $R$ ($R \gg d$), тече струм з постійною густиною $j$. Знайти векторний потенціал на площині, що розташована уздовж осі напівциліндра і спирається на його краї.
\begin{solution}
	$\vect{A} = - \frac{2\pi jRd}{c}\ln R$,  $\vect{k}$~--- орт, напрямлений вздовж провідника зі струмом.
\end{solution}
\end{problem}

\section{Магнітне поле у речовині. Граничні умови}

\begin{Theory}\small

  Теорема про циркуляцію для вектора $\Bfield$ в речовині:
  \begin{equation}
	  \oint\limits_{L} \Bfield\cdot d\vect{l} = \frac{4\pi}{c}\left( \iint\limits_S \vect{j} \cdot d\vect{S} + \iint\limits_S \vect{j}'\cdot d\vect{S}\right) ,
  \end{equation}
  де $\vect{j}$~-- густина струмів провідності,  $\vect{j}'$~-- густина молекулярних струмів.

  Намагніченість $\vect{M}$~--- величина, що дорівнює дипольному моменту одиниці об'єму речовини:
	\begin{equation}
		\vect{M} = \frac{\vect{p}_m}{V}
	\end{equation}
  Циркуляція для вектора намагніченості $\vect{M}$:
  \begin{equation}
	  c\oint\limits_{L} \vect{M} \cdot d\vect{l} = \iint\limits_S \vect{j}' \cdot d\vect{S}.
  \end{equation}

  Диференціальна форма:
  \begin{equation}
	  c\rot\vect{M} = \vect{j}'.
  \end{equation}


  Означення вектора напруженості магнітного поля:
  \begin{equation}
	  \Hfield = \Bfield - 4\pi\vect{M}
  \end{equation}

  Теорема про циркуляцію для вектора $\Hfield$:
  \begin{equation}
	  \oint\limits_{L} \Hfield d\vect{l} = \frac{4\pi}{c}\iint\limits_S \vect{j} \cdot d\vect{S},
  \end{equation}
де $\vect{j}$~-- густина струмів провідності.

Диференціальна форма:
  \begin{equation}
	  \rot\vect{H} = \frac{4\pi}{c}\vect{j},
  \end{equation}

Поверхнева густина молекулярних струмів:
\begin{equation}
	\vect{i}' = c\left[\vect{n}\times(\vect{M}_2 - \vect{M}_1) \right],
\end{equation}
де $\vect{M}_1$ та $\vect{M}_2$~--- намагніченості магнетиків на поверхні їх розділу, $\vect{n}$~--- нормаль до розділу магнетиків, яка проведена від магнетика $1$ до магнетика $2$.

  Зв'язок між вектором намагнічування та вектором напруженості магнітного поля, \emph{яке зумовило намагнічування} (у випадку ізотропних лінійних магнетиків):

  \begin{equation}
	  \vect{M} = \chi_m\Hfield,
  \end{equation}
  де $\chi_m$~-- магнітна сприйнятливість магнетика.

  \begin{equation}
	  \mu = 1 + 4\pi\chi_m,
  \end{equation}
  магнітна проникність лінійного магнетика.

  Лінійними магнетиками є речовини, для яких $\mu \gtrapprox 1$, і які називаються  парамагнітними, та для яких $\mu \lessapprox 1$, які називаються діамагнітними. Парамагнітні речовини намагнічуюся вздовж магнітного поля, а діамагнітні~--- проти.

	Зв'язок між векторами  $\Bfield$ та $\Hfield$ у випадку лінійних магнетиків:
	\begin{equation}\label{linear_magnetics}
		\Bfield = \mu \Hfield.
	\end{equation}

	Зв'язок~\ref{linear_magnetics} у випадку слабих полів також використовується і для нелінійних магнетиків (наприклад, феромагнетиків).


  Граничні умови для векторів $\Bfield$ та $\Hfield$:
  \begin{align}
	  B_{1n}                & = B_{2n},         \label{B_boundary}\\
	  \left[ \vect n \times \vect{H}_{2}\right]  - \left[ \vect n \times \vect{H}_{1}\right]  & = \frac{4\pi}{c}\vect{i} \label{H_boundary}
  \end{align}
  $\vect i$~-- поверхнева густина струму провідності, який тече по границі розділу середовищ.
\end{Theory}

%=========================================================

%\subsection*{Магнітні властивості атомів та молекул}
%
%%=========================================================
%\begin{problem}
%    Знайдіть намагніченість зразка заліза в стані насичення та магнітне поле, яке вона створює  всередині зразка. Припустіть, що кожен атом заліза має магнітний момент, що дорівнює $1\mu_B$ (магнетону Бора).  Густина заліза $\rho = 7.87$~г/см\textsuperscript{2}, молярна маса $\mu = 55.8$~г/моль.
%\begin{solution}
%	$M = 787,36$~ерг/(Гс\cdot см\textsuperscript{3}),
%	$B = $
%\end{solution}
%\end{problem}

%=========================================================
\begin{problem}
	Отримайте граничні умови~\ref{B_boundary} та  \ref{H_boundary}.
\end{problem}

%=========================================================
%\begin{problem}
%    Покажіть, що поверхнева густина струму намагнічування що тече по поверхні циліндра, що має намагніченість $\vect{M} = \const$ визначається формулою:
%	\begin{equation}
%		i_m = c(\vect M \cdot \vect l),
%	\end{equation}
%	де $\vect l$~--- одиничний вектор, що напрямлений вздовж осі циліндра.
%\end{problem}


\subsection*{Принципові питання магнітостатики магнетиків}

%=========================================================
\begin{problem}
	Довгий циліндр виготовлений з магнетика з замороженою намагніченістю $\vect{M} = \const$, що напрямлена вздовж його осі. Знайти магнітну індукцію і напруженість магнітного поля всередині і зовні циліндра. Чому дорівнює об'ємна та поверхнева густина струмів намагнічування. Зобразити схематичну картину полів вектора $\Bfield$ та вектора $\Hfield$ в середині та зовні циліндра.
\begin{solution}
	$\Bfield = 4\pi \vect{M}$, $\Hfield = 0$, $\vect{j'} = 0$, $\vect{i}' = \vect{M}\times \vect{n}$.
\end{solution}
\end{problem}

%=========================================================
\begin{problem}
    Знайдіть намагніченість довгого циліндра, вміщеного в зовнішнє однорідне магнітне поле індукцією $\Bfield_0$. Магнітна проникність матеріалу циліндра дорівнює $\mu$.
\begin{solution}
	$\vect{M} = \frac{\mu - 1}{4\pi} \Bfield_0.$
\end{solution}
\end{problem}

%=========================================================
\begin{problem}
    Знайдіть намагніченість короткого циліндра, вміщеного в зовнішнє однорідне магнітне поле індукцією $\Bfield_0$. Магнітна проникність матеріалу циліндра дорівнює $\mu$.
\begin{solution}
	$\vect{M} = \frac{\mu - 1}{4\pi\mu} \Bfield_0.$
\end{solution}
\end{problem}

%=========================================================
\begin{problem}\label{prb:surface_of_magnetic}
	Поблизу плоскої поверхні однорідного магнетика з магнітною проникністю $\mu$ індукція магнітного поля в вакуумі дорівнює $\Bfield_0$ і складає кут $\theta$ з нормаллю до поверхні (\ref{surface_of_magnetic}). Вважаючи індукцію зовні та в середині магнетика однорідною, знайти \begin{enumerate*}[label=\alph*)]
		\item потік вектора $\Hfield$ через сферу радіуса $R$, центр якої лежить на поверхні магнетика,
		\item циркуляцію  вектора $\Bfield$ вздовж контура $\Gamma$ довжиною $l$, площина якого перпендикулярна до поверхні діелектрика.
	\end{enumerate*}
	\begin{solution}
	\begin{enumerate*}[label=\alph*)]
		\item $\oint\limits_{S} \Hfield d\vect{S} = \frac{\mu  -1}{\mu} \pi R^2 B_0 \cos\theta$,
		\item $\oint\limits_{\Gamma} \Bfield d\vect{r} = - \left( \mu -1\right)  l B_0 \sin\theta$.
	\end{enumerate*}
	\end{solution}
\end{problem}

%---------------------------------------------------------
\begin{figure}[h!]\centering
	\begin{tikzpicture}[scale=2]
		\def\R{0.9}
		\fill[gray!50] (-2,-1) rectangle (2,0);
		\node at (-1.9,-0.9) {$\mu$};
		\draw[ultra thick] (-2,0) -- (2,0);
		\coordinate (P) at (-1,0);
		\coordinate (Q) at (1.5,0);
		\draw[dashed] (P) circle (\R);
		\draw[-latex] (P) -- node[pos=0.4, below] {$R$} +(-45:\R) ;
		\draw[-latex] (P) -- +(0,0.5) node[left] {$\vect{n}$};
		\draw[-latex] (P) -- +(45:{\R+0.5}) node[above left] {$\Bfield_0$}  ;
		\draw (P) ++(45:0.25) arc (45:90:0.25) node[pos=0.1, above] {$\theta$};
		\draw[dashed] (Q) ++(0,-0.25) rectangle (0.5,0.25) node[above] {$\Gamma$};
		\draw[-latex] (Q)  ++(-0.7,0.25) -- +(0.3,0);
		\draw[-latex] (Q)  ++(-0.3,-0.25) -- +(-0.3,0);
	\end{tikzpicture}
	\caption{До задачі~\ref{prb:surface_of_magnetic}}
	\label{surface_of_magnetic}
\end{figure}
%---------------------------------------------------------

\subsection*{Визначення характеристик магнітного поля в присутності маг\-нетиків з заданою намагніченістю}

%=========================================================
\begin{problem}% Griffiths 6.8
Нескінченно довгий непровідний циліндр радіусом $R$ намагнічений таким чином, що намагніченість визначається законом $\vect{M} = ar^2\vect{e}_{\phi}$, де $r$~-- відстань від осі циліндра. Визначити індукцію та напруженість магнітного поля в залежності від відстані до осі в середині та зовні циліндра.
\begin{solution}
	$\Bfield = %
		\begin{cases}
			4\pi\vect{M}, & \quad r < R \\
			0,            & \quad r > R
		\end{cases}
	$,
	$\Hfield = 0$ в усьому просторі.
\end{solution}
\end{problem}

%=========================================================
\begin{problem}
Куля радіуса $R$ має однорідну намагніченість  $\vect{M}$. Знайти магнітну індукцію і напруженість магнітного поля всередині і зовні кулі.
\begin{solution}
	При $r \le R$, $\Bfield = \frac{8\pi}{3} \vect{M}$, $\Hfield = -\frac{4\pi}{3} \vect{M}$.
\end{solution}
\end{problem}

%=========================================================
\begin{problem}\label{prb:toroid}
Тонке тороїдальне осердя радіусом $R$ з намагніченістю $\vect{M}$ має зазор розміром $d$ ($d \ll R$) (рис.~\ref{toroid}). Нехтуючи розсіюванням магнітного поля на краях зазору, визначити напруженість та індукцію поля в осерді та зазорі.
\begin{solution}
	$\Hfield_\text{осердя} \approx - 4\pi \vect{M}\frac{d}{2\pi R}$,  $\Hfield_\text{зазор} \approx 4\pi \vect{M} \left( 1 - \frac{d}{2\pi R}\right) $, $\Bfield_\text{осердя} = \Bfield_\text{зазор} = \Hfield_\text{зазор}$.
\end{solution}
\end{problem}
%---------------------------------------------------------
\begin{figure}[h!]\centering
	\begin{tikzpicture}%[tdplot_main_coords]
		\draw[gray, fill=gray!50] (5:1.5) arc (5:355:1.5) --
		(355:2) arc (355:5:2) -- cycle;

		\draw[-latex] (0,0) -- node[left] {$R$} (240:1.5) ;

		\draw[-latex] (315:1.75) arc (315:45:1.75) node[below right] {$\vect{M}$};

		\draw[-latex, cyan!70!blue] (5:1.75) arc (5:-5:1.75);
		\draw[-latex, cyan!70!blue] (5:1.5) to [bend right] (-5:1.5) node[left, black] {$\Bfield$};;
		\draw[-latex, cyan!70!blue] (5:2) to [bend left] (-5:2);

		\draw[-latex] (20:2.3) arc (20:5:2.3);
		\draw[latex-] (-5:2.3) arc (-5:-20:2.3);
		\node at (0:2.4) {$d$};
		%\foreach \i in {-27,...,25} {
		%\draw[red, rotate=-5*\i] (-1.5,0) arc(-25:210:0.28 and 0.15) node[pos=0.6, rotate=-5*\i] {\midarrow};
		%}
	\end{tikzpicture}
	\caption{До задачі~\ref{prb:toroid}}
	\label{toroid}
\end{figure}
%---------------------------------------------------------

\subsection*{Визначення характеристик магнітного поля в присутності ізотропних лінійних маг\-нетиків}

%=========================================================
\begin{problem}
Постійний струм $I$ тече вздовж довгого циліндричного дроту круглого перерізу. Дріт виготовлений з парамагнетика сприйнятливістю $\chi$. Знайти:
\begin{enumerate*}[label=\alph*)]
	\item поверхневий молекулярний струм;
	\item об'ємний молекулярний струм.
\end{enumerate*}
Як ці струми напрямлені один відносно одного?
\begin{solution}
	\begin{enumerate*}[label=\alph*)]
		\item $I'_\text{пов} = 4\pi\chi I$;
		\item $I'_\text{об} = 4\pi\chi I$.
	\end{enumerate*}
	Струми напрямлені протилежно один відносно одного.
\end{solution}
\end{problem}


%=========================================================
\begin{problem}
Довгий соленоїд заповнений неоднорідним парамагнетиком, сприйнятливість якого залежить тільки від відстані до осі соленоїда як $\chi = ar^2$, де $a$~-- додатна постійна. На осі соленоїда індукція магнітного поля дорівнює $\Bfield_0$. Радіус соленоїда $R$. Знайти залежність від $r$:
\begin{enumerate*}[label=\alph*)]
	\item намагніченості магнетика;
	\item густину об'ємного струму в магнетику.
\end{enumerate*}
Знайдіть густину поверхневого струму. Який повний молекулярний струм на одиницю довжини?
\begin{solution}
	\begin{enumerate*}[label=\alph*)]
		\item $\vect{M} = ar^2 \Bfield_0$;
		\item $\vect{j}' = 2сa\left[ \vect{r} \times \Bfield_0\right] $.
	\end{enumerate*}
Поверхневий струм $i' = -caR \left[ \vect{R} \times \Bfield_0\right]$, де $\vect{R}$~--- радіус-вектор точок на поверхні. Поверхневий та об'ємний струми течуть в різних напрямках.
Повний молекулярний струм на одиницю довжини дорівнює нулю.
\end{solution}
\end{problem}


%=========================================================
\begin{problem} % Griffiths Textbook 6.3
Нескінченно довгий соленоїд, що має $n$ витків на одиницю довжини і по якому тече струм $I$ заповнений магнетиком з сприйнятливістю $\chi$. Знайдіть напруженість та індукцію магнітного поля у всьому просторі. Визначте величину і напрямок поверхневого струму намагніченості, який тече по поверхні магнетика. В якому випадку поверхневий струм буде протилежний напрямок до $I$?
\begin{solution}
	$\Hfield = \frac{4\pi}{c} nI \vect{e}_z$, $\Bfield = \frac{4\pi}{c} (1+\chi)nI \vect{e}_z$, $\vect{i}' = с\chi nI\vect{e}_{\phi}$.
\end{solution}
\end{problem}

%=========================================================
\begin{problem}
Нескінченний прямолінійний однорідний дріт радіусом $R$, виготовлений з матеріалу, яка має магнітну проникність $\mu_1$, знаходиться в непровідному нескінченному однорідному середовищі з магнітною проникністю $\mu_2$. По дроту тече постійний струм $I$. Знайти напруженість поля, магнітну індукцію, намагніченість, об'ємну  і поверхневу густини молекулярних струмів всередині дроту і зовні як функції радіальної координати~$r$.
\begin{solution}
	$
		H(r) = %
		\begin{cases}
			\frac{2I}{cR^2}r, & \quad r < R   \\
			\frac{2I}{cr},    & \quad r \ge R
		\end{cases}
	$,
	$
		B(r) = %
		\begin{cases}
			\mu_1\frac{2I}{c R^2}r, & \quad r < R   \\
			\mu_2\frac{2I}{c r},    & \quad r \ge R
		\end{cases}
	$,\\
	$
		M(r) = %
		\begin{cases}
			(\mu_1-1)\frac{2I}{c R^2}r, & \quad r < R   \\
			(\mu_2-1)\frac{2I}{c r},    & \quad r \ge R
		\end{cases}
	$,
	$
		j'(r) = %
		\begin{cases}
			(\mu_1-1)\frac{4I}{c R^2}, & \quad r < R   \\
			0,                         & \quad r \ge R
		\end{cases}
	$,

	$i' = \frac{2I}{R} (\mu_2 - \mu_1)$, при $r = R$.
\end{solution}
\end{problem}

%=========================================================
\begin{problem}\label{prb:wire_on_bound}
Прямий тонкий нескінченно довгий циліндр малого радіусу $R$, по якому тече струм $I$, лежить на поверхні плоского непровідного однорідного магнетика з магнітною проникністю $\mu$, що займає половину простору. Знайти намагніченість, магнітну індукцію, напруженість і молекулярні струми в усьому просторі.
\begin{solution}
	$M(r) = \frac{\mu - 1}{\mu + 1}\frac{4I}{r}$,
	$B(r) = \frac{\mu}{\mu + 1}\frac{4I}{cr}$,
	$H_1(r) = \frac{\mu}{\mu + 1}\frac{4I}{cr}$,
	$H_2(r) = \frac{1}{\mu + 1}\frac{4I}{cr}$,

	$I' = I \frac{\mu - 1}{\mu + 1}$.
\end{solution}
\end{problem}

%=========================================================
\begin{problem}
Коаксіальний кабель складається з двох циліндричних трубок радіусами $R_1$ та $R_2$, які розділені непровідним магнетиком з магнітною сприйнятливістю $\chi$. По трубкам тече струм $I$ в протилежних напрямках. Визначте магнітне поле в магнетику. Знайдіть струми намагнічування, які течуть на його поверхнях. Переконайтесь, що вільні струми та струми намагнічування створюють правильне значення магнітного поля.
\begin{solution}
	$B_r = (1 + 4\pi\chi)\frac{2I}{cr}$, $I'  = 4\pi\chi I$, на зовнішній та внутрішній поверхнях магнетика напрямок струмів намагнічування співпадає з напрямком вільних струмів.
\end{solution}
\end{problem}

%=========================================================
\begin{problem} %Griffiths 6.17
Струм $I$ тече вздовж прямого мідного провідника радіусом $R$. Магнітна сприйнятливість міді дорівнює $-9.7\cdot 10^{-6}$. Визначте магнітне поле в міді. Знайдіть повний струм намагнічування який тече вздовж провідника.
\begin{solution}
	$I' = 0$.
\end{solution}
\end{problem}

%=========================================================
\begin{problem}
    По круговому контуру радіусом $40$~м, що занурений в рідкий кисень, тече струм $1$~А. Визначте намагніченість в центрі цього контуру. Магнітна сприйнятливість рідкого кисню $3.4\cdot 10^{-3}$.
\end{problem}

%=========================================================
\begin{problem}
Циліндричний провідник радіусом $R$   перпендикулярно через плоску межу розділу двох магнетиків з проникностями $\mu_1$ та $\mu_2$. По провіднику тече постійний струм $I$. Знайти розподіл полів $\Hfield$ і $\Bfield$ у всьому просторі, та накреслити картину їх ліній.
\begin{solution}
	$\Hfield =
		\begin{cases}
			\frac{2Ir}{cR^2}\vect{e}_{\phi}, & r \le R \\
			\frac{2I}{cR}\vect{e}_{\phi},    & r > R
		\end{cases}$,

	$\Bfield =
		\begin{cases}
			\frac{2Ir}{cR^2}\vect{e}_{\phi},   & r \le R                                  \\
			\frac{2I\mu_1}{cR}\vect{e}_{\phi}, & r > R  \, (\text{в середовищі з } \mu_1) \\
			\frac{2I\mu_2}{cR}\vect{e}_{\phi}, & r > R  \, (\text{в середовищі з } \mu_2)
		\end{cases}$.
\end{solution}
\end{problem}


%=========================================================
\begin{problem}\label{sphere:Magnetic_in_magnetic}
В однорідне магнітне поле $\Bfield_0$ вноситься куля радіуса $R$ з магнітною проникністю $\mu_i$. Знайдіть магнітний момент кулі,  магнітне поле $\Bfield$ в усьому просторі та розподіл струму в кулі. Магнітна проникність навколишнього середовища $\mu_e$.
\begin{solution}
	$\vect{p}_m = \frac{\mu_i - \mu_e}{\mu_i + 2\mu_e}R^3\Bfield_0$,
	$\Bfield =
		\begin{cases}
			\frac{3\mu_i}{\mu_i + 2\mu_e} \Bfield_0,                                                    & r \le R, \\
			\Bfield_0 - \frac{\vect{p}_m}{r^3} + \frac{3\left(\vect{p}_m\vect{r}\right)\vect{r} }{r^5}, & r > R.
		\end{cases}	$

	Густина об'ємних струмів намагнічування $\vect{j}' = 0$, поверхнева густина струмів намагнічування $i = \frac{3c}{4\pi} \frac{\mu_i - \mu_e}{\mu_i + 2\mu_e} \frac{\Bfield_0\vect{r}}{R}$, де $\vect{r}$~-- радіус-вектор поверхні провідника.
\end{solution}
\end{problem}


%=========================================================
\begin{problem}
Знайти магнітне поле в сферичній порожнини радіусом $R$, що знаходиться в зовнішньому магнітному полі $\Bfield_0$. Магнітна проникність середовища, що оточує порожнину, дорівнює $\mu$.
\begin{solution}
	$\Bfield = \frac{3\mu}{1 + 2\mu}\Bfield_0$.
\end{solution}
\end{problem}



\subsection*{Визначення характеристик магнітного поля в присутності ізотропних нелінійних маг\-нетиків}


%=========================================================
\begin{problem}\label{prb:Irod2.304}
Постійний магніт має вигляд кільця з вузьким поперечним зазором шириною $d$. Середній радіус кільця $R$ ($R \gg d$) (рис.~\ref{pic:Irod2.304a}). Залишкова намагніченість матеріалу магніту $M_r$, його коерцитивної сила $H_c$. Вважаючи, що залежність $M(H)$ на ділянці від $H_c$ до нуля (рис.~\ref{pic:Irod2.304b}) є лінійною і розсіювання магнітного поля на краях зазору немає, знайти індукцію магнітного поля в зазорі.
\begin{solution}
	$B = \frac{4\pi M_r}{1 + \frac{2d M_r }{RH_c}}$.
\end{solution}
\end{problem}
%---------------------------------------------------------
\begin{figure}[h!]\centering
	\subfloat[]{
		\newcommand{\midarrow}{\tikz \draw[-stealth] (0,0) -- +(0.05,0);}
		\begin{tikzpicture}
			\draw[gray, fill=gray!50] (5:1.5) arc (5:355:1.5) --
			(355:2) arc (355:5:2) -- cycle;

			\draw[-latex] (0,0) -- node[left] {$R$} (240:1.75) ;

			\draw[-latex, cyan!70!blue] (5:1.75) arc (5:-5:1.75);
			\draw[-latex, cyan!70!blue] (5:1.5) to [bend right] (-5:1.5) node[left, black] {$\Bfield$};;
			\draw[-latex, cyan!70!blue] (5:2) to [bend left] (-5:2);

			\draw[-latex] (20:2.3) arc (20:5:2.3);
			\draw[latex-] (-5:2.3) arc (-5:-20:2.3);
			\node at (0:2.4) {$d$};

			\node at (0,-2.5) {};
		\end{tikzpicture}
		\label{pic:Irod2.304a}
	}
	\quad
	%---------------------------------------------------------
	\subfloat[]{
		\begin{tikzpicture}[scale=0.7]
			\begin{axis}[   axis lines = center,
					%axis y line = left,
					%axis x line = bottom,
					%grid = both,
					clip = false,
					ylabel={$M$}, ylabel style = {above left},
					xlabel={$H$}, xlabel style = {below right},
					xmin=-10, xmax=1,
					ymin=-1,ymax=10,
					xtick = \empty,
					ytick = \empty,
				]
				\draw[thick, red] plot [smooth] coordinates {(-10,-1.5)   (-8,0)   (0,8) (2,9.3)  };
				\node[below] at (-8,0) {$H_c$};
				\node[left]  at (0,8) {$M_r$};
			\end{axis}
		\end{tikzpicture}
		\label{pic:Irod2.304b}
	}
	\caption{До задачі~\ref{prb:Irod2.304}.}
\end{figure}
%---------------------------------------------------------
%=========================================================
\begin{problem}\label{prb:Kozel6.14}% Козел 6.14
Феромагнітний матеріал має залишкову намагніченість $ M_0 = 500$~Гс та коерцитивну силу $H_0 = 500$~Е. Крива  намагніченості є чвертю кола (рис.~\ref{Kozel6.14}). Із цього матеріалу виготовлено  постійний магніт, що має вигляд тору квадратного перерізу з поперечним розрізом. Внутрішній радіус тора $R_1~=~1.5$~см,  зовнішній~-- $R_2 = 2.5$~см, ширина розрізу $d = 0.5$~cм (рис.~\ref{pic:Kozel6.14}). Нехтуючи розсіюванням магнітного поля на краях зазору, визначити величину магнітного поля в зазорі.
%\begin{solution}
%$B = 4\pi M_0\frac{1 - \frac{1}{4\pi}\frac{4d}{R_1 + R_2}}{\sqrt{1 + \left( \frac{4d}{R_1 + R_2} \right)^2 }}$.
%\end{solution}
\end{problem}
%---------------------------------------------------------
\begin{figure}[h!]\centering
	\subfloat[]{
		\begin{tikzpicture}%[tdplot_main_coords]

			\draw[gray, fill=gray!50] (10:1.5) arc (10:350:1.5) --
			(354:2.5) arc (354:6:2.5) -- cycle;

			\draw[-latex] (0,0) -- node[left] {$R_1$} (240:1.5) ;
			\draw[-latex] (0,0) -- node[above right] {$R_2$} (180:2.5) ;

			\draw[-latex, cyan!70!blue] (9.5:1.5) to [bend right] node [left, black] {$\Bfield$} (-9.5:1.5) ;
			\draw[-latex, cyan!70!blue] (8:2) -- (-8:2);
			\draw[-latex, cyan!70!blue] (6.5:2.5) to [bend left] (-6.5:2.5);

			%	\draw[-latex, cyan!70!blue] (10:1.5) to [bend right] (-10:1.5) node[left, black] {$\Bfield$};;
			%	\draw[-latex, cyan!70!blue] (5:2) to [bend left] (-10:2);

			%	\draw[-latex] (20:2.3) arc (20:5:2.3);
			%	\draw[latex-] (-5:2.3) arc (-5:-20:2.3);
			%	\node at (0:2.4) {$d$};
		\end{tikzpicture}
		\label{pic:Kozel6.14}
	}
	%---------------------------------------------------------
	\qquad
	\subfloat[]{
		\begin{tikzpicture}[scale=0.7]
			\begin{axis}[   axis lines = center,
					%axis y line = left,
					%axis x line = bottom,
					%grid = both,
					clip = false,
					ylabel={$M$, Гс}, ylabel style = {left},
					xlabel={$H$, Е}, xlabel style = {below},
					xmin=-30, xmax=10,
					ymin=-5,ymax=30,
					xtick = \empty,
					ytick = \empty,
				]
				%\addplot[thick, red, domain={-26:0}, samples=500] {25*sqrt(1-(x^2/25^2))};
				\draw [thick, red] (180:25) arc (180:90:25);
				\node[below] (H0) at (-25,0) {$H_0$};
				\node[right] (M0) at (0,25) {$M_0$};
			\end{axis}
		\end{tikzpicture}
		\label{Kozel6.14}
	}
	\caption{До задачі~\ref{prb:Kozel6.14}.}
\end{figure}
%---------------------------------------------------------

%=========================================================
\begin{problem}\label{prb:toroid_coil_saturation}
На тонке осердя  довжиною $l$ з зазором $d$ намотана котушка з числом витків $N$, по якій тече струм~(рис.~\ref{toroid_coil_saturation}). Залежність $M(H)$ для деякого подана на рис.~\ref{M(H)_toroid_coil}. Нехтуючи розсіюванням магнітного поля на краях зазору, знайти значення сили струму, при якому настане насичення осердя. Як буде змінюватися магнітна індукція $B$ в щілині осердя при $I>I_0$? Величини $M_0$ та $H_0$ вважати заданими.
\begin{solution}
	$I_0 = \frac{cl}{4\pi N} \left(H_0 + 4\pi M_0 \frac{d}{l}\right)$, $B = \frac{4\pi N}{cl} I + 4\pi M_0 \left(1 - \frac{d}{l}\right)$.
\end{solution}
\end{problem}
%---------------------------------------------------------
\begin{figure}[h!]\centering
	\subfloat[]{
		\newcommand{\midarrow}{\tikz \draw[-stealth] (0,0) -- +(0.05,0);}
		\begin{tikzpicture}
			\draw[gray, fill=gray!50] (5:1.5) arc (5:355:1.5) --
			(355:2) arc (355:5:2) -- cycle;

			%\draw[-latex] (0,0) -- node[left] {$R$} (240:1.5) ;

			\draw[-latex, cyan!70!blue] (5:1.75) arc (5:-5:1.75);
			\draw[-latex, cyan!70!blue] (5:1.5) to [bend right] (-5:1.5) node[left, black] {$\Bfield$};;
			\draw[-latex, cyan!70!blue] (5:2) to [bend left] (-5:2);

			\draw[-latex] (20:2.3) arc (20:5:2.3);
			\draw[latex-] (-5:2.3) arc (-5:-20:2.3);
			\node at (0:2.4) {$d$};
			\foreach \i in {-27,...,25} {
					\draw[red, rotate=-5*\i] (-1.5,0) arc(-25:210:0.28 and 0.15) node[pos=0.5, rotate=-5*\i] {\midarrow};
				}

			\node at (0,-2.5) {};
		\end{tikzpicture}
		\label{toroid_coil_saturation}
	}
	\quad
	%---------------------------------------------------------
	\subfloat[]{
		\begin{tikzpicture}[scale=0.7]
			\begin{axis}[   axis lines = center,
					%axis y line = left,
					%axis x line = bottom,
					%grid = both,
					clip = false,
					ylabel={$M$}, ylabel style = {above left},
					xlabel={$H$}, xlabel style = {below right},
					xmin=0, xmax=25,
					ymin=0,ymax=25,
					xtick = \empty,
					ytick = \empty,
				]
				\addplot[thick, red, domain={0:10}, samples=100, name path = curve] {-1/5*(x-10)^2+20};
				\addplot[thick,red, domain={10:25}] {20};
				\node[below] (H0) at (10,0) {$H_0$};
				\node[left] (M0) at (0,20) {$M_0$};
				\draw[dashed] (H0) -- ({M0}-|{H0});
				\draw[dashed] (M0) -- ({M0}-|{H0});
			\end{axis}
		\end{tikzpicture}
		\label{M(H)_toroid_coil}
	}
	\caption{До задачі~\ref{prb:toroid_coil_saturation}.}
\end{figure}
%---------------------------------------------------------

%=========================================================
\begin{problem}\label{prb:toroid_coil}
На тонке залізне осердя  довжиною $l = 1$~м з зазором $d = 1$~мм намотана котушка з числом витків $N = 1600$, по якій тече струм $I =1$~А  (рис.~\ref{toroid_coil}). Залежність $B(H)$ заліза подана на рис.~\ref{B(H)_toroid_coil}. Нехтуючи розсіюванням магнітного поля на краях зазору, визначити робочу точку залізного осердя. Знайти його магнітну проникність для цієї точки.
\begin{solution}
	На рисунку показана робоча точка залізного осердя ($H \approx 5$~Е, $B~\approx~15$~кГс). $\mu = 3000$.
	\begin{center}
		\begin{tikzpicture}[scale=0.7]
			\begin{axis}[   axis y line = left,
					axis x line = bottom,
					grid = both,
					ylabel={$B$, кГс},
					xlabel={$H$, Е},
					xtick = {0,5,...,25},
					ytick = {0,5,...,25},
				]

				\addplot[thick, red, domain={0:25}, samples=100, name path = curve] {25*sqrt(1-((x-25)^2/25^2))};
				\addplot[blue, domain={0:20}, name path = line] {20 - x};
				\path [name intersections={of=line and curve, by=P}];
%				\fill[red] (P) circle (0.05cm);
			\end{axis}
		\end{tikzpicture}
	\end{center}
\end{solution}
\end{problem}
%---------------------------------------------------------
\begin{figure}[h!]\centering
	\subfloat[]{
		\newcommand{\midarrow}{\tikz \draw[-stealth] (0,0) -- +(0.05,0);}
		\begin{tikzpicture}
			\draw[gray, fill=gray!50] (5:1.5) arc (5:355:1.5) --
			(355:2) arc (355:5:2) -- cycle;

			%\draw[-latex] (0,0) -- node[left] {$R$} (240:1.5) ;

			\draw[-latex, cyan!70!blue] (5:1.75) arc (5:-5:1.75);
			\draw[-latex, cyan!70!blue] (5:1.5) to [bend right] (-5:1.5) node[left, black] {$\Bfield$};;
			\draw[-latex, cyan!70!blue] (5:2) to [bend left] (-5:2);

			\draw[-latex] (20:2.3) arc (20:5:2.3);
			\draw[latex-] (-5:2.3) arc (-5:-20:2.3);
			\node at (0:2.4) {$d$};
			\foreach \i in {-27,...,25} {
					\draw[red, rotate=-5*\i] (-1.5,0) arc(-25:210:0.28 and 0.15) node[pos=0.5, rotate=-5*\i] {\midarrow};
				}

			\node at (0,-2.5) {};
		\end{tikzpicture}
		\label{toroid_coil}
	}
	\quad
	%---------------------------------------------------------
	\subfloat[]{
		\begin{tikzpicture}[scale=0.7]
			\begin{axis}[   axis y line = left,
					axis x line = bottom,
					grid = both,
					ylabel={$B$, кГс},
					xlabel={$H$, Е},
					xtick = {0,5,...,25},
					ytick = {0,5,...,25},
				]

				\addplot[thick, red, domain={0:25}, samples=100, name path = curve] {25*sqrt(1-((x-25)^2/25^2))};
				%				\addplot[blue, domain={0:25}, name path = line] {20 - x};
				%				\path [name intersections={of=line and curve, by=P}];
				%				\fill[red] (P) circle (0.05cm);
			\end{axis}
		\end{tikzpicture}
		\label{B(H)_toroid_coil}
	}
	\caption{До задачі~\ref{prb:toroid_coil}.}
\end{figure}
%---------------------------------------------------------

%=========================================================
\begin{problem}\label{prb:elmagnet}
    Електромагніт з тонким осердям квадратного перерізу зі стороною $a$, виготовлений з матеріалу з великою магнітною проникністю $\mu$, має тонкий плоский поперечний зазор ширини $d$, в якому створюється магнітне поле з індукцією $B$ (рис.~\ref{elmagnet}). Оцінити в дипольному наближенні магнітну індукцію в точці $P$, що лежить в площині зазору на великій відстані $r$ від його центру ($r \gg a$, $R \gg  a \gg  d$, де $R$ --- середній радіус тора).
\begin{solution}
	$p_m = \frac{\mu - 1}{\mu} B a^2 d$, $B = \frac{p_m}{r^3}$.
\end{solution}
\end{problem}
%---------------------------------------------------------
\begin{figure}[h!]\centering
		\begin{tikzpicture}%[tdplot_main_coords]
	\newcommand{\midarrow}{\tikz \draw[-stealth] (0,0) -- +(0.05,0);}
			\draw[gray, fill=gray!50] (5:1.5) arc (5:355:1.5) --
			(356:2) arc (356:4:2) -- cycle;

			\draw[-latex] (0,0) -- node[left] {$R$} (240:1.75) ;
			\draw[dash dot] (0,0)  circle (1.75);
			%\draw[-latex] (0,0) -- node[above right] {$R_2$} (180:2.5) ;

%			\draw[-latex, cyan!70!blue] (9.5:1.5) to [bend right] node [left, black] {$Bfield$} (-9.5:1.5) ;
%			\draw[-latex, cyan!70!blue] (8:2) -- (-8:2);
%			\draw[-latex, cyan!70!blue] (6.5:2.5) to [bend left] (-6.5:2.5);

%				\draw[-latex, cyan!70!blue] (10:1.5) to [bend right] (-10:1.5) node[left, black] {$Bfield$};;
%				\draw[-latex, cyan!70!blue] (5:2) to [bend left] (-10:2);

				\draw[-latex] (1.6,0.5) -- +(0,-0.37);
				\draw[-latex] (1.6,-0.5) -- +(0,0.37);
				\node at (0:1.3) {$d$};
				\draw[-latex] (0:1.75) -- node[below] {$r$} +(2,0) node[below] {$P$};
			\foreach \i in {-27,...,25} {
					\draw[red, rotate=-5*\i] (-1.5,0) arc(-25:210:0.28 and 0.15) node[pos=0.5, rotate=-5*\i] {\midarrow};
				}
		\end{tikzpicture}
\caption{До задачі~\ref{prb:elmagnet}}
\label{elmagnet}
\end{figure}
%---------------------------------------------------------

\section{Пондеромоторні сили в магнітному полі. Енергія та тиск поля}

\begin{Theory}\small
Сила, з якою магнітне поле діє на провідник зі струмом в магнітному полі (сила Ампера):
\begin{equation}
	\vect{F} = \frac1c \iiint\limits_V \vect{j} dV \times\Bfield.
\end{equation}


\begin{Attention}
	Хоча магнітне поле не являється потенціальним, однак, формально можна  (за \emph{відсутності феромагнетиків}) обчислити роботу, яку виконує сила Ампера, по переміщенню контуру в магнітному полі як:
	\begin{equation}
		A = - \frac{I}{c}(\Phi_2 - \Phi_1),
	\end{equation}
де $\Phi_1$ та $\Phi_2$~--- потоки магнітного поля, що пронизують контур в початковому та кінцевому положеннях.

Оскільки робота дорівнює зменшенню потенціальної енергії, формально можна ввести поняття <<потенціальної енергії>>, яка в даному випадку доцільно називати потенціальною функцією контуру в магнітному полі:
\begin{equation}
	U_m = -\frac1c I\Phi.
\end{equation}
\end{Attention}

Пондеромоторні сили (узагальнені), що діють на магнетики, або струми з боку поля:
\begin{equation}
	Q_i = - \left(\frac{\partial U_m}{\partial q_i} \right)_{\Phi} =   \left(\frac{\partial U_m}{\partial q_i} \right)_I,
\end{equation}
де $q_i$~-- узагальнені координати. Похідні беруться за умови постійного потоку (індекс $\Phi$), або постійного струму (індекс $I$).

Магнітна енергія системи електричних струмів виражається як:
\begin{equation}
	W_m =  \frac{1}{2c} \int\limits_V \vect{j}\cdot\vect{A}\,dV.
\end{equation}

Густина енергії магнітного поля:

\begin{equation}
	w_m = \frac{\Hfield\cdot \Bfield}{8\pi}.
\end{equation}

Сила, що діє на одиницю площі провідника, по якому тече струм:

\begin{equation}
	\vect{f}_{\sigma} = ( w_1 - w_2)\vect{n},
\end{equation}
де $w_1$ та $w_2$густини енергії магнітного поля по різні боки поверхні провідника, $\vect{n}$~-- вектор нормалі до поверхні розділу середовищ, напрямлений від середовища $1$ до середовища $2$.

Сила, що діє на диполь в магнітному полі:

\begin{equation}
	\vect{F} = (\vect{p}_m\cdot\vect{\nabla})\Bfield.
\end{equation}

Енергія жорсткого диполя в магнітному полі:

\begin{equation}
	W_m = -\vect{p}_m\cdot\Bfield.
\end{equation}

Енергія квазіпружного диполя в магнітному полі:

\begin{equation}
	W_m = -\frac12 \vect{p}_m\cdot\Bfield.
\end{equation}

Об'ємна густина пондеромоторних сил, що діють на лінійні магнетики:
\begin{equation}
	\vect{f}  = \frac{\mu - 1}{8\pi\mu}\vect{\nabla}B^2,
\end{equation}
де $\mu$~-- магнітна проникність магнетика.

Сила, що діє на одиницю площі поверхні розділу двох середовищ:
\begin{equation}
	\vect{f}_{\sigma} = \frac{1}{8\pi}\left( \frac{1}{\mu_2} - \frac{1}{\mu_1} \right) B^2_n\vect{n} + \frac{\mu_1 - \mu_2}{8\pi} H^2_{\tau}\vect{n},
\end{equation}
де $\vect{n}$~-- вектор нормалі до поверхні розділу середовищ, напрямлений від середовища $1$ до середовища $2$.
\end{Theory}



%=========================================================
\begin{problem}%
    По провіднику у формі кільця радіусом $R = 20$~см, що
містить $N = 500$ витків, тече струм силою $І = 1$~А. Визначити
об'ємну густину енергії магнітного полі в центрі кільця. Вважайте, що кільце розташоване у вакуумі.
%\begin{solution}
%
%\end{solution}
\end{problem}


\subsection*{Сила взаємодії між провідниками зі струмом}



%=========================================================
\begin{problem}
    Знайти силу (на одиницю довжини) взаємодії між двома довгими паралельними провідниками, по яким течуть однакові за величиною струми $I$ в одному (різних) напрямках.  Відстань між провідниками дорівнює $r$.  Визначте енергію взаємодії провідників.
\begin{solution}
	$\frac{F}{l} = \frac{2\pi I^2}{c^2r}$,
	$\frac{W}{l} = \frac{2\pi I^2}{c^2}\ln \left( \frac{r}{r_0}\right) $, де $r_0$~--- значення відстані, прийняте за нульовий рівень енергії.
\end{solution}
\end{problem}

%=========================================================
\begin{problem}
    Знайти силу  (на одиницю довжини) взаємодії між двома довгими паралельними провідниками, по яким течуть однакові за величиною струми $I$ в одному напрямку. Причому, один з провідників є лінійним, а інший являє собою тонку смужку, шириною $b$. Відстань від лінійного провідника до найближчого краю смуги дорівнює $a$.
\begin{solution}
	$\frac{F}{l} = \frac{2I^2}{c^2b}\ln\left( \frac{a + b}{a}\right) $.
\end{solution}
\end{problem}


%=========================================================
\begin{problem} % Меледин Черкасский 5.33
Вздовж довгого суцільного циліндричного провідника радіусом~$R$ тече струм $I$. Знайти тиск на поверхні провідника.
\begin{solution}
	$p = \frac{I^2}{2\pi c^2R^2}$.
\end{solution}
\end{problem}

%=========================================================
\begin{problem} % Колмаков 3.36
Вздовж довгого рідкого циліндричного провідника радіусом~$R$ тече струм $I$. Як змінюється тиск в середині рідини залежно від відстані до осі провідника.
\begin{solution}
	$p = \frac{I^2(R^2 - r^2)}{R^4\pi c^2}$.
\end{solution}
\end{problem}

%=========================================================
\begin{problem}
    Плоска рамка довільної форми площею $S$,  по якій тече  струм $I$ вміщена в однорідне магнітне поле $B$ так, що може вільно обертатись навколо осі, яка перпендикулярна до магнітного поля. В початковий момент часу кут між нормаллю до рамки і напрямком магнітного поля дорівнює $\theta$. Знайдіть узагальнену силу, що діє на рамку.
\begin{solution}
	Узагальненою силою є момент сил, що діє з боку магнітного поля $N = -\frac1c ISB\sin\theta$.
\end{solution}
\end{problem}


%=========================================================
\begin{problem} % Меледин-Черкасский
Квадратна рамка зі стороною $a$, яка виготовлена з тонкого провідника, розташована в одній площині з довгим прямим провідником, по якому тече струм $I_0$. По рамці, в свою чергу, тече струм $I$, а відстань між провідником та найближчою до нього стороною рамки дорівнює $2a$. Визначити
\begin{enumerate*}[label=\alph*)]
	\item силу, яка діє на рамку з боку провідника,
	\item роботу сил поля при переміщенні рамки на нескінченність.
\end{enumerate*}
\end{problem}

%=========================================================
\begin{problem}
    По двом нескінченним паралельним площинам течуть струми з лінійною густиною $i$. Знайдіть силу, що діє на одиницю поверхні кожної площини.
\begin{solution}
	$p = \frac{2\pi i^2}{c^2}$.
\end{solution}
\end{problem}


%=========================================================
\begin{problem}\label{prb:ABCD} % Сивухін 337
Дротяний контур $ ABCD $ в формі квадрата (рис.~\ref{ABCD}) знаходиться в магнітному полі довгого тонкого стрижня з магнітним моментом одиниці об'єму $ M_0 $ і перерізом $ S $, причому північний полюс $ \mathbf{N} $ магніту знаходиться в центрі квадрата, а сам магніт перпендикулярний до площини дротяного контуру. До протилежних кінців діагоналі $ AC $ підключають джерело ЕРС, внаслідок чого по сторонам контуру йде струм сили $ I $. Знайти момент пари сил $ N $, що обертають контур, і його напрямок.
\begin{solution}
	$N  = 2I  M_0 S$.
\end{solution}
\end{problem}
%---------------------------------------------------------
\begin{figure}[h!]\centering
	\newcommand{\midarrow}{\tikz \draw[-triangle 45] (0,0) -- +(.1,0);}
	\begin{tikzpicture}[scale=0.7, >=latex', every node/.style={sloped,allow upside down}]
		\draw [thick] (-2,2) node [below right] {$A$} -- node [] {\midarrow} node [above left] {$I$} (2,2) node [above] {$B$} -- node [] {\midarrow} (2,-2);
		\draw [thick] (-2,2) -- node [] {\midarrow} node [above left, rotate = 90] {} (-2,-2) node [below] {$D$} -- node [] {\midarrow}  (2,-2) node [above left] {$C$};
		\draw[decoration={aspect=0.3, segment length=2mm, amplitude=3mm,coil},decorate] (2,-2) -- (3,-3);
		\draw[decoration={aspect=0.3, segment length=2mm, amplitude=3mm,coil},decorate] (-3,3) -- (-2,2);
		\draw [fill, blue] (0,0) circle (0.1) node [below right, black]{$\mathbf{N}$};
	\end{tikzpicture}
	\caption{До задачі~\ref{prb:ABCD}.}
	\label{ABCD}
\end{figure}
%---------------------------------------------------------

%=========================================================
\begin{problem}% Сивухін 338
У магнітному полі магніту задачі~\ref{prb:ABCD} поміщений круговий провідник радіусом $ R $, по якому тече струм $ I $ за  годинниковою стрілкою, якщо дивитися з боку північного полюса магніту. Магніт розташований по осі кругового струму, і його північний полюс розташований на відстані $ d $ від центру провідника. Визначити сили, що діють на провідник з боку магніту.
\begin{solution}
	$F_\parallel/l  = I\cdot  B_\bot = I\cdot B\cos\phi$.
\end{solution}
\end{problem}

%=========================================================
\begin{problem}
Котушку зі струмом $I$ помістили в однорідне магнітне поле так, що її вісь співпала з напрямком поля. Обмотка котушки має один шар з мідного дроту діаметром $d$, радіус витків $R$. При якому значенні індукції зовнішнього магнітного поля обмотка котушки може бути розірвана? Межу міцності міді на розрив $\sigma_{\max}$ вважати відомою.
\begin{solution}
	$B = \frac{c\pi d^2 \sigma_{\max}}{4RI}$.
\end{solution}
\end{problem}

\subsection*{Дії магнітного поля на магнітні диполі. Взаємодія диполів}
%=========================================================
\begin{problem}
Соленоїд радіусом $R$ і довжиною $l$ ($l \gg R$) має обмотку, що складається з $N$ витків. По соленоїду тече струм $I$. У центрі соленоїда на його осі розміщена невелика котушка, яка має магнітний момент $\vect{p}_m$, і розташована перпендикулярно осі соленоїда. Визначити величину моменту сили, що діє на котушку.
\begin{solution}
	$M = \frac{4\pi p_m NI}{cl}$.
\end{solution}
\end{problem}

%=========================================================
\begin{problem}
Невеликий виток зі струмом знаходиться на відстані $r$ від довгого прямого провідника зі струмом $I$. Магнітний момент витка дорівнює $\vect{p}_m$. Знайти силу і момент сили, що діє на виток, якщо:
\begin{enumerate*}[label=\alph*)]
	\item $\vect{p}_m$ паралельний прямому провіднику,
	\item $\vect{p}_m$ напрямлений по радіусу-вектору $\vect{r}$,
	\item $\vect{p}_m$ збігається за напрямком з магнітним полем струму $I$ в місці розташування витка.
\end{enumerate*}
\begin{solution}
	\begin{enumerate*}[label=\alph*)]
		\item $\vect{F} = 0$, $\vect{M} = - \frac{2I}{cr^2} \vect{r} (\vect{p} \cdot \vect{k})$,
		\item $\vect{F} = - \frac{2I}{cr^2} \left[ \vect{k} \times \vect{p}\right]$, $\vect{M} = 0$,
		\item $\vect{F} = \frac{2I}{cr^2} \left[ \vect{k} \times \vect{p}\right]$, $\vect{M} = 0$,
	\end{enumerate*}
	де $\vect{k}$~--- орт, напрямлений вздовж струму.
\end{solution}
\end{problem}

%=========================================================
\begin{problem}
    Дві котушки з магнітними моментами $\vect{p}_1$ і $\vect{p}_2$ розташовані довільним чином одна відносно одної. Відстань $l$ між котушками велика в порівнянні з їх розмірами. Визначити силу взаємодії між ними. Розгляньте можливі часткові випадки
\end{problem}

%=========================================================
\begin{problem}
    На осі кругового витка радіусом $R$, по якому тече струм $I$, знаходиться невелика котушка зі струмом, що має магнітний момент $\vect{p}_m$, орієнтований вздовж осі витка. Знайти модуль сили, що діє на котушку, якщо її відстань від центру витка дорівнює $z$.
\begin{solution}
	$\vect{F} = -\frac{6\pi}{c}\frac{IR^2z}{(R^2 + z^2)^{5/2}}\vect{p}_m$.
\end{solution}
\end{problem}



%%=========================================================
%\begin{problem}
%    Нескінченна плоска пластина-магніт товщиною $d$ намагнічена так, що вектор намагніченості $\vect{M}$ утворює кут $\theta$ з нормаллю до її поверхні. Знайти магнітну енергію одиниці площі пластини.
%\begin{solution}
%
%\end{solution}
%\end{problem}

\subsection*{Пондеромоторні сили, що діють на магнетики в магнітному полі}

%=========================================================
\begin{problem}\label{prb:Fmax}
В установці, наведеній на рис.~\ref{pic:Fmax}, вимірюють силу, з якою парамагнітна кулька об’ємом \( V \) притягується до полюса електромагніту $M$. Індукція магнітного поля на осі полюса залежить від відстані за законом \( B = B_0 \exp(-a z^2) \), де константи \( B_0 \) та \( a \) відомі.

Знайти:
\begin{enumerate}[label=\alph*)]
    \item на якій висоті \( z \) потрібно помістити кульку, щоб сила притягання була максимальною;
    \item магнітну сприйнятливість \( \chi \) парамагнетика, якщо максимальна сила притягання дорівнює \( F_{\text{max}} \).
\end{enumerate}
\begin{solution}

\begin{enumerate}[label=\alph*)]
    \item $z = \frac{1}{2\sqrt{a}}$;
    \item $\chi = \frac{F_{\max}}{V B_0^2 }\sqrt{\frac{e}{a}}$.
\end{enumerate}


\end{solution}
\end{problem}

%---------------------------------------------------------
\begin{figure}[h!]\centering
    \begin{tikzpicture}[>=latex, midarrow/.style={%
       				postaction={ decorate,
       						decoration={ markings, mark=at position .7 with
       									{\arrow{stealth}}}}}]

    \begin{scope}[rotate=90]
        \draw[inner color=red, middle color=white, outer color=red!20] (1,0) ellipse(0.2 and 1) ;
        \fill[draw, shading=axis, top color=red, middle color=red, bottom color=red!50] (0, 1) -- ++(1, 0) arc(90:270: 0.2 and 1) -- ++(-1, 0);

                   				\foreach \y in {-4,...,4}{
                   						\draw[red!20, midarrow, line width=0.5pt] plot[domain=1:6] (\x,
                   						0.2*\y+0.01*\y*\x^2);
                   					}
        \node[circle, ball color=red, inner sep=0.1cm, draw] (C) at (3,0) {} ;


        \draw[ultra thick, gray] (C) -- ++(2, 0) ;

        \draw[ultra thick] (5.15,0.1) arc(45:270:0.1) coordinate (H);

        \fill[draw, shading=axis, top color=gray!50, middle color=gray, bottom color=gray!50] (H) -- ++({-90-10}:2) to[bend right] ++({4*sin(10)}, 0) -- cycle;

    \end{scope}

    \end{tikzpicture}
\caption{Рисунок до задачі~\ref{prb:Fmax}}
\label{pic:Fmax}
\end{figure}
%---------------------------------------------------------


%=========================================================
\begin{problem}\label{prb:ballunderring_magnetic}
На осі симетрії тонкого кільця радіусом $R$, по якому тече струм $I$, на відстані $z$ від його центру розташована кулька з магнетика радіусом $r$ ($r \ll R$) і магнітною проникністю $\mu$~(рис.~\ref{ballunderring_magnetic}). Яка сила діє на кульку з боку магнітного поля?
\begin{solution}
	$\vect F = - \left( \frac{2\pi IR^2}{c}\right)^2 \frac{3z}{(z^2 + R^2)^4} \vect k.$
\end{solution}
\end{problem}

%=========================================================
\begin{problem}\label{prb:diamagnetic_in_field}
    Тонкий діамагнітний стрижень маси $m = 0.1$~г і густини $\rho = 9.8$~г/см\textsuperscript{3}, що має магнітну сприйнятливість $\chi_m = - 14.5\cdot 10^{-6}$, підвішено за центр мас, який розташований на осі кругового струму на відстані $d$ (значно більшу за довжину стрижня) від площини витка (рис.~\ref{diamagnetic_in_field}). Радіус витка $R = d = 10$~см; сила струму $I = 100$~А. Як зорієнтується стрижень і яка результуюча сила нього діє?
\begin{solution}
	Діамагнетик орієнтується перпендикулярно осі витка. $F = 1.1\cdot 10^{-7}$~дин.
\end{solution}
\end{problem}

%=========================================================
\begin{figure}[h!]\centering
%---------------------------------------------------------
\begin{minipage}[t]{0.45\linewidth}\centering
	\begin{tikzpicture}
		\draw[thick, red, -latex] (0,0) + (0:4) arc(0:90:4 and 1) node[black, below right] {$I$};
		\draw[thick, red] (0,0) + (90:1) arc(90:361:4 and 1);
		\draw[-latex'] (0,0) -- node[below] {$R$} (170:3.3);
		\draw[dash dot] (0,0) -- node [left] {$z$} (0,3);
		\draw[ball color=gray!50] (0,3.3) circle (0.3) node {$\mu$};
		\path (0,0) -- +(0,-2);
	\end{tikzpicture}
	\caption{До задачі~\ref{prb:ballunderring_magnetic}}
	\label{ballunderring_magnetic}
\end{minipage}
%---------------------------------------------------------
\begin{minipage}[t]{0.45\linewidth}\centering
	\begin{tikzpicture}
		\draw[thick, red, -latex] (0,0) + (0:0.5) arc(0:90:0.5 and 2) node[black, below] {$I$};
		\draw[thick, red] (0,0) + (90:2) arc(90:361:0.5 and 2);
		\draw[-latex'] (0,0) -- node[right] {$R$} (136:0.5 and 2);
		\node[cylinder,draw=black,thick,aspect=0.3,minimum height=10mm,minimum     width=1mm, rotate=230,cylinder uses custom fill, cylinder body     fill=red!30,cylinder end fill=red!10] at (2,0) {};
		\draw[dash dot] (0,0) -- node [below] {$d$} (2,0);
		\draw (2,2) -- +(0,-2) coordinate (A);
	\end{tikzpicture}
\caption{До задачі~\ref{prb:diamagnetic_in_field}}
\label{diamagnetic_in_field}
\end{minipage}
%---------------------------------------------------------
\end{figure}
%=========================================================

%=========================================================
\begin{problem}\label{prb:paramagnetic_in_field}
    Розв'яжіть задачу~\ref{prb:diamagnetic_in_field}, якщо замість діамагнетика підвісили довгий тонкий алюмінієвий стрижень маси $m = 0.1$~г. Густина алюмінію $\rho = 2.7$~г/см\textsuperscript{3}, магнітна сприйнятливість магнітну сприйнятливість $\chi_m = 2.1\cdot 10^{-5}$.
\begin{solution}
	Алюмінієвий стрижень орієнтується вздовж осі витка.
\end{solution}
\end{problem}

%=========================================================
\begin{problem}
На відстані $l$ від нескінченного прямого проводу, по якому
йде постійний струм $I$, розташована кулька радіуса
$r$ ($r \ll l$), з магнітною проникністю $\mu$. Знайти силу, що діє
на кульку.
\begin{solution}
	$F = - \frac{4I^2r^3}{c^2l^3}\frac{\mu - 1}{\mu + 2}$.
\end{solution}
\end{problem}

%=========================================================
\begin{problem}
Довгий соленоїд, намотаний на тонкостінний капіляр, занурений одним кінцем в парамагнітну рідину на глибину, що значно перевищує його діаметр. Густина рідини $\rho$, магнітна проникність $\mu$. На скільки підніметься рівень рідини в капілярі, якщо по соленоїду пропустити струм $I$? Число витків на одиницю довжини соленоїда дорівнює $n$.
\begin{solution}
	$h = \frac{2\pi\mu(\mu - 1)n^2 I^2}{g\rho c^2}$.
\end{solution}
\end{problem}


%%=========================================================
%\begin{problem} %Матвеев, ст  344
%З якою силою один соленоїд втягує (виштовхує) інший. Густини намотки соленоїдів та сили струмів, що течуть по ним $n_1$, $I_1$ та $n_2$, $I_2$, відповідно. Вважати соленоїди близькими за діаметром і достатньо довгими.
%\begin{solution}
%	$F_x = \frac{4\pi}{c} n_1U_1n_2I_2 $.
%\end{solution}
%\end{problem}


%=========================================================
\begin{problem} %Матвеев, ст  344
В соленоїд, площа колового перерізу якого $S$, довжина $l$ і густина намотки $n$ внесено магнетик з проникністю $\mu$. Знайти силу, що діє на магнетик, якщо по соленоїду тече струм $I$.
\begin{solution}
	$F = \frac{2\pi}{c} (\mu - 1) n^2 I^2S$.
\end{solution}
\end{problem}

%=========================================================
\begin{problem}
    Знайдіть силу, що діє на провідник малого радіусу $R$, який лежить на поверхні магнетика з проникністю $\mu$. \emph{Вказівка}: скористайтесь відповідями до задачі~\ref{prb:wire_on_bound}.
\end{problem}

%=========================================================
\begin{problem}
    Оцінити порядок величини <<виштовхувальної>> сили, що діє на непровідне тіло об'ємом $V$, поміщене в провідну немагнітну рідину, в якій тече струм густиною $\vect{j}$ поперек магнітного поля з індукцією $\Bfield$. Які ефекти можуть привести до відмінності реальної виштовхувальної сили від сили, отриманої за такою оцінкою?
\begin{solution}
	$\vect{F} = \frac{V}{c} \left[ \vect{j}\times\Bfield\right] $.
\end{solution}
\end{problem}

%=========================================================
\begin{problem}
    Між полюсами магніту створено поле, індукція якого змінюється з відстанню від осі за законом  $B=B_0e^{-\frac{r^3}{r_0^3}}$, де $B_0$ та $r_0$~--- додатні постійні. Маленький діамагнітний зразок у вигляді тонкого диска з об'ємом $V$ і з магнітною сприйнятливістю $\chi_m$ вміщують по середині між полюсами на такій відстані від осі, щоб на нього діяла максимальна сила, що виштовхує зразок з міжполюсного простору. Знайти величину цієї сили.
\end{problem}

%=========================================================
\begin{problem}\label{prb:halves_of_torus}
    Котушка, що складається з $N$ витків, намотана на залізне тороїдальне осердя з магнітною проникністю $\mu$. Радіус тора $R$, радіус перерізу осердя $r$, причому $r \ll R$. Тор розрізаний на дві половини, що розсунуті так, щоб між ними утворився повітряний зазор ширини $d$, як показано на рис.~\ref{halves_of_torus}. Визначити величину сили взаємодії між половинами тора, якщо в обмотці протікає струм $I$, а $\mu \gg 1$.
\begin{solution}
	$F = \frac1{c^2}\left( \frac{\pi\mu I N r}{\pi R + \mu d}\right)^2$.
\end{solution}
\end{problem}

%---------------------------------------------------------
\begin{figure}[h!]\centering
		\newcommand{\midarrow}{\tikz \draw[-stealth] (0,0) -- +(0.05,0);}
		\begin{tikzpicture}
			\draw[-stealth] (0,0) -- (45:1.5) node[pos=0.5, above] {$R$};
			\draw[gray, fill=gray!50] (0:1.5) coordinate (C1) arc (0:180:1.5) coordinate (C2) to[bend left]
			(180:2) coordinate (C3) arc (180:0:2) coordinate (C4) to[bend left] cycle;

			\draw[yscale=-1, yshift=0.25cm, gray, fill=gray!50] (0:1.5) coordinate (B1)  arc (0:180:1.5)  coordinate (B2)  --
			(180:2) coordinate (B3) arc (180:0:2) coordinate (B4) -- cycle;

			%\draw[-latex] (0,0) -- node[left] {$R$} (240:1.5) ;

			\begin{scope}[yshift=-0.125cm]
				\draw[-latex] (15:2.2) arc (15:3:2.2);
				\draw[latex-] (-3:2.2) arc (-3:-15:2.2);
				\node at (0:2.4) {$d$};
			\end{scope}

			\foreach \i in {0,4,...,34} {
					\draw[red, rotate=-5*\i] (-1.5,0) arc(-25:210:0.28 and 0.15) node[pos=0.5, rotate=-5*\i] {\midarrow};
				}

			\foreach \i in {36,40,...,70} {
					\draw[yshift=-0.25cm, red, rotate=-5*\i] (-1.5,0) arc(-25:210:0.28 and 0.15) node[pos=0.5, rotate=-5*\i] {\midarrow};
}

			\draw [red] (C1) -- ([yshift=+0.01cm]B4);
			\draw [red] (C2) -- ([yshift=+0.01cm]B3);
			\draw [red] ([yshift=-0.01cm]C3) -- (B2);
			\draw [red] (C4) -- ([yshift=-0.01cm]B1);
			\draw[gray, fill=gray!50] (1.75,-0.25) ellipse (0.25 and 0.05);
			\draw[gray, fill=gray!50] (-1.75,-0.25) ellipse (0.25 and 0.05);
		\end{tikzpicture}
\caption{До задачі~\ref{prb:halves_of_torus}}
\label{halves_of_torus}
\end{figure}
%---------------------------------------------------------

%=========================================================
\begin{problem}
    Тороїдальне осердя виготовлені із зігнутого в кільце стрижня діаметром $1$~см і довжиною $1$~м. На нього рівномірно намотана обмотка з щільністю $100$ витків/см. Магнітна проникність стрижня $\mu \approx 1$.
\begin{enumerate}[label=\alph*)]
	\item Яке магнітне поле в осерді, якщо через обмотку тече струм силою 100 А?
	\item Знайдіть індуктивність обмотки (в генрі), припускаючи, що намотування є дуже щільною і її товщиною можна знехтувати;
	\item Розрахуйте електричну енергію, яка витрачається на створення магнітного поля в осерді при струмі силою
	100 А;
	\item Визначте енергію магнітного поля, зосередженого в осерді.
\end{enumerate}
Знехтувати радіальною залежністю поля, вважати його однорідним і рівним за величиною полю в центрі осердя.
\begin{solution}
\begin{enumerate}[label=\alph*)]
	\item $B = 1.257$~Тл.
	\item $L = 10^{-2}$~Гн;
	\item $W_e \approx 50$~Дж;
	\item  $W_m \approx 50$~Дж.
\end{enumerate}
\end{solution}
\end{problem}


\Closesolutionfile{answer}

