\protect \section *{\nameref *{Magnetostatics}}
\begin{Solution}{3.{3}}
	\begin{enumerate*}[label = \alph*)]
		\item $\frac{I}{c} \left( \frac{2\pi - \phi}{a}  + \frac{\phi}{b}\right) $;
		\item $\frac{I}{c} \left( \frac{3\pi}{a}  + \frac{\sqrt2}{b}\right) $.
	\end{enumerate*}
\end{Solution}
\begin{Solution}{3.{4}}
	$\Bfield = \frac{2\pi i }{c} \left[ \vect{\tau}\times\vect{n}\right] $, де $\vect{\tau}$~-- тангенціальний вектор, напрямлений в бік протікання струму, $\vect{n}$~-- вектор нормалі до поверхні.
\end{Solution}
\begin{Solution}{3.{5}}
			$\Bfield = - \frac{4\pi i }{c} \left[ \vect{\tau}\times\vect{n}\right] $, де $\vect{\tau}$~-- тангенціальний вектор, напрямлений в бік протікання струму, $\vect{n}$~-- вектор зовнішньої нормалі до поверхні соленоїда.
	
\end{Solution}
\begin{Solution}{3.{6}}
	$B = \frac{4I}{c\pi R}$.
\end{Solution}
\begin{Solution}{3.{7}}
	$\Bfield = \left[ \frac{\vect{v}}{c}\times\Efield\right]  $, де $\Efield$~--- напруженість електричного поля диполя.
\end{Solution}
\begin{Solution}{3.{8}}
	$B = \frac{2\pi I}{cp}$, де $p = \frac{b^2}{a}$~-- фокальний параметр еліпса.
\end{Solution}
\begin{Solution}{3.{9}}
	$B_z = \frac{2\pi}{c}\frac{I R^2}{(R^2 + z^2)^{3/2}}$,
	\begin{enumerate*}[label=\alph*)]
		\item $B_z = \frac{2\pi}{c} In (\cos\alpha_1 - \cos\alpha_2)$,
		\item $B_z(0) = \frac{2\pi}{c} In$,
		\item $B_z(0) = \frac{4\pi}{c} In$.
	\end{enumerate*}
\end{Solution}
\begin{Solution}{3.{10}}
	$B_z(z) = \frac{2q\omega}{cR^2}\left(\frac{2z^2 + R^2}{\sqrt{R^2 + z^2}} - 2z\right)$.
\end{Solution}
\begin{Solution}{3.{11}}
	$B(r) = %
		\begin{cases}
			\frac{2Ir}{cR_1^2}, & \quad r < R_1           \\
			\frac{2I}{cr},      & \quad R_1 \le r \le R_2 \\
			0,                  & \quad r \ge R_2
		\end{cases}
	$.
\end{Solution}
\begin{Solution}{3.{12}}
	$B(r) = %
		\begin{cases}
			\frac{2I\rho_1r}{c(\rho_1(R_2^2 - R_1^2) + \rho_2R_1^2)}, & \quad r < R_1           \\
			\frac{2I(\rho_1(r^2 - R_1^2) + \rho_2R_1^2)}{cr(\rho_1(R_2^2 - R_1^2) + \rho_2R_1^2)},      & \quad R_1 \le r \le R_2 \\
			\frac{2I}{cr},                  & \quad r \ge R_2.
		\end{cases}
	$
\end{Solution}
\begin{Solution}{3.{13}}
	$B = \frac{2\pi}{c} jd$.
\end{Solution}
\begin{Solution}{3.{14}}
	$B(r) = \frac{1}{c}\frac{Id}{rR}$.
\end{Solution}
\begin{Solution}{3.{15}}
	$\Bfield = %
		\begin{cases}
			\frac{4\pi}{c} a \omega r^2 \vect{e}_r, & \quad r < R \\
			0,                                      & \quad r > R
		\end{cases}
	$.
\end{Solution}
\begin{Solution}{3.{21}}
		$p = \frac1c ib \pi R^2$.
	
\end{Solution}
\begin{Solution}{3.{22}}
	$p = \frac{IND^2}{2c}$.
\end{Solution}
\begin{Solution}{3.{23}}
	$B = \frac{q\omega}{2c}\frac{R^2}{z^3}$, $B = \frac{2p_m}{z^3}$.
\end{Solution}
\begin{Solution}{3.{24}}
	Для сфери $\vect{p}_m = \frac{qR^2}{3c}\vect{\omega}$,
	для кулі $\vect{p}_m = \frac{qR^2}{5c}\vect{\omega}$.
	В обох випадках $\frac{p_m}{L} = \frac{q}{2mc}$.
\end{Solution}
\begin{Solution}{3.{29}}
	$\vect{A} =-  \frac{2I}{c}\ln r \vect{k}$, $\vect{k}$~--- орт, напрямлений вздовж провідника зі струмом.
	% Вказівка, скористайтесь теоремою Стокса і обчисліть потік вектора \Bfield
\end{Solution}
\begin{Solution}{3.{30}}
	$\vect{A} = \frac12 \left[ \Bfield\times \vect{r} \right] $.
\end{Solution}
\begin{Solution}{3.{31}}
	$\vect{A} = \frac{\vect{p}_m\times\vect{r}}{r^3}$, де $\vect{p}_m = \frac{I\pi R^2}{c}\vect{n}$~-- дипольний момент колового витка, $\vect{n}$~-- вектор нормалі до поверхні витка.
\end{Solution}
\begin{Solution}{3.{32}}
	З рівняння~\eqref{rotvect} випливає, що $\rot\vect{A} = \rot\left( \frac{\vect{p}_m\times\vect{r}}{r^3}\right)  = -(\vect{p}_m\cdot\vect{\nabla}) \frac{\vect{r}}{r^3}$.
	Розписуючи останній вираз  в декартовій системі координат, отримуємо шукану формулу.
\end{Solution}
\begin{Solution}{3.{33}}
	При калібровці $\vect{A}(0) = 0$,

	$
		\vect{A}(r) = %
		\begin{cases}
			-\frac{1}{c}\pi r^2\vect{j},                                          & \quad r \le R \\
			-\frac{1}{c}2\pi R^2\left( \frac12 + \ln\frac{r}{R}\right) \vect{j} , & \quad r \ge R
		\end{cases}
	$.
\end{Solution}
\begin{Solution}{3.{34}}
	$\Delta\Bfield = -\frac{e^2 \Bfield_0}{3amc^2}$.
\end{Solution}
\begin{Solution}{3.{35}}
	$\vect{A} =
		\begin{cases}
			\frac{4\pi}{3}R\omega\sigma r\sin\theta \vect{e}_{\phi} = \frac{\vect{p}_m\times\vect{r}}{R^3},              & r \le R \\
			\frac{4\pi}{3}R^4\omega\sigma \frac{\sin\theta}{r^2} \vect{e}_{\phi} = \frac{\vect{p}_m\times\vect{r}}{r^3}. & r \ge R \\
		\end{cases}$,

	$\Bfield =
		\begin{cases}
			\frac{8\pi}{3c}R\sigma\vect{\omega} = \frac{2\vect{p}_m}{R^3},     & r \le R \\
			\frac{3(\vect{p}_m\vect{r})\vect{r}}{r^5} -\frac{\vect{p}_m}{r^3}. & r \ge R \\
		\end{cases}$.
\end{Solution}
\begin{Solution}{3.{36}}
	$\vect{A} = - \frac{2\pi jRd}{c}\ln R$,  $\vect{k}$~--- орт, напрямлений вздовж провідника зі струмом.
\end{Solution}
\begin{Solution}{3.{38}}
	$\Bfield = 4\pi \vect{M}$, $\Hfield = 0$, $\vect{j'} = 0$, $\vect{i}' = \vect{M}\times \vect{n}$.
\end{Solution}
\begin{Solution}{3.{39}}
	$\vect{M} = \frac{\mu - 1}{4\pi} \Bfield_0.$
\end{Solution}
\begin{Solution}{3.{40}}
	$\vect{M} = \frac{\mu - 1}{4\pi\mu} \Bfield_0.$
\end{Solution}
\begin{Solution}{3.{41}}
	\begin{enumerate*}[label=\alph*)]
		\item $\oint\limits_{S} \Hfield d\vect{S} = \frac{\mu  -1}{\mu} \pi R^2 B_0 \cos\theta$,
		\item $\oint\limits_{\Gamma} \Bfield d\vect{r} = - \left( \mu -1\right)  l B_0 \sin\theta$.
	\end{enumerate*}
	
\end{Solution}
\begin{Solution}{3.{42}}
	$\Bfield = %
		\begin{cases}
			4\pi\vect{M}, & \quad r < R \\
			0,            & \quad r > R
		\end{cases}
	$,
	$\Hfield = 0$ в усьому просторі.
\end{Solution}
\begin{Solution}{3.{43}}
	При $r \le R$, $\Bfield = \frac{8\pi}{3} \vect{M}$, $\Hfield = -\frac{4\pi}{3} \vect{M}$.
\end{Solution}
\begin{Solution}{3.{44}}
	$\Hfield_\text{осердя} \approx - 4\pi \vect{M}\frac{d}{2\pi R}$,  $\Hfield_\text{зазор} \approx 4\pi \vect{M} \left( 1 - \frac{d}{2\pi R}\right) $, $\Bfield_\text{осердя} = \Bfield_\text{зазор} = \Hfield_\text{зазор}$.
\end{Solution}
\begin{Solution}{3.{45}}
	\begin{enumerate*}[label=\alph*)]
		\item $I'_\text{пов} = 4\pi\chi I$;
		\item $I'_\text{об} = 4\pi\chi I$.
	\end{enumerate*}
	Струми напрямлені протилежно один відносно одного.
\end{Solution}
\begin{Solution}{3.{46}}
	\begin{enumerate*}[label=\alph*)]
		\item $\vect{M} = ar^2 \Bfield_0$;
		\item $\vect{j}' = 2сa\left[ \vect{r} \times \Bfield_0\right] $.
	\end{enumerate*}
Поверхневий струм $i' = -caR \left[ \vect{R} \times \Bfield_0\right]$, де $\vect{R}$~--- радіус-вектор точок на поверхні. Поверхневий та об'ємний струми течуть в різних напрямках.
Повний молекулярний струм на одиницю довжини дорівнює нулю.
\end{Solution}
\begin{Solution}{3.{47}}
	$\Hfield = \frac{4\pi}{c} nI \vect{e}_z$, $\Bfield = \frac{4\pi}{c} (1+\chi)nI \vect{e}_z$, $\vect{i}' = с\chi nI\vect{e}_{\phi}$.
\end{Solution}
\begin{Solution}{3.{48}}
	$
		H(r) = %
		\begin{cases}
			\frac{2I}{cR^2}r, & \quad r < R   \\
			\frac{2I}{cr},    & \quad r \ge R
		\end{cases}
	$,
	$
		B(r) = %
		\begin{cases}
			\mu_1\frac{2I}{c R^2}r, & \quad r < R   \\
			\mu_2\frac{2I}{c r},    & \quad r \ge R
		\end{cases}
	$,\\
	$
		M(r) = %
		\begin{cases}
			(\mu_1-1)\frac{2I}{c R^2}r, & \quad r < R   \\
			(\mu_2-1)\frac{2I}{c r},    & \quad r \ge R
		\end{cases}
	$,
	$
		j'(r) = %
		\begin{cases}
			(\mu_1-1)\frac{4I}{c R^2}, & \quad r < R   \\
			0,                         & \quad r \ge R
		\end{cases}
	$,

	$i' = \frac{2I}{R} (\mu_2 - \mu_1)$, при $r = R$.
\end{Solution}
\begin{Solution}{3.{49}}
	$M(r) = \frac{\mu - 1}{\mu + 1}\frac{4I}{r}$,
	$B(r) = \frac{\mu}{\mu + 1}\frac{4I}{cr}$,
	$H_1(r) = \frac{\mu}{\mu + 1}\frac{4I}{cr}$,
	$H_2(r) = \frac{1}{\mu + 1}\frac{4I}{cr}$,

	$I' = I \frac{\mu - 1}{\mu + 1}$.
\end{Solution}
\begin{Solution}{3.{50}}
	$B_r = (1 + 4\pi\chi)\frac{2I}{cr}$, $I'  = 4\pi\chi I$, на зовнішній та внутрішній поверхнях магнетика напрямок струмів намагнічування співпадає з напрямком вільних струмів.
\end{Solution}
\begin{Solution}{3.{51}}
	$I' = 0$.
\end{Solution}
\begin{Solution}{3.{53}}
	$\Hfield =
		\begin{cases}
			\frac{2Ir}{cR^2}\vect{e}_{\phi}, & r \le R \\
			\frac{2I}{cR}\vect{e}_{\phi},    & r > R
		\end{cases}$,

	$\Bfield =
		\begin{cases}
			\frac{2Ir}{cR^2}\vect{e}_{\phi},   & r \le R                                  \\
			\frac{2I\mu_1}{cR}\vect{e}_{\phi}, & r > R  \, (\text{в середовищі з } \mu_1) \\
			\frac{2I\mu_2}{cR}\vect{e}_{\phi}, & r > R  \, (\text{в середовищі з } \mu_2)
		\end{cases}$.
\end{Solution}
\begin{Solution}{3.{54}}
	$\vect{p}_m = \frac{\mu_i - \mu_e}{\mu_i + 2\mu_e}R^3\Bfield_0$,
	$\Bfield =
		\begin{cases}
			\frac{3\mu_i}{\mu_i + 2\mu_e} \Bfield_0,                                                    & r \le R, \\
			\Bfield_0 - \frac{\vect{p}_m}{r^3} + \frac{3\left(\vect{p}_m\vect{r}\right)\vect{r} }{r^5}, & r > R.
		\end{cases}	$

	Густина об'ємних струмів намагнічування $\vect{j}' = 0$, поверхнева густина струмів намагнічування $i = \frac{3c}{4\pi} \frac{\mu_i - \mu_e}{\mu_i + 2\mu_e} \frac{\Bfield_0\vect{r}}{R}$, де $\vect{r}$~-- радіус-вектор поверхні провідника.
\end{Solution}
\begin{Solution}{3.{55}}
	$\Bfield = \frac{3\mu}{1 + 2\mu}\Bfield_0$.
\end{Solution}
\begin{Solution}{3.{56}}
	$B = \frac{4\pi M_r}{1 + \frac{2d M_r }{RH_c}}$.
\end{Solution}
\begin{Solution}{3.{58}}
	$I_0 = \frac{cl}{4\pi N} \left(H_0 + 4\pi M_0 \frac{d}{l}\right)$, $B = \frac{4\pi N}{cl} I + 4\pi M_0 \left(1 - \frac{d}{l}\right)$.
\end{Solution}
\begin{Solution}{3.{59}}
	На рисунку показана робоча точка залізного осердя ($H \approx 5$~Е, $B~\approx~15$~кГс). $\mu = 3000$.
	\begin{center}
		\begin{tikzpicture}[scale=0.7]
			\begin{axis}[   axis y line = left,
					axis x line = bottom,
					grid = both,
					ylabel={$B$, кГс},
					xlabel={$H$, Е},
					xtick = {0,5,...,25},
					ytick = {0,5,...,25},
				]

				\addplot[thick, red, domain={0:25}, samples=100, name path = curve] {25*sqrt(1-((x-25)^2/25^2))};
				\addplot[blue, domain={0:20}, name path = line] {20 - x};
				\path [name intersections={of=line and curve, by=P}];
%				\fill[red] (P) circle (0.05cm);
			\end{axis}
		\end{tikzpicture}
	\end{center}
\end{Solution}
\begin{Solution}{3.{60}}
	$p_m = \frac{\mu - 1}{\mu} B a^2 d$, $B = \frac{p_m}{r^3}$.
\end{Solution}
\begin{Solution}{3.{61}}
	$\frac{F}{l} = \frac{2\pi I^2}{c^2r}$,
	$\frac{W}{l} = \frac{2\pi I^2}{c^2}\ln \left( \frac{r}{r_0}\right) $, де $r_0$~--- значення відстані, прийняте за нульовий рівень енергії.
\end{Solution}
\begin{Solution}{3.{62}}
	$\frac{F}{l} = \frac{2I^2}{c^2b}\ln\left( \frac{a + b}{a}\right) $.
\end{Solution}
\begin{Solution}{3.{63}}
	$p = \frac{I^2}{2\pi c^2R^2}$.
\end{Solution}
\begin{Solution}{3.{64}}
	$p = \frac{I^2(R^2 - r^2)}{R^4\pi c^2}$.
\end{Solution}
\begin{Solution}{3.{65}}
	Узагальненою силою є момент сил, що діє з боку магнітного поля $N = -\frac1c ISB\sin\theta$.
\end{Solution}
\begin{Solution}{3.{67}}
	$p = \frac{2\pi i^2}{c^2}$.
\end{Solution}
\begin{Solution}{3.{68}}
	$N  = 2I  M_0 S$.
\end{Solution}
\begin{Solution}{3.{69}}
	$F_\parallel/l  = I\cdot  B_\bot = I\cdot B\cos\phi$.
\end{Solution}
\begin{Solution}{3.{70}}
	$B = \frac{c\pi d^2 \sigma_{\max}}{4RI}$.
\end{Solution}
\begin{Solution}{3.{71}}
	$M = \frac{4\pi p_m NI}{cl}$.
\end{Solution}
\begin{Solution}{3.{72}}
	\begin{enumerate*}[label=\alph*)]
		\item $\vect{F} = 0$, $\vect{M} = - \frac{2I}{cr^2} \vect{r} (\vect{p} \cdot \vect{k})$,
		\item $\vect{F} = - \frac{2I}{cr^2} \left[ \vect{k} \times \vect{p}\right]$, $\vect{M} = 0$,
		\item $\vect{F} = \frac{2I}{cr^2} \left[ \vect{k} \times \vect{p}\right]$, $\vect{M} = 0$,
	\end{enumerate*}
	де $\vect{k}$~--- орт, напрямлений вздовж струму.
\end{Solution}
\begin{Solution}{3.{74}}
	$\vect{F} = -\frac{6\pi}{c}\frac{IR^2z}{(R^2 + z^2)^{5/2}}\vect{p}_m$.
\end{Solution}
\begin{Solution}{3.{75}}
	$\vect F = - \left( \frac{2\pi IR^2}{c}\right)^2 \frac{3z}{(z^2 + R^2)^4} \vect k.$
\end{Solution}
\begin{Solution}{3.{76}}
	Діамагнетик орієнтується перпендикулярно осі витка. $F = 1.1\cdot 10^{-7}$~дин.
\end{Solution}
\begin{Solution}{3.{77}}
	Алюмінієвий стрижень орієнтується вздовж осі витка.
\end{Solution}
\begin{Solution}{3.{78}}
	$F = - \frac{4I^2r^3}{c^2l^3}\frac{\mu - 1}{\mu + 2}$.
\end{Solution}
\begin{Solution}{3.{79}}
	$h = \frac{2\pi\mu(\mu - 1)n^2 I^2}{g\rho c^2}$.
\end{Solution}
\begin{Solution}{3.{80}}
	$F = \frac{2\pi}{c} (\mu - 1) n^2 I^2S$.
\end{Solution}
\begin{Solution}{3.{82}}
	$\vect{F} = \frac{V}{c} \left[ \vect{j}\times\Bfield\right] $.
\end{Solution}
\begin{Solution}{3.{84}}
	$F = \frac1{c^2}\left( \frac{\pi\mu I N r}{\pi R + \mu d}\right)^2$.
\end{Solution}
\begin{Solution}{3.{85}}
\begin{enumerate}[label=\alph*)]
	\item $B = 1.257$~Тл.
	\item $L = 10^{-2}$~Гн;
	\item $W_e \approx 50$~Дж;
	\item  $W_m \approx 50$~Дж.
\end{enumerate}
\end{Solution}
