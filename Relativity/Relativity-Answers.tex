\protect \section *{\nameref *{Relativity}}
\begin{Solution}{6.{1}}
	$\Efield = \frac{q^2}{r^3}\vect{r} \frac{1 - \frac{v^2}{c^2}}{ \left( 1 - \frac{v^2}{c^2}\sin^\theta \right)^{\nfrac32}}$, де $\theta$~-- кут між вектором швидкосі $\vect{v}$ і радіус-вектором $\vect{r}$, $\Bfield = \frac1c \frac{\left[ \vect{v}\times\vect{r}\right] }{r^3}$.
\end{Solution}
\begin{Solution}{6.{7}}
	Сила в лабораторній системі менша в $\Gamma = \frac{1}{\sqrt{1- \frac{v^2}{c^2}}}$ разів.
\end{Solution}
\begin{Solution}{6.{8}}
	$\sigma  = \frac{e^2}{2\pi a^2}\sqrt{1 - \frac{v^2}{c^2}}$.
\end{Solution}
\begin{Solution}{6.{9}}
	$\rho' = \frac{\rho_0}{\sqrt{1 - \frac{V^2}{c^2}}}$, $j' = \rho' V$.
\end{Solution}
\begin{Solution}{6.{10}}
	$\rho' = \frac{- v j}{\sqrt{1 - \frac{V^2}{c^2}}}$, $j' = \frac{j}{\sqrt{1 - \frac{V^2}{c^2}}}$.
\end{Solution}
\begin{Solution}{6.{11}}
	У власній системі відліку частинки $\phi'  = \frac{q}{\sqrt{x'^2 + y'^2 + z'^2}}$, $\vect{A}' = 0$, де $x'$, $y'$, $z'$~-- декартові координати власної системи відліку.
	В лабораторній системі відліку $\phi = \frac{q}{\sqrt{(x - Vt)^2 + \left( 1 - V^2/c^2\right)(y^2 + z^2) }}$, $\vect{A} = \phi\frac{\vect{V}}{c}$.
\end{Solution}
