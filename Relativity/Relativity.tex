% !TeX program = lualatex
% !TeX encoding = utf8
% !TeX spellcheck = uk_UA
% !TeX root =../EMProblems.tex

%=========================================================
\Opensolutionfile{answer}[\currfilebase/\currfilebase-Answers]
\Writetofile{answer}{\protect\section*{\nameref*{\currfilebase}}}
\chapter{Релятивіствька електродинаміка}\label{\currfilebase}
%=========================================================
%http://www.rec.vsu.ru/rus/ecourse/eldin/izluch.pdf

\section{Перетворення електричних та магнітних полів}
\begin{Theory}
Закони лоренцівські перетворення векторів $\Efield$ та $\Bfield$:

\begin{align*}
	E_x' =  E_x, \quad                                        & B_x' = B_x \\
	E_y' = \left( E_y - \dfrac Vc B_z\right) \Gamma, \quad &
	B_y' = \left( B_y + \dfrac Vc E_z\right) \Gamma                     \\
	E_z' = \left( E_z + \dfrac Vc B_y\right) \Gamma, \quad &
	B_z' = \left( B_z - \dfrac Vc E_y\right) \Gamma,                    \\
\end{align*}
де $\Gamma = \frac{1}{\sqrt{1 - \frac{V^2}{c^2}}}$. 
\end{Theory}

%=========================================================
\begin{problem}
    Заряджена частинка $q$ рухається зі швидкістю $v \approx c$ в лабораторній системі відліку. Знайдіть величину електричного і магнітного поля частинки в цій системі відліку в довільній точці простору.
\begin{solution}
	$\Efield = \frac{q^2}{r^3}\vect{r} \frac{1 - \frac{v^2}{c^2}}{ \left( 1 - \frac{v^2}{c^2}\sin^\theta \right)^{\nfrac32}}$, де $\theta$~-- кут між вектором швидкосі $\vect{v}$ і радіус-вектором $\vect{r}$, $\Bfield = \frac1c \frac{\left[ \vect{v}\times\vect{r}\right] }{r^3}$.
\end{solution}
\end{problem}

%=========================================================
\begin{problem}
    Як пояснюється виникнення ЕРС індукції в провіднику, що рухається в магнітному полі, в системі відліку, що  пов'язана з провідником?
\end{problem}

%=========================================================
\begin{problem}
    Який критерій того, що може існувати така система відліку, в якій електромагнітне поле буде або як суто магнітне, або як суто електричне?
\end{problem}

%=========================================================
\begin{problem}
    Якщо в деякій системі відліку поле лише суто електричне або лише суто магнітне, то яку загальну властивість буде мати в цьому випадку електромагнітне поле у всіх інших системах відліку?
\end{problem}

%=========================================================
\begin{problem}
    Заряд, що пролітає повз магнітну стрілку буде впливати на неї, оскільки навколо рухомого заряду існує магнітне поле. Як пояснити явище впливу заряду на стрілку відносно системи відліку, в якій заряд знаходиться в спокої?
\end{problem}

%=========================================================
\begin{problem}
    Доведіть, що рухомий коловий струм має дипольний електричний момент. Який зв'язок між магнітним моментом струму і його дипольним електричним моментом?
\end{problem}

%=========================================================
\begin{problem}
    Два електрона з однаковими швидкостями $v \approx c$ рухаються паралельно один одному на відстані $a$ один від одного. В скільки разів відрізняється сила взаємодії між ними в системі відліку, що пов'язана з ними від сили в лабораторній системі відліку.
\begin{solution}
	Сила в лабораторній системі менша в $\Gamma = \frac{1}{\sqrt{1- \frac{v^2}{c^2}}}$ разів.
\end{solution}
\end{problem}

%=========================================================
\begin{problem}
    Два електрона з однаковими швидкостями $v \approx c$ рухаються паралельно один одному по обидва боки від нескінченної позитивно зарядженої площини на відстані $a$ від неї. Заряди розподілені по поверхні площини рівномірно з деякою густиною. При якій густині відстань між електронами залишатиметься постійною і рівною $2a$?
\begin{solution}
	$\sigma  = \frac{e^2}{2\pi a^2}\sqrt{1 - \frac{v^2}{c^2}}$.
\end{solution}
\end{problem}

%=========================================================
\begin{problem}
    Нескінченно заряджена нитка має густину заряду $\rho$. Знайти густину зарядів та густину струму нитки відносно системи відліку, що рухається вздовж нитки з швидкістю $V$. 
\begin{solution}
	$\rho' = \frac{\rho_0}{\sqrt{1 - \frac{V^2}{c^2}}}$, $j' = \rho' V$.
\end{solution}
\end{problem}

%=========================================================
\begin{problem}
    По нескінченно довгому провіднику тече струм густиною $j$. Знайти густину зарядів та густину струму нитки відносно системи відліку, що рухається вздовж нитки з швидкістю $V$ ($\vect{j} \uparrow\uparrow \vect{v}$). 
\begin{solution}
	$\rho' = \frac{- v j}{\sqrt{1 - \frac{V^2}{c^2}}}$, $j' = \frac{j}{\sqrt{1 - \frac{V^2}{c^2}}}$.
\end{solution}
\end{problem}

%=========================================================
\begin{problem}
    Знайти скалярний та векторний потенціал зарядженої частинки, що рухається в лабораторній системі відліку з швидкістю $v \approx c$. 
\begin{solution}
	У власній системі відліку частинки $\phi'  = \frac{q}{\sqrt{x'^2 + y'^2 + z'^2}}$, $\vect{A}' = 0$, де $x'$, $y'$, $z'$~-- декартові координати власної системи відліку.
	В лабораторній системі відліку $\phi = \frac{q}{\sqrt{(x - Vt)^2 + \left( 1 - V^2/c^2\right)(y^2 + z^2) }}$, $\vect{A} = \phi\frac{\vect{V}}{c}$.
\end{solution}
\end{problem}

\Closesolutionfile{answer}

