\protect \section *{\protect \nameref  *{Electrodynamics}}
\begin{Solution}{4.{1}}
	$v = c\frac{mgR}{(Bl)^2}$.
\end{Solution}
\begin{Solution}{4.{2}}
	$v = c \frac{\EMF + \frac{mg}{Bl}r}{\left( 1 +  \frac{r}{R}\right)Bl}$.
\end{Solution}
\begin{Solution}{4.{3}}
	$v = \frac32 \frac{mgRc^2}{B^2l^2}$.
\end{Solution}
\begin{Solution}{4.{4}}
	$I = \frac{\omega B r^2}{2cR}$.
\end{Solution}
\begin{Solution}{4.{5}}
	$V_1  = \frac{B\omega C_2 \left| R_1^2 \pm R_2^2 \right| }{2(C_1 + C_2)}$, $V_2  = \frac{B\omega C_1 \left| R_1^2 \pm R_2^2 \right| }{2(C_1 + C_2)}$. Знак <<$+$>> стосується випадку обертання дисків
	в різних напрямках, <<$-$>>~--- до обертання дисків в одному напрямку.
\end{Solution}
\begin{Solution}{4.{6}}
	\begin{enumerate*}[label=\alph*)]
		\item $V = \frac{B\omega R^2}{2c}$;
		\item $\sigma \frac{\omega B R}{4\pi}$, $\rho = -\frac{\omega B}{2\pi}$.
	\end{enumerate*}
\end{Solution}
\begin{Solution}{4.{7}}
		$\rho' = -\frac{\epsilon - 1}{2\pi c} (\vect\omega\times\Bfield)$, $\sigma' = \frac{\epsilon - 1}{2\pi c} (\vect\omega\times\Bfield)R$.
\end{Solution}
\begin{Solution}{4.{8}}
	$V = \frac{Bl^2\omega}{8c}$.
\end{Solution}
\begin{Solution}{4.{9}}
	\begin{enumerate*}[label=\alph*)]
		\item $\EMF = \frac{2I_ol}{c(a+vt)}$,
		\item $I = \frac{I_0 l v}{cr(a+vt)(l+vt)}$.
	\end{enumerate*}
\end{Solution}
\begin{Solution}{4.{10}}
	$\EMF = \frac{2Ia^2v}{c(b+vt)(a+ b  +vt)}$, $L_{12} = 2a\ln \left( 1+ \frac{a}{s}\right) $.
\end{Solution}
\begin{Solution}{4.{11}}
	$x = \frac{v_0}{\omega}\sin\omega t$,
	$I = -v_0\sqrt{\frac{m}{L}}\sin\omega t$,
	де $\omega = \frac{lB}{\sqrt{mL}}$.
\end{Solution}
\begin{Solution}{4.{12}}
	$\omega = \frac{qB(t)}{2mc}$.
\end{Solution}
\begin{Solution}{4.{13}}
	$q_1 = - \frac1c \dot{\Phi}\frac{C_1(C_2 + C_3/2)}{C_1 + C_2 +C_3}$, $q_2 =  \frac1c \dot{\Phi}\frac{C_2(C_1 + C_3/2)}{C_1 + C_2 +C_3}$, $q_3 =  \frac1c \dot{\Phi}\frac{C_3/2(C_2 - C_1)}{C_1 + C_2 +C_3}$.
\end{Solution}
\begin{Solution}{4.{14}}
	$V_1 = \alpha \frac{R_1}{R_1 + R_2}$, $V_2 = - \alpha \frac{R_2}{R_1 + R_2}$.
\end{Solution}
\begin{Solution}{4.{15}}
	$L = \frac{4\pi\nu N^2 S}{l}$.
\end{Solution}
\begin{Solution}{4.{16}}
	$L  = 2bM\mu\ln\left( 1 + \frac{a}{R}\right) $.
\end{Solution}
\begin{Solution}{4.{17}}
	$L_{12} = 2bN\mu \ln\left( 1 + \frac{a}{R}\right) $.
\end{Solution}
\begin{Solution}{4.{18}}
	$L/l = \frac{\mu}{2} \left( 1 + 4\ln\frac{R_2}{R_1}\right)$.
\end{Solution}
\begin{Solution}{4.{19}}
	$L = 4\pi\frac{a}{b}$.
\end{Solution}
\begin{Solution}{4.{20}}
	$L = \frac{1}{4\pi^2}\ln\frac{a}{R}$.
\end{Solution}
\begin{Solution}{4.{21}}
	$L_{12} = 2a\ln\left( 1 + \frac{a}{l}\right).$
\end{Solution}
\begin{Solution}{4.{22}}
	$L_{12} \approx \frac{2\pi^2R^4}{l^3}$.
\end{Solution}
\begin{Solution}{4.{23}}
	$M = 8\pi a$.
\end{Solution}
\begin{Solution}{4.{24}}
		\begin{enumerate*}[label=\alph*)]
		\item $L_{12} \approx \frac{2\pi^2R_1^2}{R_2}$
		\item $\Phi_{12} = \frac{2\pi^2}{c} \frac{R_1^2}{R_2} I$.
		\end{enumerate*}
\end{Solution}
\begin{Solution}{4.{25}}
	$\Phi = \frac{2Ia}{c}\ln\left( 1 + \frac{a}{b}\right) $.
\end{Solution}
\begin{Solution}{4.{26}}
	$I = \frac{\pi^2 r^2 G \omega}{\sqrt{2}c^2R}i_0\sin\omega t$.
\end{Solution}
\begin{Solution}{4.{27}}
	$I = \frac{2p_m\pi a^2\omega}{cRl^3}\sin\omega t$.
\end{Solution}
\begin{Solution}{4.{28}}
	В SI $p_m = \frac{\phi R D}{4\pi b n}$.
\end{Solution}
\begin{Solution}{4.{29}}
	$\vect{p}_m = - \frac12 R^3\Bfield_0$,
	$
		\Bfield =
		\begin{cases}
			0,                                                                                                & r \le R \\
			\left( 1 + \frac{R^3}{2r^3}\right)\Bfield_0 - \frac{3R^3(\Bfield_0\cdot \vect{r})\vect{r}}{2r^5}, & r > R,
		\end{cases}.
	$

	Густина об'ємних струмів намагнічування $\vect{j}' = 0$.

	Поверхнева густина струмів намагнічування
	$i = -\frac{3c}{8\pi} \frac{\Bfield_0\vect{r}}{R}$,
	де $\vect{r}$~-- радіус-вектор поверхні провідника.
\end{Solution}
\begin{Solution}{4.{30}}
	$F  = \frac{9}{64}B_0^2R^2$.
\end{Solution}
\begin{Solution}{4.{31}}
	Магнітний потік, що пронизує надпровідне кільце зберігається $\Phi = \const$. Оскільки, в спочатку диполь не було внесено, то $\Phi = 0$, і залишатиметься таким же. Коли магнітний диполь опиниться в центрі кільця, то магнітний потік, що пронизує кільце, можна порахувати як:
	\[
		\Phi = \frac1c LI + \frac1c L_{21}I_m = 0,
	\]
	де $L$~-- індуктивність кільця, $I$~-- струм, що тече по кільцю, $L_{21}$~-- коефіцієнт взаємоіндукції, $I_m$~-- умовний струм, що циркулює в диполі $p_m$ ($I_m = \frac{cp_m}{S}$, $S$~-- умовна площа витка диполя).

	З закону збереження магнітного потоку випливає, що $I = - \frac{L_{12}}{L}\frac{cp_m}{S}$.
	Для знаходження $L_{21}$, скористаємось теоремою взаємності, $L_{21} = L_{12} = \frac{2\pi S}{R}$.
	Отже,
	\[
		I = - \frac{ 2\pi c p_m}{RL}.
	\]
	Знак мінус вказує на те, що індукційний що магнітний момент струму протилежний магнітного моменту диполя.
\end{Solution}
\begin{Solution}{4.{32}}
	$I = \frac{c\Phi_0}{L}$.
\end{Solution}
\begin{Solution}{4.{33}}
	$I = \frac{cB\pi R^2}{L}$, $A = \frac{\Phi^2}{2L} = \frac{B^2\pi^2 R^4}{2L}$.
\end{Solution}
\begin{Solution}{4.{34}}
	$h = \frac12 \sqrt[4]{\frac{3p_m^2}{mg}}$.
\end{Solution}
\begin{Solution}{4.{35}}
	$I \ge c \sqrt{\frac{mgh}{2\pi R}} = 8.4\cdot 10^{10}$~Фр/с = $25$~А.
\end{Solution}
\begin{Solution}{4.{36}}
	$v_0 = B_0 \sqrt{\frac{R^3}{2m}} \approx 20$~м/c.
\end{Solution}
\begin{Solution}{4.{37}}
	$F = \left( \frac{2\pi I R}{c} \right)^2 \frac{3r^3 z}{2(R^2 + z^2)^4} $.
\end{Solution}
\begin{Solution}{4.{38}}
	$ p = \frac{9}{32\pi} \Bfield_0\cdot\vect{r}$, де $\vect{r}$~-- радіус-вектор поверхні сфери.
\end{Solution}
\begin{Solution}{4.{40}}
	$\vect\omega = -\frac{q\Bfield(t)}{2mc}$.
\end{Solution}
\begin{Solution}{4.{41}}
	$\vect\omega = \frac{2\sigma\Bfield}{c\rho R}$.
\end{Solution}
\begin{Solution}{4.{42}}
	$E(r) = \frac{2\pi\epsilon}{c^2}\frac{nI_0R^2\omega}{r}\sin\omega t$.
\end{Solution}
\begin{Solution}{4.{43}}
	$
		\Efield =
		\begin{cases}
			- \frac{2\pi}{c^2} n\dot{I} r\vect{e}_{\phi},               & r < R, \\
			- \frac{2\pi}{c^2} n a^2\dot{I} \frac{1}{r}\vect{e}_{\phi}, & r > R.
		\end{cases}
	$
\end{Solution}
\begin{Solution}{4.{44}}
	$
		\Bfield =
		\begin{cases}
			\frac{2\pi\rho\omega}{c}(R^2 - r^2)\vect{e}_z, & r< R,    \\
			0,                                             & r \ge R,
		\end{cases}
	$

	$
		\Efield =
		\begin{cases}
			\frac{\pi k \rho r\omega}{2c^2}(2R^2 - r^2)\vect{e}_{\phi} + 2\pi\rho r \vect{e}_r, & r< R,    \\
			\frac{\pi k \rho r R^4}{2c^2r}\vect{e}_{\phi} + \frac{2\pi \rho R^2}{r}\vect{e}_r,  & r \ge R.
		\end{cases}
	$
\end{Solution}
\begin{Solution}{4.{47}}
	$B = \frac{2I}{cR^2}s$.
\end{Solution}
\begin{Solution}{4.{48}}
	$B = \frac{\epsilon\omega r}{2cd}V_0\cos\omega t$, де $r$~-- відстань від осі конденсатора.
\end{Solution}
\begin{Solution}{4.{49}}
	$B = \frac{2\pi\lambda r V_0}{cd}\left( \sin\omega t + \frac{\epsilon \omega}{4\pi\lambda}\cos\omega t\right) $.
\end{Solution}
\begin{Solution}{4.{51}}
	Це можливо за умови $\vect{j}_0\cdot\vect{k} = 0$.
\end{Solution}
\begin{Solution}{4.{53}}
	$\vect{S} = \frac{q^2}{4\pi r^5} \left( \vect{v} \cdot \vect{r}\right)$ .
\end{Solution}
\begin{Solution}{4.{54}}
	Вектор Пойнтінга в дипольному наближенні $\vect{S} = \frac{1}{4\pi} \left( \Efield^2 \vect{v} - \Efield\left(\Efield\cdot\vect{v} \right) \right) $.
\end{Solution}
\begin{Solution}{4.{55}}
	$\Phi = I^2R$
\end{Solution}
\begin{Solution}{4.{56}}
	W = $\frac{2\pi^2}{c^2} \frac{N^2\EMF^2}{lR^2} = \frac{1}{2c^2} LI^2$.
\end{Solution}
\begin{Solution}{4.{60}}
	$S  = \frac{P}{4\pi\ln\left( \frac{d}{R} \right) } \frac{d^2}{x^2\left( x - d\right)^2 }$, де $x$~-- відстань від точки до осі одного з провідників.
\end{Solution}
\begin{Solution}{4.{61}}
	$\Delta p = \frac{EB}{2\pi c}V = 10^{-6}$~г$\cdot$см/с.
\end{Solution}
