% !TeX program = lualatex
% !TeX encoding = utf8
% !TeX spellcheck = uk_UA
% !BIB program = biber
% !TeX root =../EMProblems.tex

%=========================================================
\Opensolutionfile{answer}[\currfilebase/\currfilebase-Answers]
\Writetofile{answer}{\protect\section*{\nameref*{\currfilebase}}}
\chapter{Електродинаміка}\label{\currfilebase}
%=========================================================

\section{Закон Фарадея}



\subsection*{Індукція струмів в провідниках, що рухаються в магнітному полі}

\begin{Theory}\small
	Рекомендується прочитати~\cite[Глава 16, 17]{FLF6}.
	
	Причиною виникнення індукційних струмів в рухомих провідниках з точки зору лабораторної системи відліку є сила, що діє на заряди в рухомих провідниках з боку магнітного поля. 

	Для знаходження струмів та ЕРС, що виникають в таких умовах, зручно користуватись правилом потоку: ЕРС, що індукується в контурі можна визначити за формулою: 	
	\begin{equation}
		\EMF = - \frac1c \Bfield \cdot \frac{d\vect{S}}{dt},
	\end{equation} 
де $d\vect{S}$~-- площа, що замітається провідником за час $dt$.
\begin{Attention}
	Слід зауважити, що у задачах з рухомими провідниками, як правило, магнітне поле не змінюється в часі $\Bfield(t) = \const$, тому правило потоку можна записати у вигляді, що формально нагадує закон електромагнітної індукції Фарадея. Причини цього розкриваються в релятивістській електродинаміці.
\end{Attention}
\end{Theory}

%=========================================================
%\begin{problem}
%    Квадратна рамка з мідного дроту переміщується вертикально з зі швидкістю $v = 5$~м/с через область однорідного магнітного поля індукцією $B = 0.8$~Тл. Сторона рамки має $30$~см та діаметр дроту $d = 1$~мм, питомий опір $\rho = 1.7\cdot 10^{-8}$~Ом$\cdot$м. Магнітне поле займає область, одна зі сторін якої має довжину $20$~см. (див. рис.).
%\begin{enumerate}[label=\alph*)]
%	\item Обчисліть електричний струм, що протікає в рамці,
%	\item Обчисліть силу з якою необхідно переміщувати рамку?
%	\item Перевірте закон збереження енергії, порівнявши енергію, що розсіюється в контурі провідника, та механічну потужність, необхідну для переміщення дроту.
%\end{enumerate}
%\end{problem}


%=========================================================
\begin{problem} % Козел 9.47
Дві довгі паралельні мідні рейки, розташовані вертикально на відстані $l$ одна від одної, замкнуті вгорі на опір $R$ і поміщені в однорідне магнітне поле $B$, перпендикулярне до площини шин. Уздовж шин падає мідний провідник масою $m$. Тертя відсутнє. Чому дорівнює встановлене значення швидкості падіння?
\begin{solution}
	$v = c\frac{mgR}{(Bl)^2}$.
\end{solution}
\end{problem}


%=========================================================
\begin{problem}\label{prb:siv278_3}
По двом вертикальним рейкам, які з'єднані внизу опором $R$ і вгорі батареєю з ЕРС $\EMF$ та внутрішнім опором $r$, без тертя ковзає провідник довжиною $l$ та масою $m$ (рис.~\ref{siv278_3}). Вся система знаходиться в однорідному магнітному полі, індукція якого $\Bfield$ перпендикулярна до площини рисунка. Знайти максимальну швидкість провідника в полі тяжіння, нехтуючи опором рейок.
\begin{solution}
	$v = c \frac{\EMF + \frac{mg}{Bl}r}{\left( 1 +  \frac{r}{R}\right)Bl}$.
\end{solution}
\end{problem}


%=========================================================
\begin{problem}\label{prb:siv278_4}
По двом вертикальним рейкам, які з'єднані вгорі та внизу опорами $R$, без тертя ковзає провідник довжиною $l$, масою $m$ та опором $R$ (рис.~\ref{siv278_4}). Вся система знаходиться в однорідному магнітному полі, індукція якого $\Bfield$ перпендикулярна до площини рисунка. Знайти максимальну швидкість провідника в полі тяжіння, нехтуючи опором рейок.
\begin{solution}
	$v = \frac32 \frac{mgRc^2}{B^2l^2}$.
\end{solution}
\end{problem}

%=========================================================
\begin{figure}[h!]\centering
	%---------------------------------------------------------
	\begin{minipage}[t]{0.45\linewidth}\centering
		\begin{tikzpicture}
			\pgfmathsetmacro{\w}{1.5}
			\draw (-\w,-2) -- (-\w,2) to [battery={info={$\EMF$}}] (\w,2) -- (\w,-2) to [resistor={info={$R$}}]  (-\w,-2);
			\draw[thick] ({-\w-0.1},0.5) arc (-180:0:0.1) -- ({\w-0.1},0.5) arc (-180:0:0.1);
			\draw[-latex'] (0,0.5) -- +(0,-1) node[below] {$\vect{v}$};
			\draw (0,1.3) circle (0.2) node[right=1ex] {$\Bfield$};\fill (0,1.3) circle (0.05);
		\end{tikzpicture}
		\caption{До задачі~\ref{prb:siv278_3}}
		\label{siv278_3}
	\end{minipage}
	%---------------------------------------------------------
	\begin{minipage}[t]{0.45\linewidth}\centering
		\begin{tikzpicture}
			\pgfmathsetmacro{\w}{1.5}
			\draw (-\w,-2) -- (-\w,2) to [resistor={info={$R$}}] (\w,2) -- (\w,-2) to [resistor={info={$R$}}]  (-\w,-2);
			\draw[thick] ({-\w-0.1},0.5) arc (-180:0:0.1) to[resistor={info={$R$}}] ({\w-0.1},0.5) arc (-180:0:0.1);
			\draw[-latex'] (0,0.4) -- +(0,-1) node[below] {$\vect{v}$};
			\draw (0,1.3) circle (0.2) node[right=1ex] {$\Bfield$};\fill (0,1.3) circle (0.05);
		\end{tikzpicture}
		\caption{До задачі~\ref{prb:siv278_4}}
		\label{siv278_4}
	\end{minipage}
	%---------------------------------------------------------
\end{figure}

%=========================================================
\begin{problem}\label{prb:rotated_disk_with_current} %Griffiths 7.4
Металевий диск радіусом $r$ обертається з кутовою швидкістю $\omega$ навколо вертикальної осі у однорідному магнітному полі $\Bfield$, яке перпендикулярне площині диска. Схема здійснюється шляхом підключення одного кінця резистора, опором $R$ до осі, а іншого~-- до рухомого контакту, який торкається зовнішнього краю диска (рис.~\ref{rotated_disk_with_current}). Знайдіть струм у колі.
\begin{solution}
	$I = \frac{\omega B r^2}{2cR}$.
\end{solution}
\end{problem}

%=========================================================
\begin{problem}[Дисковий генератор Фарадея]\label{prb:2rotateddisks}
Два диска радіусами $R_1$ та $R_2$ обертаються з кутовою швидкістю $\omega$ в однорідному магнітному полі з індукцією $\Bfield$, перпендикулярній їх площині (рис.~\ref{2rotateddisks}). Центри дисків приєднані до обкладок конденсатора $C_1$, ободи --- через ковзаючі контакти до обкладок конденсатора $C_2$. Знайти напруги на конденсаторах, якщо диски обертаються в одному напрямку і якщо в різних.
\begin{solution}
	$V_1  = \frac{B\omega C_2 \left| R_1^2 \pm R_2^2 \right| }{2(C_1 + C_2)}$, $V_2  = \frac{B\omega C_1 \left| R_1^2 \pm R_2^2 \right| }{2(C_1 + C_2)}$. Знак <<$+$>> стосується випадку обертання дисків
	в різних напрямках, <<$-$>>~--- до обертання дисків в одному напрямку.
\end{solution}
\end{problem}

%=========================================================
\begin{figure}[h!]\centering
	%---------------------------------------------------------
	\begin{minipage}[t]{0.45\linewidth}\centering
		\begin{tikzpicture}
			\draw[black, fill=lightgray] (0,0) circle (1); \draw[-latex'] (0,0) -- node[pos=0.1, above] {$R_1$} (45:1);
			\draw[black, fill=lightgray] (4,0) circle (0.8);\draw[-latex'] (4,0) -- node[pos=0.1, above] {$R_2$} +(45:0.8);
			\fill (4,0) circle (0.05);
			\fill (0,0) circle (0.05);
			\draw (0,0) -- (0,-2) to[capacitor={label={$C_1$}}] ++(4,0) -- (4,0) ;
			\draw (150:1.1) arc (150:30:1.1);
			\draw (4,0) +(150:0.9) arc (150:30:0.9);
			\draw (90:1.1) -- (0,2) to[capacitor={label={$C_2$}}] ++(4,0) -- +(-90:1.1);
			\draw (2,0) circle (0.2) node[below=1ex] {$\Bfield$};\fill (2,0) circle (0.05);
		\end{tikzpicture}
		\caption{До задачі~\ref{prb:2rotateddisks}}
		\label{2rotateddisks}
	\end{minipage}
	%---------------------------------------------------------
	\begin{minipage}[t]{0.45\linewidth}\centering
		\begin{tikzpicture}
			\fill[gray!50, draw=black] (0,0) ellipse (2 and 1);
			\draw [-latex'] (0,0) [partial ellipse=135:405:1cm and 0.5cm];
			\draw [thick](0,0) node[contact] {} -- (0,2);
			\fill[red!50, draw=black] (0.2,2.2) -- (0.2,2.07) arc (0:-180:0.2 and 0.07) -- (-0.2,2.2);
			\fill[red!50, draw=black] (0,2.2) ellipse (0.2 and 0.07) ;
			\fill[red!50, draw=black, yshift=-4.4cm] (0.2,2.2) -- (0.2,2.07) arc (0:-180:0.2 and 0.07) -- (-0.2,2.2);
			\fill[red!50, draw=black, yshift=-4.4cm] (0,2.2) ellipse (0.2 and 0.07) ;
			\draw [thick] (-90:1) -- (-90:2.2);
			\draw ++(2.1,0.1) arc (90:270:0.1) -- ++(0,-3) to [resistor={info'={R}}] ++(-1.8,0) -- ++(0,0.73) arc (270:90:0.1);
			\draw [-latex', blue] (-1,0) -- +(0,2) node[left, black] {$\Bfield$};
			\draw [-latex', blue, xshift=2cm] (-1,0) -- +(0,2) node[right, black] {$\Bfield$};
		\end{tikzpicture}
		\caption{До задачі~\ref{prb:rotated_disk_with_current}}
		\label{rotated_disk_with_current}
	\end{minipage}
	%---------------------------------------------------------
\end{figure}


%=========================================================
\begin{problem}
Металевий циліндр радіусом $R$ вміщений в однорідне постійне магнітне поле $\Bfield$, спрямоване уздовж його осі. Циліндр обертають з постійною кутовою швидкістю $\vect\omega$ навколо своєї осі. Знайти:
\begin{enumerate*}[label=\alph*)]
	\item різницю потенціалів між поверхнею циліндра і віссю;
	\item поверхневу і об'ємну  густини зарядів в циліндрі.
\end{enumerate*}
Переконайтесь, що повний заряд на одиницю довжини дорівнює нулю.
\begin{solution}
	\begin{enumerate*}[label=\alph*)]
		\item $V = \frac{B\omega R^2}{2c}$;
		\item $\sigma \frac{\omega B R}{4\pi}$, $\rho = -\frac{\omega B}{2\pi}$.
	\end{enumerate*}
\end{solution}
\end{problem}

%=========================================================
\begin{problem}
Діелектричний циліндр радіусом $R$ та проникністю $\epsilon$ вміщений в однорідне постійне магнітне поле $\Bfield$, спрямоване уздовж його осі. Циліндр обертають з постійною кутовою швидкістю $\vect\omega$ навколо своєї осі. Знайти об'ємну і поверхневу густини зв'язаних зарядів в циліндрі. Переконайтесь, що повний заряд на одиницю довжини дорівнює нулю.
\begin{solution}
		$\rho' = -\frac{\epsilon - 1}{2\pi c} (\vect\omega\times\Bfield)$, $\sigma' = \frac{\epsilon - 1}{2\pi c} (\vect\omega\times\Bfield)R$.
\end{solution}
\end{problem}

%=========================================================
\begin{problem}\label{prb:Zhurnal_kvant_1976_1_p54}
    На непровідному диску радіуса $R$ закріплений по хорді дріт довжиною $l$ (рис.~\ref{Zhurnal_kvant_1976_1_p54}). Диск обертається з постійною кутовою швидкістю $\omega$. Перпендикулярно до диска напрямлене магнітне поле з індукцією $B$. Знайти напругу між серединою і кінцем дроту.
\begin{solution}
	$V = \frac{Bl^2\omega}{8c}$.
\end{solution}
\end{problem}

%=========================================================
\begin{problem}\label{prb:wire_and_frame_with_jumper}
Довгий прямолінійний провідник, по якому тече струм $I_0$, і $\Pi$-подібний провідник $ABCD$ з рухомою перемичкою $AB$ довжини $l$ розташовані в одній площині. Сторона $CD$ контуру розташована на відстані $a$ від провідника. Перемичку переміщують із заданою постійною швидкістю $v$ (рис.~\ref{wire_and_frame_with_jumper}). Знайти:
\begin{enumerate*}[label=\alph*)]
	\item ЕРС індукції в контурі як функцію відстані від перемички до провідника,
	\item силу струму в контурі, якщо опір одиниці довжини всіх складових його провідників дорівнює $r$.
\end{enumerate*}
Індуктивністю контура можна знехтувати.
\begin{solution}
	\begin{enumerate*}[label=\alph*)]
		\item $\EMF = \frac{2I_ol}{c(a+vt)}$,
		\item $I = \frac{I_0 l v}{cr(a+vt)(l+vt)}$.
	\end{enumerate*}
\end{solution}
\end{problem}



%=========================================================
\begin{figure}[h!]\centering
	%---------------------------------------------------------
	\begin{minipage}[t]{0.45\linewidth}\centering
			\begin{tikzpicture}
				\fill[gray!50, draw=black] (0,-0.05) ellipse (2 and 1);
				\fill[gray!50, draw=black] (0,0) ellipse (2 and 1);
				\draw [-latex'] (0,2) [partial ellipse=135:405:0.5cm and 0.25cm];
				\draw [thick](0,0)  -- (0,2) node[above] {$\omega$};
				\draw [thick] (-90:1.05) -- (-90:2.2);
				\draw [latex'-, blue] (-1,0) -- +(0,2) node[left, black] {$\Bfield$};
				\draw [latex'-, blue, xshift=2cm] (-1,0) -- +(0,2) node[right, black] {$\Bfield$};
				\draw[-stealth] (0,0) -- (135:2 and 1) node[pos=0.5, below] {$R$};
				\draw[line width = 2pt, red, rounded corners] (0,0) +(275:2 and 1) -- node[above, black] {$l$} +(355:2 and 1);
			\end{tikzpicture}
	\caption{До задачі~\ref{prb:Zhurnal_kvant_1976_1_p54}}
	\label{Zhurnal_kvant_1976_1_p54}
	\end{minipage}
	%---------------------------------------------------------
	\begin{minipage}[t]{0.45\linewidth}\centering
		\begin{tikzpicture}
			\draw [thick] (0,0)  -- (0,3) to [current direction={info={$I_0$}}] (0,4);
			\draw (4,1) -- ++(-3,0) node [below left] {$C$} -- ++(0,2) node [above left] {$D$} -- +(3,0);
			\fill [gray, draw=black, thin] (3,0.9)  node[below,black] {$A$} arc (-90:90:0.1)
			-- (3,2.9) arc (-90:90:0.1) node[above,black] {$B$} --
			++(0,0.02) arc (90:-80:0.12) -- ++(0,-1.764) arc (80:-90:0.12) -- cycle
			;
			\draw [-latex'] (3.02,2) -- (4,2) node[right] {$\vect{v}$};
			\draw[latex-latex] (0,2) -- node[above] {$a$} +(1,0);
		\end{tikzpicture}
		\caption{До задачі~\ref{prb:wire_and_frame_with_jumper}}
		\label{wire_and_frame_with_jumper}
	\end{minipage}
	%---------------------------------------------------------
\end{figure}
%=========================================================

%=========================================================
\begin{problem}\label{prb:EMF_throu_rectangle}
Квадратна рамка зі стороною $a$ рухається зі постійною швидкістю в напрямку від провідника, по якому тече струм $I$ (рис.~\ref{EMF_throu_rectangle}). Відстань від провідника до лівої сторони  рамки $s(t)$. Знайдіть ЕРС індукції, яка виникає в рамці. В якому напрямку тече струм? Визначити взаємну індуктивність системи.
\begin{solution}
	$\EMF = \frac{2Ia^2v}{c(b+vt)(a+ b  +vt)}$, $L_{12} = 2a\ln \left( 1+ \frac{a}{s}\right) $.
\end{solution}
\end{problem}

%=========================================================
\begin{problem}\label{prb:dropedwire}
По двом мідним шинам, сполученим котушкою індуктивності $L$, без тертя ковзає провідник довжиною $l$ і масою $m$ (рис.~\ref{dropedwire}). Система знаходиться в однорідному магнітному полі з індукцією $\Bfield$, зображеному на рисунку. У початковий момент часу провіднику поштовхом надали швидкість $v_0$. Нехтуючи опором контуру, знайти закон руху провідника та закон зміни сили струму в контурі.
\begin{solution}
	$x = \frac{v_0}{\omega}\sin\omega t$,
	$I = -v_0\sqrt{\frac{m}{L}}\sin\omega t$,
	де $\omega = \frac{lB}{\sqrt{mL}}$.
\end{solution}
\end{problem}

%=========================================================
\begin{figure}[h!]\centering
%---------------------------------------------------------
\begin{minipage}[t]{0.45\linewidth}\centering
	\begin{tikzpicture}
		\draw (4,1) -- ++(-4,0)  to[inductor={info={$L$}}] ++(0,2)  -- +(4,0);

		\fill [gray, draw=black, thin] (3,0.9)  arc (-90:90:0.1)
		-- (3,2.9) arc (-90:90:0.1)  --
		++(0,0.02) arc (90:-80:0.12) -- ++(0,-1.764) arc (80:-90:0.12) -- cycle
		;
		\draw [-latex'] (3.02,2) -- (4,2);
		\draw (1.5,2) circle (0.2) node[below=1ex] {$\Bfield$};\fill (1.5,2) circle (0.05);
	\end{tikzpicture}
	\caption{До задачі~\ref{prb:dropedwire}}
	\label{dropedwire}
\end{minipage}
%---------------------------------------------------------
	\begin{minipage}[t]{0.45\linewidth}\centering
		\begin{tikzpicture}
			\draw [thick] (0,0)  -- (0,3) to [current direction={info={$I_0$}}] (0,4);
			\draw [thick] (1,1.5) rectangle +(1,1);
			\draw [latex'-latex'] (0,2) -- node [below] {$s$} (1,2);
			\draw [-latex'] (2,2) -- (3,2) node [right] {$\vect{v}$};
		\end{tikzpicture}
		\caption{До задачі~\ref{prb:EMF_throu_rectangle}}
		\label{EMF_throu_rectangle}
	\end{minipage}
%---------------------------------------------------------
\end{figure}
%=========================================================
\subsection*{Явище електромагнітної індукції}

\begin{Theory}
	Закон електромагнетної індукції Фарадея: ЕРС, що виникає в контурі, прямо пропорційна швидкості зміни магнітного потоку, що пронизує площу $S$, яка охоплюється цим контуром:
	\begin{equation}
		\EMF = - \frac1c\frac{\partial}{\partial t}\iint\limits_S \Bfield\cdot d\vect{S}.
	\end{equation}
	Причиною виникнення ЕРС в цьому випадку в лабораторній системі відліку є вихрове електричне поле, яке виникає за рахунок зміни в часі магнітного поля.
\end{Theory}
\begin{problem}% КРС 3.117
Електричний заряд $q$ рівномірно розподілений по довжині твердого непровідного тонкого кругового кільця масою $m$. Кільце може вільно обертатися навколо своєї осі. Спочатку кільце знаходилось в спокої. Потім було увімкнено однорідне магнітне поле $B = B (t)$, перпендикулярне площині кільця, яке довільно змінюється у часі. Знайти кутову швидкість обертання кільця.
\begin{solution}
	$\omega = \frac{qB(t)}{2mc}$.
\end{solution}
\end{problem}

%=========================================================
\begin{problem}\label{prb:condensators_around}
Контур є колом, з'єднаним по діаметру конденсаторами ємностями $C_1$, $C_2$ та $C_3$, які увімкнені в розриви провідників (рис.~\ref{condensators_around}), і знаходиться в однорідному змінному магнітному полі. Швидкість зміни магнітного потоку через площу кільця постійний і дорівнює $\dot{\Phi}$. Знайти заряди конденсаторів, якщо спочатку конденсатори були не заряджені.
\begin{solution}
	$q_1 = - \frac1c \dot{\Phi}\frac{C_1(C_2 + C_3/2)}{C_1 + C_2 +C_3}$, $q_2 =  \frac1c \dot{\Phi}\frac{C_2(C_1 + C_3/2)}{C_1 + C_2 +C_3}$, $q_3 =  \frac1c \dot{\Phi}\frac{C_3/2(C_2 - C_1)}{C_1 + C_2 +C_3}$.
\end{solution}
\end{problem}

%=========================================================
\begin{problem}\label{prb:Griffiths7.53} %https://www.youtube.com/watch?v=PQN7Dke9pX8
Потік магнітного поля створений довгим соленоїдом, пронизує електричне коло, що складається з двох резисторів опорами $R_1$ та $R_2$ збільшується з часом за законом $\Phi = \alpha t$ (рис.~\ref{Griffiths7.53}). Що покажуть вольтметри?
\begin{solution}
	$V_1 = \alpha \frac{R_1}{R_1 + R_2}$, $V_2 = - \alpha \frac{R_2}{R_1 + R_2}$.
\end{solution}
\end{problem}

%=========================================================
\begin{figure}[h!]\centering
	%---------------------------------------------------------
	\begin{minipage}[t]{0.45\linewidth}\centering
		\begin{tikzpicture}
			\newlength{\radiic}\setlength{\radiic}{1.25cm}
			\draw (0:\radiic) arc (0:85:\radiic);
			\draw (85:\radiic) to [capacitor= {info'={$C_1$}}] (95:\radiic);
			\draw (95:\radiic) arc (95:180:\radiic);
			\draw (180:\radiic) arc (180:265:\radiic);
			\draw (275:\radiic) arc (275:360:\radiic);
			\draw (265:\radiic) to [capacitor = {info={$C_2$}}] (275:\radiic);
			\draw (180:\radiic) node[contact] {} to [capacitor = {info={$C_3$}}] (360:\radiic) node[contact] {};
		\end{tikzpicture}
		\caption{До задачі~\ref{prb:condensators_around}}
		\label{condensators_around}
	\end{minipage}
	%---------------------------------------------------------
	\begin{minipage}[t]{0.45\linewidth}\centering
		\begin{tikzpicture}[scale=1.5]
			\draw (0,0) node[contact] (A) {} to [resistor={info={$R_1$}}] ++(0,2) node[contact] (B) {}-- ++(2,0) node[contact] (C) {} to [resistor={info={$R_2$}}] ++(0,-2) node[contact] (D) {} -- (0,0)
			(B) -- ++(-1.5,0) to [multimeter={info=center:$\mathrm{V}$}] ++(0,-2) -- (A)
			(C) -- ++(1.5,0) to [multimeter={info=center:$\mathrm{V}$}] ++(0,-2) -- (D);
			\node at (1,1) {$\oplus$};
			\node [below] at (1,1) {$\Phi$};
			\draw [-latex'] (1,1) +(180:0.7) arc (180:-5:0.7);
			\draw [-latex'] (1,1) +(0:0.7) arc (0:-185:0.7);
		\end{tikzpicture}
		\caption{До задачі~\ref{prb:Griffiths7.53}}
		\label{Griffiths7.53}
	\end{minipage}
	%---------------------------------------------------------
\end{figure}

\subsection*{Індуктивністі та коефіцієнт взаємної індукції. Теорема взаємності}

\begin{Theory}
Магнітний потік, що пронизує магнітне поле $i$-го контура, що пронизує $k$-й контур:
\begin{equation}
	\Phi_{i} = \sum\limits_{i,k}\frac1c L_{ik}I_i,
\end{equation}
де $L_{ik}$~-- називається коефіцієнтом самоіндукції (або індуктивністю), якщо $i = k$, і коефіцієнтом взаємоіндукції, якщо $i \neq k$.

Для випадку двох витків, магнітні потоки, що взаємно пронизують два витка (або дві котушки), по яких течуть струми $I_1$ та $I_2$:
	\begin{align*}
		\Phi_1 = \frac1c L_{11}I_1 + \frac1c L_{12}I_2 \\
		\Phi_2 = \frac1c L_{21} I_1 + \frac1c L_{22} I_2,
	\end{align*}
де $L_{11}$ та $L_{22}$~-- індуктивності витків (або котушок), $L_{12}$ та $L_{21}$~-- коефіцієнти взаємної індукції витків (або котушок).

Теорема взаємності стверджує, що 
	\begin{equation}
		L_{12} = L_{21}.
	\end{equation}

Згідно закону електромагнітної індукції,  ЕРС, що виникають в контурах (або котушках):
\begin{align}
	\EMF_{1} &= -\frac 1c L_{11}\frac{dI_1}{dt} - \frac 1c L_{12}\frac{dI_2}{dt} \\
	\EMF_{2} &= -\frac 1c L_{21} \frac{dI_1}{dt} - \frac 1c L_{22}\frac{dI_2}{dt},
\end{align}

Коефіцієнт зв'язку контурів:
\begin{equation}
	k = \sqrt{\frac{L_{12}^2}{L_1L_2}} \le 1.
\end{equation}
%, які визначаються за формулою:
%\begin{equation}
%	L_{ik} = \mu\oint\limits_{L_1}\oint\limits_{L_2}\frac{I_idl_i I_kdl_k}{\left|\vect{r} - \vect{r}_i\right|}.
%\end{equation}

Магнітна енергія системи електричних струмів:

\begin{equation}
	W_m = \frac{1}{2c^2} \sum\limits_{i,k} L_{ik}I_iI_k.
\end{equation}

У випадку одинарного контуру (або котушки), енергія магнітного поля:
\begin{equation}
	W = \frac{1}{c^2}\frac{LI^2}{2}.
\end{equation} 
\end{Theory}

%=========================================================
\begin{problem}
    Обчисліть індуктивність довгого соленоїда довжиною $l$, який має $N$ витків. Витки намотані на магнетик, з проникністю $\mu$, площа перерізу якого $S$. 
\begin{solution}
	$L = \frac{4\pi\nu N^2 S}{l}$.
\end{solution}
\end{problem}

%=========================================================
\begin{problem}\label{prb:cylinder_inductance}
    Обчисліть індуктивність тороїдальної котушки, що намотана на магнетик проникністю $\mu$, який має форму циліндра висотою $b$ з внутрішнім радіусом $R$ і зовнішнім радіусом $R + a$. Число витків дорівнює $N$.
\begin{solution}
	$L  = 2bM\mu\ln\left( 1 + \frac{a}{R}\right) $.
\end{solution}
\end{problem}

%=========================================================
\begin{problem}
    Знайдіть взаємну індукцію тороїдальної котушки, параметри якої подано в задачі~\ref{prb:cylinder_inductance} та прямолінійного нескінченно довгого провідника, який проходить вздовж осі циліндра.
\begin{solution}
	$L_{12} = 2bN\mu \ln\left( 1 + \frac{a}{R}\right) $.
\end{solution}
\end{problem}

%=========================================================
\begin{problem}% Киселев 8.3.13
Коаксіальний кабель складається з суцільного внутрішнього провідника радіусом $R_1$ і тонкого зовнішнього циліндричного провідника радіусом $R_2$. Знайти індуктивність одиниці довжини кабелю. Вважати, що магнітна проникність матеріалу провідників і зазору між ними дорівнює $\mu$, струм розподілений по провідникам рівномірно.
\begin{solution}
	$L/l = \frac{\mu}{2} \left( 1 + 4\ln\frac{R_2}{R_1}\right)$.
\end{solution}
\end{problem}

%=========================================================
\begin{problem}\label{prb:inductance_two_plates}
Знайти індуктивність одиниці довжини двох широких нескінченно довгих провідників, які знаходяться на відстані $a$ один від одного. Ширина провідників $b$ ($b \gg a$). Площини провідників паралельні.
\begin{solution}
	$L = 4\pi\frac{a}{b}$.
\end{solution}
\end{problem}

%=========================================================
\begin{problem}\label{prb:inductance_two_wires}
Знайти індуктивність одиниці довжини двох циліндричних нескінченно довгих провідників, які знаходяться на відстані $a$ один від одного. Радіус провідників $R$ ($R \ll a$).
\begin{solution}
	$L = \frac{1}{4\pi^2}\ln\frac{a}{R}$.
\end{solution}
\end{problem}

%=========================================================
\begin{problem}
    Знайдіть коефіцієнт взаємоіндукції довгого дроту та квадратної рамки стороною $a$. Рамка і дріт лежать в одній площині. Сторони рамки і дроту паралельні, найближча до дроту сторона рамки знаходиться на відстані $l$ від нього.
\begin{solution}
	$L_{12} = 2a\ln\left( 1 + \frac{a}{l}\right).$
\end{solution}
\end{problem}

%=========================================================
\begin{problem}
    Знайдіть коефіцієнт взаємоіндукції двох тонких витків однакового радіуса $R$, які лежать на відстані $l \gg R$. Осі витків співпадають. Площі витків паралельні.
\begin{solution}
	$L_{12} \approx \frac{2\pi^2R^4}{l^3}$.
\end{solution}
\end{problem}


%=========================================================
\begin{problem}
Між двом паралельним провідникам нескінченної довжини, які знаходяться на відстані $2a$ один від одного лежить кругле провідне кільце радіусом $a$, яке дотикається до провідників і ізольоване від них. Знайдіть коефіцієнт взаємної індукції між круглим провідником і двома прямими провідниками.
\begin{solution}
	$M = 8\pi a$.
\end{solution}
\end{problem}

\subsection*{Задачі, які розв'язуються за допомогою теореми взаємності}

%=========================================================
\begin{problem}
    Два тонкі колові провідники, осі яких співпадають, лежать в одній площині. Радіус зовнішнього провідника $R_1$ внутрішнього $R_2$ ($R_2 \ll R_1$). Знайдіть
	\begin{enumerate*}[label=\alph*)]
		\item коефіцієнт взаємоіндукції провідників та
		\item магнітний потік, що пронизує площу зовнішнього провідника, якщо по внутрішньому провіднику тече струм $I$.
	\end{enumerate*}
\begin{solution}
		\begin{enumerate*}[label=\alph*)]
		\item $L_{12} \approx \frac{2\pi^2R_1^2}{R_2}$
		\item $\Phi_{12} = \frac{2\pi^2}{c} \frac{R_1^2}{R_2} I$.
		\end{enumerate*}
\end{solution}
\end{problem}


%=========================================================
\begin{problem}\label{prb:flux_throu_infinite_plane}
Знайти потік вектора індукції магнітного поля, який створюється квадратною рамкою зі стороною $a$, по якій тече струм $I$, через напівплощину, границя якої розташована на відстані $b$ від однієї з сторін рамки (рис.~\ref{flux_throu_infinite_plane}).
\begin{solution}
	$\Phi = \frac{2Ia}{c}\ln\left( 1 + \frac{a}{b}\right) $.
\end{solution}
\end{problem}
%---------------------------------------------------------
\begin{figure}[h!]\centering
	\begin{tikzpicture}[scale=0.9]
		\node (wall) [ground, minimum width=4cm, minimum height=2cm, rotate=90] at (0,0) {};
		\draw (wall.south west) -- (wall.south east);
		\draw [thick] ([shift={(2,-1)}]wall.south) -- ++(0,2) to [current direction={info={$I$}}] ++(2,0) -- ++(0,-2) -- ++(-2,0) ;

		\draw [latex'-latex'] (wall.south) -- node [below] {$b$}([xshift=2cm]wall.south);
	\end{tikzpicture}
	\caption{До задачі~\ref{prb:flux_throu_infinite_plane}}
	\label{flux_throu_infinite_plane}
\end{figure}
%----------------------------------------------------------

%=========================================================
\begin{problem}% КРС 7.15
Два коаксіальних кругових витка радіусами $R$ і $r$ ($r \ll R$) розміщені на відстані $R$ один від одного. По малому витку пропускається струм  $i = i_0\cos\omega t$. Знайти струм $I$ в великому витку, провідність якого дорівнює~$G$.
\begin{solution}
	$I = \frac{\pi^2 r^2 G \omega}{\sqrt{2}c^2R}i_0\sin\omega t$.
\end{solution}
\end{problem}

%=========================================================
\begin{problem}\label{prb:rotated_magnetic_needle} %КРС 7.13
Магнітний диполь з моментом $p_m$ обертається з частотою $\omega$ навколо осі, яка проходить через його центр і перпендикулярна магнітному моменту (рис.~\ref{rotated_magnetic_needle}). Знайти струм в плоскому нерухомому кільці радіусом $a$ з опором $R$, яке знаходиться на відстані $l \gg a$ від диполя. Нормаль $\vect{n}$ до площини кільця перпендикулярна осі обертання диполя. Самоіндукцією рамки знехтувати.
\begin{solution}
	$I = \frac{2p_m\pi a^2\omega}{cRl^3}\sin\omega t$.
\end{solution}
\end{problem}
%---------------------------------------------------------
\begin{figure}[h!]\centering
	\begin{tikzpicture}[scale=1.5]
		\tdplotsetmaincoords{0}{0}
		\begin{scope}[tdplot_main_coords, rotate around x=70, rotate around z=-45]
			\draw [dashed] (0,0,0) circle (1);
			\fill [blue] (180:0.2) -- (90:1) -- (0:0.2) -- cycle;
			\fill [red] (180:0.2) -- (-90:1) -- (0:0.2) -- cycle;
		\end{scope}
		\draw [-latex'] (0,-1) -- (0,1) node [above] {$\omega$};
		\draw [-latex'] (3,0) -- (3.5,0) node [right] {$\vect{n}$};
		\begin{scope}[shift={(3,0,0)}, tdplot_main_coords, rotate around y=60]
			\draw [thick] (0,0,0) circle (0.4);
			\draw [-latex'] (0,0) -- +(110:0.4) node[above] {$a$};
		\end{scope}
		\draw (3,0) -- ([yshift=-1cm]3,0);
		\draw [dash dot] (0,0) -- (3,0);
		\draw [latex'-latex']  (0,-0.8) -- node [above] {$l$}(3,-0.8);
	\end{tikzpicture}
	\caption{До задачі~\ref{prb:rotated_magnetic_needle}}
	\label{rotated_magnetic_needle}
\end{figure}
%---------------------------------------------------------

%=========================================================
\begin{problem} %КРС 7.19
Намагнічена куля пролітає вздовж осі тонкої котушки, з'єднаної з балістичним гальванометром через ідеальний випрямляючий елемент. Куля намагнічена уздовж своєї осі, її розміри малі в порівнянні з діаметром котушки $D$. Визначити магнітний момент кулі $p_m$, якщо відомо, що гальванометр відхилився після прольоту кулі на кут $\phi$. Відомі балістична постійна гальванометра $b$~[рад/Кл], число витків котушки $n$ і опір кола $R$.
\begin{solution}
	В SI $p_m = \frac{\phi R D}{4\pi b n}$.
\end{solution}
\end{problem}

\section{Надпровідники у магнітному полі}

\begin{Theory}
	Рекомендується вивчити~\cite[\S~80]{Siv3}.
\end{Theory}
%=========================================================
\begin{problem}\label{sphere:Superconductor_in_field} % КРС 3.163 1987
Надпровідна куля радіусом $R$ вміщена в однорідне магнітне поле з індукцією $\Bfield_0$. Знайдіть магнітний момент кулі, магнітне поле $\Bfield$ в усьому просторі та розподіл струму в кулі.
\begin{solution}
	$\vect{p}_m = - \frac12 R^3\Bfield_0$,
	$
		\Bfield =
		\begin{cases}
			0,                                                                                                & r \le R \\
			\left( 1 + \frac{R^3}{2r^3}\right)\Bfield_0 - \frac{3R^3(\Bfield_0\cdot \vect{r})\vect{r}}{2r^5}, & r > R,
		\end{cases}.
	$

	Густина об'ємних струмів намагнічування $\vect{j}' = 0$.

	Поверхнева густина струмів намагнічування
	$i = -\frac{3c}{8\pi} \frac{\Bfield_0\vect{r}}{R}$,
	де $\vect{r}$~-- радіус-вектор поверхні провідника.
\end{solution}
\end{problem}

%=========================================================
\begin{problem}
Надпровідна куля радіусом $R$, яка складається з однакових півсфер, вміщена в однорідне магнітне поле з індукцією $B_0$, перпендикулярне площині розрізу. Знайдіть силу, необхідну для відриву однієї півсфери від іншої.
\begin{solution}
	$F  = \frac{9}{64}B_0^2R^2$.
\end{solution}
\end{problem}

\subsection*{Закон збереження магнітного потоку для надпровідників}


%=========================================================
\begin{problem}
В центр надпровідного кільця індуктивністю $L$ і радіусом $R$ внесено магнітний диполь з дипольним моментом $p_m$, який напрямлений вздовж осі кільця. Який струм установиться в кільці?
\begin{solution}
	Магнітний потік, що пронизує надпровідне кільце зберігається $\Phi = \const$. Оскільки, в спочатку диполь не було внесено, то $\Phi = 0$, і залишатиметься таким же. Коли магнітний диполь опиниться в центрі кільця, то магнітний потік, що пронизує кільце, можна порахувати як:
	\[
		\Phi = \frac1c LI + \frac1c L_{21}I_m = 0,
	\]
	де $L$~-- індуктивність кільця, $I$~-- струм, що тече по кільцю, $L_{21}$~-- коефіцієнт взаємоіндукції, $I_m$~-- умовний струм, що циркулює в диполі $p_m$ ($I_m = \frac{cp_m}{S}$, $S$~-- умовна площа витка диполя).

	З закону збереження магнітного потоку випливає, що $I = - \frac{L_{12}}{L}\frac{cp_m}{S}$.
	Для знаходження $L_{21}$, скористаємось теоремою взаємності, $L_{21} = L_{12} = \frac{2\pi S}{R}$.
	Отже,
	\[
		I = - \frac{ 2\pi c p_m}{RL}.
	\]
	Знак мінус вказує на те, що індукційний що магнітний момент струму протилежний магнітного моменту диполя.
\end{solution}
\end{problem}


%=========================================================
\begin{problem}
Провідне кільце з індуктивністю $L$ знаходиться в нормальному стані в зовнішньому магнітному полі (магнітний потік через контур кільця дорівнює $\Phi_0$). Потім температура знижується і кільце переходить в надпровідний стан. Який струм буде текти по кільцю, якщо вимкнути зовнішнє магнітне поле?
\begin{solution}
	$I = \frac{c\Phi_0}{L}$.
\end{solution}
\end{problem}

%=========================================================
\begin{problem}
У постійному однорідному магнітному полі з індукцією $B$ знаходиться кругле жорстке надпровідникове кільце радіусом $R$ малого перерізу. У початковий момент площина кільця паралельна напрямку магнітного поля, а струм в кільці відсутній. Визначити силу струму в кільці відразу після того, як воно було повернуто так, що площина кільця стала перпендикулярна до ліній магнітного поля. Знайти витрачену роботу.
\begin{solution}
	$I = \frac{cB\pi R^2}{L}$, $A = \frac{\Phi^2}{2L} = \frac{B^2\pi^2 R^4}{2L}$.
\end{solution}
\end{problem}

\subsection*{Метод зображень в магнітостатиці}
%=========================================================
\begin{problem}
На якій висоті постійний магніт з магнітним моментом $p_m$ і масою $m$ буде левітувати в горизонтальному положенні над плоскою горизонтальною поверхнею надпровідника I роду? Магніт вважати точковим диполем.
\begin{solution}
	$h = \frac12 \sqrt[4]{\frac{3p_m^2}{mg}}$.
\end{solution}
\end{problem}

%=========================================================
\begin{problem}
Над плоскою поверхнею надпровідника I роду на ізолюючому шарі товщини $h = 5$~мм лежить тонке надпровідний кільце радіусом $R = 10$~см, по якому тече постійний струм. При якому струмі кільце почне левітувати над поверхнею надпровідника, якщо маса кільця $m = 1$~г?
\begin{solution}
	$I \ge c \sqrt{\frac{mgh}{2\pi R}} = 8.4\cdot 10^{10}$~Фр/с = $25$~А.
\end{solution}
\end{problem}

\subsection*{Сили, що діють на надпровідники в магнітному полі}

%=========================================================
\begin{problem} %КРС 7.21
Металева надпровідна кулька летить у напрямку до соленоїда уздовж його осі. Поле соленоїда $B_0 = 10^3$ Гс. Якою має бути початкова швидкість кульки, щоб вона змогла влетіти в соленоїд? Радіус кульки $2$~см, маса $1$~г.
\begin{solution}
	$v_0 = B_0 \sqrt{\frac{R^3}{2m}} \approx 20$~м/c.
\end{solution}
\end{problem}

%=========================================================
\begin{problem}
Невелика надпровідна кулька радіусом $r$ знаходиться на осі на відстані $z$ від площини кільця радіусом $R$, по якому тече струм $I$. Знайдіть силу взаємодії між кулькою та кільцем.
\begin{solution}
	$F = \left( \frac{2\pi I R}{c} \right)^2 \frac{3r^3 z}{2(R^2 + z^2)^4} $.
\end{solution}
\end{problem}


%=========================================================
\begin{problem}
Знайти розподіл магнітного тиску по поверхні надпровідного кулі радіусом $R$, що внесена в однорідне зовнішнє магнітне поле $\Bfield_0$.
\begin{solution}
	$ p = \frac{9}{32\pi} \Bfield_0\cdot\vect{r}$, де $\vect{r}$~-- радіус-вектор поверхні сфери.
\end{solution}
\end{problem}


\section{Рівняння Максвелла. Вектор Пойнтінга}

\subsection*{Закон електромагнітної індукції в інтегральній формі}
\begin{Theory}
	Закон електромагнетної індукції Фарадея: циркуляція вектора напруженості електричного поля вздовж довільного замкнутого контуру $L$, прямо пропорційна швидкості зміни магнітного потоку, що пронизує площу $S$, яка охоплюється цим контуром:
	\begin{equation}
		\oint\limits_L \Efield \cdot d\vect{r} = - \frac1c\iint\limits_S \frac{\partial \Bfield}{\partial t}\cdot d\vect{S}.
	\end{equation}
\end{Theory}

%=========================================================
\begin{problem}
    Чи можна за допомогою змінного в часі магнітного поля створити однорідне електричне поле?
\end{problem}

%=========================================================
\begin{problem}\label{prb:aka_Faynmann_disk_paradox}
Непровідне кільце масою $m$, яке має заряд $q$, може вільно обертатись навколо своєї осі. У початковий момент кільце знаходиться в стані спокою і магнітне поле відсутнє. Потім увімкнули однорідне магнітне поле, перпендикулярне до площини кільця, яке почало зростати за деяким законом $\Bfield(t)$. Знайти кутову швидкість кільця в залежності від величини поля.
\begin{solution}
	$\vect\omega = -\frac{q\Bfield(t)}{2mc}$.
\end{solution}
\end{problem}


%=========================================================
\begin{problem}
На поверхні довгого суцільного непровідного циліндра радіусом $R$ рівномірно розподілений заряд з поверхневою густиною $\sigma$. Циліндр може обертатися без тертя навколо своєї осі. Зовнішнє однорідне магнітне поле з вектором індукції $\Bfield$ направлено вздовж осі циліндра. Знайти кутову швидкість обертання, яку набуде циліндр після вимикання магнітного поля. Густина маси речовини циліндра $\rho$, спочатку циліндр нерухомий.
\begin{solution}
	$\vect\omega = \frac{2\sigma\Bfield}{c\rho R}$.
\end{solution}
\end{problem}


%=========================================================
\begin{problem}\label{prb:solenod_in_dielectric}
На довгий соленоїд з густиною намотки $n$ та радіусом $R$, по витках якого тече змінний струм величиною, що змінюється за законом $I(t) = I_0 \cos\omega t$, щільно надітий тор з діелектрика проникністю $\epsilon$ (рис.~\ref{solenod_in_dielectric}). У торі є дуже тонкий поперечний розріз. Знайти напруженість електричного поля в зазорі залежно від відстані до осі соленоїда. 
\begin{solution}
	$E(r) = \frac{2\pi\epsilon}{c^2}\frac{nI_0R^2\omega}{r}\sin\omega t$.
\end{solution}
\end{problem}
%---------------------------------------------------------
\begin{figure}[h!]\centering
		\begin{tikzpicture}
			\draw[pattern=north west lines,pattern color=red!50, draw=red!50] (0,0) circle (1.5);
			\draw[gray, fill=gray!50] (5:1.5) arc (5:355:1.5) --
			(356:2) arc (356:4:2) -- cycle;
			\node at (90:1.7 5) {$\epsilon$};
			\draw[-latex] (0,0) -- node[below] {$r$} (0:1.75);
			\draw[-latex] (0,0) -- node[above=2pt] {$R$} (45:1.5);
		\end{tikzpicture}
	\caption{До задачі~\ref{prb:solenod_in_dielectric}}
	\label{solenod_in_dielectric}
\end{figure}
%--------------------------------------------------------- 


%%=========================================================
%\begin{problem}
%По двом паралельними нескінченним площинам, відстань між якими дорівнює $d$ течуть однакові по модулю протилежно напрямлені струми з поверхневою густиною, яка змінюється в часі за законом $i(t) = a t$, де $a$~-- позитивна константа. Знайти вихрове електричне поле.
%\begin{solution}
%	$
%		E_y =
%		\begin{cases}
%			\frac{4\pi}{c^2}a x, \quad  x < \nfrac{d}{2}, \\
%			\frac{4\pi}{c^2}a d, \quad  x \ge \nfrac{d}{2}.
%		\end{cases}
%	$
%	(Вісь $Ox$ напрямлена перпендикулярно площинам, вісь $Oy$~-- вздовж площин, вісь $Oz$ співпадає з напрямком магнітного поля.)
%\end{solution}
%\end{problem}

%=========================================================
\begin{problem}
У довгому соленоїді радіусом $R$ з густиною намотування $n$ змінюють струм з постійною швидкістю $\dot{I}$. Знайти модуль напруженості вихрового електричного поля $E(r)$ як функцію відстані $r$ від осі соленоїда.
\begin{solution}
	$
		\Efield =
		\begin{cases}
			- \frac{2\pi}{c^2} n\dot{I} r\vect{e}_{\phi},               & r < R, \\
			- \frac{2\pi}{c^2} n a^2\dot{I} \frac{1}{r}\vect{e}_{\phi}, & r > R.
		\end{cases}
	$
\end{solution}
\end{problem}

%=========================================================
\begin{problem}
Однорідно заряджений з об'ємною густиною заряду $\rho$ нескінченний циліндр радіусом $R$ обертається навколо своєї осі кутовою швидкістю, яка залежить від часу за законом $\omega = kt$. Знайти електричне та магнітне поле у всьому просторі в залежності від часу.
\begin{solution}
	$
		\Bfield =
		\begin{cases}
			\frac{2\pi\rho\omega}{c}(R^2 - r^2)\vect{e}_z, & r< R,    \\
			0,                                             & r \ge R,
		\end{cases}
	$

	$
		\Efield =
		\begin{cases}
			\frac{\pi k \rho r\omega}{2c^2}(2R^2 - r^2)\vect{e}_{\phi} + 2\pi\rho r \vect{e}_r, & r< R,    \\
			\frac{\pi k \rho r R^4}{2c^2r}\vect{e}_{\phi} + \frac{2\pi \rho R^2}{r}\vect{e}_r,  & r \ge R.
		\end{cases}
	$
\end{solution}
\end{problem}

\subsection*{Струм зміщення}

\begin{Theory}
	Закон Ампера з добавкою Максвелла:
	\begin{equation}
		\oint\limits_L \Hfield \cdot d\vect{r} = \frac{4\pi}{c} \iint\limits_S \vect{j}\cdot d\vect{S} +  \frac1c\iint\limits_S \frac{\partial \Dfield}{\partial t}\cdot d\vect{S}.
	\end{equation}
	Фізична сутність цього закону полягає в тому, що причиною виникнення магнітного поля є як струми провідності, густина яких характеризується величиною $\vect{j}$, так і змінне в часі електричне поле. Історично, змінне в часі електричне поле характеризують так званою густиною струму зміщення, яку визначають як:
	\begin{equation}
		\vect{j}_\text{зм} = \frac{1}{4\pi} \frac{\partial \Dfield}{\partial t}.
	\end{equation}
\end{Theory}

%=========================================================
\begin{problem}
В необмежене однорідне провідне середовище вміщена металева куля, якій наданий електричний заряд. Оскільки середовище провідне, то за рахунок стікання заряду з кулі, з'являться електричні струми. Показати, що в цьому випадку магнітного поля не виникатиме.
\end{problem}

%=========================================================
\begin{problem}
Плоский конденсатор складається з двох однакових металевих дисків, простір між якими заповнено однорідним діелектриком що має слабку провідність. Спочатку конденсатор було під'єднано до джерела постійної напрури. Нехтуючи крайовими ефектами, покажіть, що магнітне поле в просторі між обкладками конденсатора буде відсутнє.
\end{problem}

%=========================================================
\begin{problem}\label{prb:Griffiths7.34}
По дроту радіусом $R$, тече постійний струм $I$, який рівномірно розподілений по його поперечному перерізу. Вузький розрив в дроті шириною $d$ ($d \ll R$), утворює конденсатор з паралельними пластинами. Знайдіть магнітне поле в розриві, на відстані $s < R$ від осі.
\begin{solution}
	$B = \frac{2I}{cR^2}s$.
\end{solution}
\end{problem}


%=========================================================
\begin{problem}\label{prb:Bfield_in_condensator}
Плоский конденсатор складається з двох однакових металевих дисків, простір між якими заповнено однорідним діелектриком з діелектричною проникністю $\epsilon$. Відстань між внутрішніми поверхнями дисків дорівнює $d$. Між обкладками конденсатора підтримується змінна напруга $V = V_0\sin\omega t$. Нехтуючи крайовими ефектами, знайти магнітне поле в просторі між обкладками конденсатора.
\begin{solution}
	$B = \frac{\epsilon\omega r}{2cd}V_0\cos\omega t$, де $r$~-- відстань від осі конденсатора.
\end{solution}
 \end{problem}

%=========================================================
\begin{problem}\label{prb:Bfield_in_condensator_with_currnet}
    Для умов задачі~\ref{prb:Bfield_in_condensator}, визначте, чому дорівнює  магнітне поле в просторі між обкладками конденсатора, якщо діелектрик має провідність $\lambda$.
\begin{solution}
	$B = \frac{2\pi\lambda r V_0}{cd}\left( \sin\omega t + \frac{\epsilon \omega}{4\pi\lambda}\cos\omega t\right) $.
\end{solution}
\end{problem}

\subsection*{Закони збереження. Вектор Пойнтінга}

\begin{Theory}
Рівняння неперервності струму (закон збереження електричного заряду).
	\begin{equation} \label{charge_conservation_law}
		\frac{\partial \rho}{\partial t} + \divg\vect{j} = 0
	\end{equation}

Вектор Пойнтінга (вектор густини потоку енергії електромагнітного поля):
	\begin{equation}
			\vect{S} = \frac{c}{4\pi} \Efield\times\Hfield.
	\end{equation}

Густина імпульсу електромагнітного поля:
\begin{equation}
	\vect{\mathfrak{P}} = \frac{\vect{S}}{c^2}.
\end{equation}

Рівняння неперервності для енергії електромагнітного поля:
	\begin{equation} 
		\frac{\partial w}{\partial t} + \divg\vect{S} = -  \vect{j}\cdot\Efield,
	\end{equation}
$ w $~-- густина енергії електромагнітного поля, визначається формулою:
	\begin{equation} 
		w = \frac{1}{8\pi} (\Efield\cdot\Dfield + \Hfield\cdot\Bfield),
	\end{equation}
величина
	\begin{equation} 
		p =\vect{j}\cdot\Efield ,
	\end{equation}
є потужністю, що виділяється в одиниці об'єму речовини (закон Джоуля-Ленца).
\end{Theory}


%=========================================================
\begin{problem}
    Довести, що з рівнянь Максвелла виплаває закон збереження електричного заряду~\ref{charge_conservation_law}.
\end{problem}

%=========================================================
\begin{problem}
За яких умов електричний струм, об'ємна густина якого змінюється в просторі за законом  $\vect{j}(\vect{r}) = \vect{j}_0\cos(\vect{k}\cdot\vect{r})$, де $\vect{j}_0$, $\vect{k}$~-- постійні вектори, забезпечити стаціонарний розподіл зарядів у просторі?
\begin{solution}
	Це можливо за умови $\vect{j}_0\cdot\vect{k} = 0$.
\end{solution}
\end{problem}

%=========================================================
\begin{problem}
    Електричний заряд знаходиться в центрі кругового витка, по якому тече струм. Покажіть схематично напрямок вектора Пойнтінга в довільних точках простору навколо системи.
\end{problem}

%=========================================================
\begin{problem}
    Частинка, що має заряд $q$ рухається у вакуумі із постійним вектором швидкості $\vect{v} \ll c$. Знайдіть вектор Пойнтінга електромагнітного поля частинки.
\begin{solution}
	$\vect{S} = \frac{q^2}{4\pi r^5} \left( \vect{v} \cdot \vect{r}\right)$ .
\end{solution}
\end{problem}

%=========================================================
\begin{problem}
    Точковий диполь, що має дипольний момент $\vect{p}$ рухається у вакуумі із постійним вектором швидкості $\vect{v} \ll c$. Знайдіть вектор Пойнтінга електромагнітного поля диполя (в дипольному наближенні). Розгляньте випадки коли 
	\begin{enumerate*}[label=\alph*)]
		\item 	$\vect{p}_e \parallel \vect{v}$ та
		\item   $\vect{p}_e \perp \vect{v}$. 
	\end{enumerate*} 

	Вказівка: \emph{Скористайтесь відповіддю до задачі \ref{Bfield_of_electric_dipole}.}
\begin{solution}
	Вектор Пойнтінга в дипольному наближенні $\vect{S} = \frac{1}{4\pi} \left( \Efield^2 \vect{v} - \Efield\left(\Efield\cdot\vect{v} \right) \right) $. 
\end{solution}
\end{problem}

%=========================================================
\begin{problem}% Іродов 3.251
По прямому провіднику круглого перерізу тече постійний струм~$I$. Знайти потік вектора Пойнтінга через бічну поверхню даного провідника, що має опір $R$.
\begin{solution}
	$\Phi = I^2R$
\end{solution}
\end{problem}

%=========================================================
\begin{problem}
    Довгий соленоїд радіусом $r$, кількість витків якого  $N$ під'єднується до джерела постійної ЕРС $\EMF$ через опір $R$ (опором самого соленоїда можна знехтувати). Знайти електромагнітну енергію, яка зайшла до соленоїду в процесі встановлення струму, і порівняти її з магнітною енергією соленоїда.
\begin{solution}
	W = $\frac{2\pi^2}{c^2} \frac{N^2\EMF^2}{lR^2} = \frac{1}{2c^2} LI^2$.
\end{solution}
\end{problem}

%=========================================================
\begin{problem}\label{prb:charging_condensator}
Плоский конденсатор складається з двох круглих пластинок, які знаходяться на відстані $d$  одна від одної, кожна  радіус $R$. Через конденсатор тече постійний струм $I$, як вказано на рис.~\ref{charging_condensator}, що заряджає конденсатор. Як виглядають силові лінії електричного та магнітного поля в конденсаторі? Знайдіть вектор Пойнтінга на відстані $r$ від центра конденсатора. Показати, що потік вектора Пойнтінга через бічну поверхню конденсатора дорівнює збільшенню його енергії за одиницю часу. Розсіюванням поля на краях при розрахунку знехтувати.
\end{problem}

%=========================================================
\begin{problem}\label{prb:field_momentum}
    Між пластинами плоского конденсатора діє однорідне магнітне поле напруженості індукцією $\Bfield$, паралельне його пластинам (рис.~\ref{field_momentum}). Пластини конденсатора заряджають до заряду $q$, а потім їх з'єднують провідником, що має значний опір. Доведіть, що імпульс системи має зберігатись. 
\end{problem}

%=========================================================
\begin{figure}[h!]\centering
%---------------------------------------------------------
\begin{minipage}[t]{0.45\linewidth}\centering
	\begin{tikzpicture}
		\fill[red!50, draw=black] (-2,0.05) -- (-2,0) arc (180:360:2 and 0.1) -- +(0,0.05) arc (0:-180:2 and 0.1) arc (180:0:2 and 0.1) arc (0:-180:2 and 0.1);
		\fill[red!50, draw=black, yshift=-1cm] (-2,0.05) -- (-2,0) arc (180:360:2 and 0.1) -- +(0,0.05) arc (0:-180:2 and 0.1) arc (180:0:2 and 0.1) arc (0:-180:2 and 0.1);
		\draw [] (0,0) to [current direction={info={$I$}}] (0,1);
		\draw [yshift=-2.1cm] (0,0) to [current direction={info={$I$}}] (0,1);
		\draw [dashed] (0,-0.4) -- node[below] {$r$} (-1,-0.4) node[contact] {};
		\draw[dash dot] (0,-0.95) -- +(0,0.85);
	\end{tikzpicture}
	\caption{До задачі~\ref{prb:charging_condensator}}
	\label{charging_condensator}
\end{minipage}
%---------------------------------------------------------
\begin{minipage}[t]{0.45\linewidth}\centering
	\begin{tikzpicture}
		\fill[red!50, draw=black, yshift=-1cm] (-2,0.05) -- (-2,0) arc (180:360:2 and 0.1) -- +(0,0.05) arc (0:-180:2 and 0.1) arc (180:0:2 and 0.1) arc (0:-180:2 and 0.1);
		\draw[thick] (0,0) -- +(0,-0.95);
		\fill[red!50, draw=black] (-2,0.05) -- (-2,0) arc (180:360:2 and 0.1) -- +(0,0.05) arc (0:-180:2 and 0.1) arc (180:0:2 and 0.1) arc (0:-180:2 and 0.1);
		\foreach \i in {-0.2,-0.4,-0.6,-0.8} {
		\draw[-latex', blue] (-2,\i) -- +(4,0);
		}
		\node[right] at (2,-0.5) {$\Bfield$};
		\node[left] at (-2,0) {$+q$};
		\node[left] at (-2,-1) {$-q$};
	\end{tikzpicture}
	\caption{До задачі~\ref{prb:field_momentum}}
	\label{field_momentum}
\end{minipage}
%---------------------------------------------------------
\end{figure}
%=========================================================

%=========================================================
\begin{problem}%Белонучкин 2.131
Циліндричний нерелятивістський електронний пучок радіуса $R$ поширюється у вільному просторі. Електрони пучка летять паралельно, їхня енергія дорівнює  $W$, а концентрація $n$. Знайти величину і напрямок вектора Пойнтінга в будь-якій точці простору (зовні та всередині пучка).
\end{problem}

%=========================================================
\begin{problem}
    По двопровідній лінії з радіус провідників $R$, які знаходяться на відстані $d$ один від одного, передається потужність $P$ при постійній напрузі. Нехтуючи опором провідників, визначити як змінюється вектор Пойнтінга в площині між провідниками. 
\begin{solution}
	$S  = \frac{P}{4\pi\ln\left( \frac{d}{R} \right) } \frac{d^2}{x^2\left( x - d\right)^2 }$, де $x$~-- відстань від точки до осі одного з провідників.
\end{solution}
\end{problem}


%=========================================================
\begin{problem}\label{prb:KRS3.191}
Заряджений плоский повітряний конденсатор з напруженістю електричного поля між пластинами $282$~В/см поміщений всередині соленоїда, поперечний переріз якого має форму прямокутника зі сторонами, паралельними і перпендикулярними пластинам конденсатора. У колі обмотки соленоїда є батарея постійного струму і ключ. Вся система (разом з батареєю) поміщена на горизонтальних рейках, які паралельні пластинам конденсатора. Система може переміщатися по рейкам без тертя. Спочатку коло соленоїда розімкнуте. Потім ключ замикається і в соленоїді створюється постійне магнітне поле з індукцією $2000$~Гс. Знайти зміну механічного імпульсу системи після замикання ключа. Об'єм повітряного простору між пластинами конденсатора дорівнює $200$~см$^3$.
\begin{solution}
	$\Delta p = \frac{EB}{2\pi c}V = 10^{-6}$~г$\cdot$см/с.
\end{solution}
\end{problem}


%=========================================================
\begin{problem}[Аналог парадоксу диска Фейнмана~\cite{FLF6}]
    Чи порушується закон збереження моменту імпульсу в задачі~\ref{prb:aka_Faynmann_disk_paradox}.
\end{problem}

\Closesolutionfile{answer}

