% !TeX program = lualatex
% !TeX encoding = utf8
% !TeX spellcheck = uk_UA
% !TeX root =../EMProblems.tex

\introtrue
\chapter*{Рекомендації щодо розв'язку задач}

Як відомо, фізична теорія описує певне коло явищ природи і має свої межі застосування. Для чіткого усвідомлення основних аспектів теорії, меж її застосування та пояснення на її основі заданих явищ, необхідно тренуватись в  розв'язуванні задач. 

В основу кожної задачі покладено деяке фізичне явище, яке описуються одним або кількома фундаментальними законами. Тому, першою рекомендацією для розв'язку задач буде наступна:
\begin{itemize}
\item ретельно опрацюйте теорію.
\end{itemize}

Однак, саме по собі знання теорії ще не гарантує успіху в досягненні розв'язку. Як правило, явище, що описується в задачі, на перший погляд може здатись складним, таким, що залежить від багатьох факторів, тому крім знання основ теорії слід ще визначити, які з цих факторів мають суттєве значення, а які другорядні. Саме визначення таких факторів є процесом побудови фізичної моделі. Тому наступною рекомендацією буде:
\begin{itemize}
\item побудуйте фізичну модель явища.
\end{itemize}

Енциклопедія дає наступне означення поняттю фізичної моделі: \emph{фізична модель~--- це фізичне уявлення про явище з метою його дослідження, яке представляється за допомогою іншого фізичного явища, що має в тому чи іншому аспекті аналогічну динаміку поведінки.}

Крім того, навіть якщо фізична ситуація є зрозумілою, вона може бути складною в математичному відношенні, тому вдало побудована фізична модель може значно спростити математичні розрахунки.

Наступним важливим етапом розв'язку задачі є: 
\begin{itemize}
\item схематичне креслення, яке робить умову задачі більш наочною і полегшує її розв’язання. 
\end{itemize}

Схематичне креслення є важливим евристичним методом, що полегшує побудову та аналіз фізичної моделі. Побудувавши таку модель і усвідомивши межі її застосовності
\begin{itemize}
\item слід записати математичні співвідношення, між шуканими та відомими величинами у вигляді системи рівнянь.
\end{itemize}

Такі рівняння складають основу математичної моделі фізичного явища. Перш ніж розв'язувати складену систему, 
\begin{itemize}
\item переконайтеся в тому, що число невідомих дорівнює числу рівнянь.
\end{itemize}
 
Після того, як отриманий результат у вигляді математичного співвідношення (формули), перш за все, 
\begin{itemize}
\item перевірте розмірність результату.
\end{itemize} 

Проте правильна розмірність ще не гарантує правильного результату, окрім цього
\begin{itemize}
\item розгляньте граничні випадки в отриманих формулах, переконайтесь, що вони дають фізично адекватні результати.
\end{itemize}

Лише після всього вище сказаного, проведіть арифметичні розрахунки, які слід проводити за правилами наближених обчислень. Отримавши відповідь, за можливості, необхідно 
\begin{itemize}
\item оцінити, наскільки числовий результат відповідає дійсності.
\end{itemize}

Іноді така оцінка дозволяє встановити помилковість отриманого результату.

\introfalse





