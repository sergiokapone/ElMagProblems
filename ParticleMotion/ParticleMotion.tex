% !TeX program = lualatex
% !TeX encoding = utf8
% !TeX spellcheck = uk_UA
% !TeX root =../EMProblems.tex

%=========================================================
\Opensolutionfile{answer}[\currfilebase/\currfilebase-Answers]
\Writetofile{answer}{\protect\section*{\nameref*{\currfilebase}}}
\chapter{Рух частинок в електромагнітному полі}\label{\currfilebase}
%=========================================================

\section{Рух заряду електричному полі}

\begin{Theory}\small
Сила, що діє на заряджену частинку в електромагнітному полі (сила Лоренца):
	\begin{equation}
		\vect{F} = q\Efield + q\left[ \frac{\vect{v}}{c} \times \Bfield\right].
	\end{equation}
\end{Theory}

%=========================================================
\begin{problem}
Однорідно заряджене кільце радіуса $R$ має заряд $Q$. Покажіть, що точковий заряд $-q$ маси $m$, який знаходиться на осі кільця буде здійснювати гармонічні коливання при невеликому зміщенні вздовж осі відносно центра та знайдіть частоту цих коливань.
\begin{solution}
	$\omega = \sqrt{\frac{qQ}{mR^3}}$.
\end{solution}
\end{problem}

%=========================================================
\begin{problem}
    Визначте кут розсіювання для частинки масою $m$ з електричним зарядом $q$, що рухається в електричному полі нерухомого центру, що має заряд $Q$. Швидкість частинки до розсіювання $v_0$ і значення прицільного
    параметра $b$ вважати відомими.
\begin{solution}
	$\tg\theta2 = \frac qm\frac{Q}{v_0^2b}$
\end{solution}
\end{problem}

%=========================================================
\begin{problem}
Частинка з питомим зарядом $q/m$ рухається  прямолінійно під дією електричного поля $E = E_0 - kx$, де $k$~--- додатна постійна, $x$~-- відстань від точки, в частинка знаходилась в стані спокою в початковий момент. Знайти відстань, яку пройде частинка до зупинки.
\begin{solution}
	$x = \frac{2E_0}{k}$.
\end{solution}
\end{problem}

%=========================================================
\begin{problem}
Знайти траєкторію релятивістської частини що має зарядом $q$, початкову кінетичну енергію $\mathcal{E}_0$ і початковий імпульсом $p_0$, напрямлений вздовж осі $Ox$
в постійному однорідному електричному полі $E$, що протилежно осі $Oy$. Знайти траєкторію частинки.
\begin{solution}
	$x = \frac{p_0c}{qE} \mathrm{arcsh}\frac{cqEt}{\mathcal{E}_0}$, $y = \frac{1}{qE}\sqrt{\mathcal{E}_0^2 + (qcEt)^2} - \frac{\mathcal{E}_0}{qE}$. Траєкторія частинки $y = \frac{\mathcal{E}_0}{qE} \left( \mathrm{ch}\frac{eEx}{cp_0} - 1\right) $.
\end{solution}
\end{problem}

\section{Рух заряду в магнітному полі}

%=========================================================
\begin{problem}
Релятивістська частинка зарядом $q$ рухається в постійному однорідному магнітному полі $B$. Знайти закони її руху, а також радіус і частоту обертання.
\end{problem}

%=========================================================
\begin{problem}
Конічний маятник складається з кульки питомим зарядом $q/m$,  яка підвішена на довгій нитці. Як зміниться кутова швидкість обертання маятника після увімкнення вертикального магнітного поля $\Bfield$?
В якому випадку кутова швидкість збільшиться і в якому зменшиться?
\begin{solution}
	$\Delta\vect{\omega} = \pm\frac{qB}{2mc}$. Частота $\Delta\vect{\omega}>0$, якщо $\Bfield \uparrow\downarrow \vect{\omega}$,і
	$\Delta\vect{\omega}>0$, якщо $\Bfield \uparrow\uparrow \vect{\omega}$.
\end{solution}
\end{problem}

%=========================================================
\begin{problem}
З поверхні циліндричного провідника радіусом $R$, по якому тече постійний струм $I$, вилітає електрон з початковою швидкістю $v_0$, перпендикулярно до поверхні провідника. На яку максимальну відстань віддалиться електрон від осі провідника, перш ніж повернути назад під дією магнітного поля струму?
\begin{solution}
	$l_{\max} = Re^{\frac{mv_0}{2cqI}}$.
\end{solution}
\end{problem}

%=========================================================
\begin{problem}[Градієнтний дрейф частинки в неоднорідному магнітному полі]
    Частинка, питомим зарядом $q/m$ залітає в область сильного магнітного поля, перпендикулярно до силових ліній з швидкістю $v$. Магнітне поле напрямлене вздовж осі $Oz$ і слабо змінюється вздовж осі $Ox$ за законом $B_z = B_0 + Ax$, де $A$~--- позитивна константа. В цьому випадку виникає дрейф вздовж осі $y$. Знайдіть швидкість градієнтного дрейфу. Намалюйте схематично траєкторію руху частинки.
\begin{solution}
	$v_G = \frac{m}{q}\frac{cv^2A}{2B^2}.$
\end{solution}
\end{problem}



\section{Рух заряду в електричному та магнітному полях}


%=========================================================
\begin{problem}
    Визначте кут $\theta$ розсіювання для частинки масою $m$ з електричним зарядом $q$, що рухається в електричному полі нерухомого розсіювального центру, що має заряд $Q$. Швидкість частинки до розсіювання $v_0$ і значення прицільного параметра $b$ вважати відомими.
\begin{solution}
	$\tg\frac{\theta}{2} = \frac qm \frac{Q}{bv_0^2}$.
\end{solution}
\end{problem}

%=========================================================
\begin{problem}\label{MotioninEM}
Розглянемо частинку питомим зарядом $q/m$, що рухається в присутності постійних і однорідних електромагнітних полів, заданих у вигляді $\Efield = E_0\vect{e}_y$ та $\Bfield = B_0\vect{e}_z$. В початковий момент частинка знаходиться в стані спокою у початку координат. Знайти закони руху частинки. Зобразіть траєкторію частинки в площині $xOy$ для випадків $q < 0$ та  $q> 0$.
\begin{solution}
	$x(t)  = \frac{E_0}{B_0}c\left( t - \frac{1}{\Omega}\sin\Omega t \right) $,
	$y(t)  = \frac{E_0}{B_0} \frac{c}{\Omega}\left( 1 - \cos\Omega t \right) $, де $\Omega = \frac{qB_0}{mc}$~--- циклотронна частота.
\end{solution}
\end{problem}

%=========================================================
\begin{problem}
При якій початковій швидкості зарядженої частинки її траєкторія прямолінійна при русі в ортогональних електричних і магнітних полях?
\begin{solution}
	$v_0 = c\frac{E_0}{B_0}$.
\end{solution}
\end{problem}


%=========================================================
\begin{problem}
Розв'язати задачу~\ref{MotioninEM} для випадку, якщо електричне поле змінюється за законом $\Efield = E_0\cos\Omega t\,\vect{e}_y$,  де $\Omega = \frac{qB_0}{mc}$~--- циклотронна частота.
\begin{solution}
	$x(t)  = \frac{qE_0}{2m\Omega^2}\left( \sin\Omega t - \Omega t\cos\Omega t \right) $,
	$y(t)  = \frac{qE_0}{2m\Omega}t\sin\Omega t$.
\end{solution}
\end{problem}


%=========================================================
\begin{problem}
Розглянемо пучок іонів питомим зарядом $q/m$, що рухається влітають в область постійних і однорідних паралельних електромагнітних полів, заданих у вигляді $\Efield = E_0\vect{e}_y$ та $\Bfield = B_0\vect{e}_y$. з швидкістю $\vect{v} = v_0 \vect{e}_x$. На відстані $l$ від точки початку координат знаходиться плоский екран, орієнтований перпендикулярно осі $Ox$. Знайти рівняння сліду іонів на екрані. Показати, що при $z \ll l$ слід матиме вигляд параболи.
\begin{solution}
	$z = l\tg\sqrt{\frac{qB_0^2}{2mc^2E_0}y}$.
\end{solution}
\end{problem}


%=========================================================
\begin{problem}
Частинка з питомим зарядом $q/m$ знаходиться всередині соленоїда круглого перерізу на відстані $r$ від його осі. В обмотці увімкнули струм, і індукція магнітного поля стала рівною $B_0$. Знайти швидкість частинки і радіус кривизни її траєкторії, якщо за час наростання струму в соленоїді її зміщенням можна знехтувати.
\begin{solution}
	$v = \frac{rqB_0}{2mc}$, $\rho = \frac{r}{2}$.
\end{solution}
\end{problem}

%=========================================================
\begin{problem}[Рух заряду в полі монополя та діона]
\parindent=1em%
    Частинка, яка створює радіальне магнітне поле вигляду $\Bfield = \frac{G\vect{r}}{r^3}$ називається \emph{магнітним монополем}, а $G$~--- магнітний заряд цього монополя, якщо крім магнітного заряду,  частинка має ще й електричний заряд $Q$, то вона називається \emph{діоном}.

	Від'ємний заряд $q$ масою $m$ запустили таким чином, що він рухається в поле діона по коловій орбіті з постійною за модулем швидкістю $v \ll c$. Знайдіть радіус цієї орбіти, та відстань площини орбіти до діона.
	
	Вкажіть величини, що зберігаються при русі заряду $q$, запишіть їх для розглядуваної ситуації. Вказівка: \emph{Скористайтесь виразом $\frac{d}{dt}\left( \frac{\vect{r}}{r}\right)  = \frac{\vect{v}}{r} - \frac{\vect{r}(\vect{v}\cdot\vect{r})}{r^3}$}.
\begin{solution}
	Радіус орбіти $R = \frac{qQc}{mv^2}\sin\theta$, відстань площини орбіти до діона $d = \frac{qQc}{mv^2}\cos\theta$, де $\tg\theta = \frac{Q c}{G v}$.

	Зберігається енергія $E = \frac{mv^2}{2} - \frac{Qq}{r}$, та вектор $\vect{J} = \vect{L} - \frac{Gq}{c}\frac{\vect{r}}{r} $. Для умов нашої задачі $E = -\frac{mv^2}{2}$, а $\vect{J} = (mvR - \frac{Gq}{c}\cos\theta) \vect{n} $, де $\vect{n}$~--- вектор нормалі до траєкторії.
\end{solution}
\end{problem}



\subsection*{Прискорення частинок в установках}

\begin{Theory}
\href{http://nuclphys.sinp.msu.ru/experiment/accelerators/betatron.htm}{Бетатрон}~--- циклічний  прискорювач електронів з фіксованою рівноважною коловою орбітою, прискорення в якому відбувається за допомогою вихрового електричного поля. Магнітне поле при цьому перпендикулярне до орбіти електронів і змінюється з часом.

Для того, щоб електрони в бетатроні рухались по рівноважній коловій орбіті радіуса $R$, необхідно, щоб виконувалась \emph{бетатронна умова}:
\begin{equation}\label{Betatron_equation}
	\left\langle B(t)\right\rangle  = 2 B(R,t),
\end{equation}
де $\left\langle B(t)\right\rangle$~--- середнє значення індукції магнітного поля по площі, що обмежена орбітою,  $B(R,t)$~--- значення індукції магнітного поля в точках на рівноважній орбіті електронів. 

\href{http://nuclphys.sinp.msu.ru/experiment/accelerators/ciclotron.htm}{Циклотрон}~--- це прискорювач заряджених частинок (протонів, іонів), в якому частинки  утримуються на спіральній траєкторії в постійному  однорідному магнітному полі і прискорюються швидкозмінним (радіочастотним) електричним полем.
\end{Theory}

%=========================================================
\begin{problem}
   Виведіть бетатронну умову~\ref{Betatron_equation}. Покажіть за допомогою бетатронної умови, що напруженість вихрового електричного поля в бетатроні має екстремум на рівноважній орбіті.
\end{problem}

%=========================================================
\begin{problem}
    За допомогою бетатронної умови~\ref{Betatron_equation} знайдіть радіус $R$ колової траєкторії електрона, якщо індукція магнітного поля змінюється з відстанню $r$ від центра бетатрона за умовою:
\[
	B = B_0 - kr^2,
\]
де $B_0$ та $k$~--- додатні постійні.
\begin{solution}
	$R = \sqrt{\frac{2B_0}{3a}}.$
\end{solution}
\end{problem}

\begin{problem}\label{prb:mass_spectrometer}%Журнал Квант, 1978 № 8, Ф486, #45
На рис.~\ref{mass_spectrometer} зображена схема мас-спектрометра. У іонізаторі $I$ утворюються іони, які прискорюються напругою $V = 10$~кВ і входять через щілину в магнітне поле з індукцією $B = 0.1$~Т. Після повороту іони потрапляють на фотографічну пластинку $P$ і викликають її почорніння. На якій відстані, один від одного будуть перебувати на платівці смуги іонів \ce{H^+}, \ce{{}^2H^+},\ce{{}^3H^+}, \ce{He^+}? Якою повинна бути ширина щілини, щоб смуги іонів \ce{{}^16O+} та \ce{{}^15N^+} можна було розділити? 
\begin{solution}
	$\Delta l = \frac{2}{B}\sqrt{\frac{2V}{q}} \left( \sqrt{m_2} - \sqrt{m_1} \right) $.
\end{solution}
\end{problem}
%---------------------------------------------------------
\begin{figure}[h!]\centering
    \begin{tikzpicture}
        \fill[red, draw=black] (0,0.1) rectangle +(1,0.1) (1.1,0.1) rectangle ++(1,0.1) ++(0,-0.05) -- +(0.5,0) node[contact] {};
		\fill[red, draw=black] (0,-0.1) rectangle +(1,0.1) (1.1,-0.1) rectangle ++(1,0.1) ++(0,-0.05) -- +(0.5,0) node[contact] {};
		\node[right] at (2.5,0.05) {$V$};

		\fill[red] (0.8,-0.23) rectangle ++(0.5,-0.47) ;
		\draw[ultra thick] (0.8,-0.2) -- ++(0,-0.5) -- node[below] {$I$} ++(0.5,0) -- ++(0,0.5);

		\draw[red] (1.05, 0.1) -- +(0,-0.5);
		\foreach \i in {1.5,2,2.5,3} {%
		\draw[red] (1.05, 0.1) arc (180:0:\i);
		}
		\draw (3.5,0) rectangle +(4,0.1) node[pos=0.5, below] {$P$};
		\foreach \i in {0,...,7} {
		\foreach \j in {0,...,4} {
		\fill[blue] (\i+0.05,\j+0.4) circle (0.05); \draw[blue] (\i+0.05,\j +0.4) circle (0.1);
		}
		}
    \end{tikzpicture}
\caption{До задачі~\ref{prb:mass_spectrometer}}
\label{mass_spectrometer}
\end{figure}
%---------------------------------------------------------

%=========================================================
\begin{problem}
Магнетрон складається з довгого циліндричного анода радіусом $a$ і коаксіального з ним циліндричного катода радіусом $b$ ($b < a $). На осі системи є нитка з струмом розжарення $I$,яка створює в навколишньому просторі магнітне поле. Знайти найменшу різницю потенціалів між катодом і анодом, при якій термоелектрони, що залишають катод без початкової швидкості, почнуть досягати анода.
\begin{solution}
	$V = 2\frac{e}{m}\frac{I^2}{c^2}\ln\frac{a}{b}$.
\end{solution}
\end{problem}


%=========================================================
\begin{problem}
Магнетрон що складається з нитки розжарення радіусом $a$ і коаксіального циліндричного анода радіусом $b$, які знаходяться в однорідному магнітному полі, паралельному нитці. Між ниткою і анодом прикладена прискорююча різниця потенціалів $V$. Знайти значення індукції магнітного
поля, при якому електрони, що вилітають з нульовою початковою швидкістю з нитки, будуть досягати анода.
\begin{solution}
	$B \le \sqrt{\frac{8mV}{e}} \frac{b}{b^2 - a^2}$.
\end{solution}
\end{problem}

\Closesolutionfile{answer}

