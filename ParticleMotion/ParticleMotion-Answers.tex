\protect \section *{\nameref *{ParticleMotion}}
\begin{Solution}{5.{1}}
	$\omega = \sqrt{\frac{qQ}{mR^3}}$.
\end{Solution}
\begin{Solution}{5.{3}}
    $15^\circ$.
\end{Solution}
\begin{Solution}{5.{4}}
    $100$~В/м.
\end{Solution}
\begin{Solution}{5.{5}}
    $v = \sqrt[3]{\frac{9kel}{2m}} = 1.6\cdot10^6$~см/с.
\end{Solution}
\begin{Solution}{5.{6}}
	$\tg\theta2 = \frac qm\frac{Q}{v_0^2b}$
\end{Solution}
\begin{Solution}{5.{7}}
	$x = \frac{2E_0}{k}$.
\end{Solution}
\begin{Solution}{5.{8}}
	$x = \frac{p_0c}{qE} \mathrm{arcsh}\frac{cqEt}{\mathcal{E}_0}$, $y = \frac{1}{qE}\sqrt{\mathcal{E}_0^2 + (qcEt)^2} - \frac{\mathcal{E}_0}{qE}$. Траєкторія частинки $y = \frac{\mathcal{E}_0}{qE} \left( \mathrm{ch}\frac{eEx}{cp_0} - 1\right) $.
\end{Solution}
\begin{Solution}{5.{9}}
    $d = 59$~мм.
\end{Solution}
\begin{Solution}{5.{12}}
    $T = 7$~нс.
\end{Solution}
\begin{Solution}{5.{13}}
	$\Delta\vect{\omega} = \pm\frac{qB}{2mc}$. Частота $\Delta\vect{\omega}>0$, якщо $\Bfield \uparrow\downarrow \vect{\omega}$,і
	$\Delta\vect{\omega}>0$, якщо $\Bfield \uparrow\uparrow \vect{\omega}$.
\end{Solution}
\begin{Solution}{5.{14}}
	$l_{\max} = Re^{\frac{mv_0}{2cqI}}$.
\end{Solution}
\begin{Solution}{5.{15}}
	$v_G = \frac{m}{q}\frac{cv^2A}{2B^2}.$
\end{Solution}
\begin{Solution}{5.{16}}
	$\tg\frac{\theta}{2} = \frac qm \frac{Q}{bv_0^2}$.
\end{Solution}
\begin{Solution}{5.{17}}
	$x(t)  = \frac{E_0}{B_0}c\left( t - \frac{1}{\Omega}\sin\Omega t \right) $,
	$y(t)  = \frac{E_0}{B_0} \frac{c}{\Omega}\left( 1 - \cos\Omega t \right) $, де $\Omega = \frac{qB_0}{mc}$~--- циклотронна частота.
\end{Solution}
\begin{Solution}{5.{18}}
	$v_0 = c\frac{E_0}{B_0}$.
\end{Solution}
\begin{Solution}{5.{19}}
	$x(t)  = \frac{qE_0}{2m\Omega^2}\left( \sin\Omega t - \Omega t\cos\Omega t \right) $,
	$y(t)  = \frac{qE_0}{2m\Omega}t\sin\Omega t$.
\end{Solution}
\begin{Solution}{5.{20}}
	$z = l\tg\sqrt{\frac{qB_0^2}{2mc^2E_0}y}$.
\end{Solution}
\begin{Solution}{5.{21}}
	$v = \frac{rqB_0}{2mc}$, $\rho = \frac{r}{2}$.
\end{Solution}
\begin{Solution}{5.{22}}
	Радіус орбіти $R = \frac{qQc}{mv^2}\sin\theta$, відстань площини орбіти до діона $d = \frac{qQc}{mv^2}\cos\theta$, де $\tg\theta = \frac{Q c}{G v}$.

	Зберігається енергія $E = \frac{mv^2}{2} - \frac{Qq}{r}$, та вектор $\vect{J} = \vect{L} - \frac{Gq}{c}\frac{\vect{r}}{r} $. Для умов нашої задачі $E = -\frac{mv^2}{2}$, а $\vect{J} = (mvR - \frac{Gq}{c}\cos\theta) \vect{n} $, де $\vect{n}$~--- вектор нормалі до траєкторії.
\end{Solution}
\begin{Solution}{5.{24}}
	$R = \sqrt{\frac{2B_0}{3a}}.$
\end{Solution}
\begin{Solution}{5.{25}}
	$\Delta l = \frac{2}{B}\sqrt{\frac{2V}{q}} \left( \sqrt{m_2} - \sqrt{m_1} \right) $.
\end{Solution}
\begin{Solution}{5.{26}}
	$V = 2\frac{e}{m}\frac{I^2}{c^2}\ln\frac{a}{b}$.
\end{Solution}
\begin{Solution}{5.{27}}
	$B \le \sqrt{\frac{8mV}{e}} \frac{b}{b^2 - a^2}$.
\end{Solution}
\begin{Solution}{5.{28}}
$R_H = \frac{3\pi}8 \rho (u_n + u_e) \approx 136$~м$^3$/Кл.
\end{Solution}
\begin{Solution}{5.{29}}
$n_q = \frac{3\pi B j b}{4U_H e} = 5.3 10^{16}$~м$^{}-3$.
\end{Solution}
