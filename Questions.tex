%%============================ Compiler Directives =======================%%
%%                                                                        %%
% !TeX program = pdflatex							    	
% !TeX encoding = utf8
% !TeX spellcheck = uk_UA
%%                                                                        %%
%%============================== Клас документа ==========================%%
%%                                                                        %%
\documentclass[12pt]{article}
%%                                                                        %%
%%============================= Мови та кодування ========================%%
%%                                                                        %%
\usepackage[utf8]{inputenc}
\usepackage[T2A,T1]{fontenc}
\usepackage[english, russian, ukrainian]{babel}
%%                                                                        %%
%%=========================== Киририличні корекції =======================%%   
\usepackage{indentfirst}
\usepackage{cmap}
\IfFileExists{ukrcorr.sty}{\usepackage{ukrcorr}}{}
%%                                                                        %%
%%============================= Геометрія сторінки =======================%%
%%                                                                        %%        
\usepackage[%
	a4paper,%
	footskip=1cm,%
	headsep=0.3cm,% 
	top=2cm, %поле сверху
	bottom=2cm, %поле снизу
	left=2cm, %поле ліворуч
	right=2cm, %поле праворуч
    ]{geometry}
%%                                                                        %%
%%================================= Шрифти ===============================%%   
%%                                                                        %%
%\fontsize{12pt}{13pt}\selectfont                                         %%
%%============================== Інтерліньяж  ============================%%
%%                                                                        %%
\renewcommand{\baselinestretch}{1}
%-------------------------  Подавление висячих строк  --------------------%%
\clubpenalty =10000
\widowpenalty=10000
%---------------------------------Інтервали-------------------------------%%
\setlength{\parskip}{0.5ex}%
\setlength{\parindent}{2.5em}%
%%                                                                        %%
%%                                                                        %%
%%=========================== Математичні пакети і графіка ===============%%
%%                                                                        %%
\usepackage{amsmath}
\usepackage{graphicx}
\usepackage{floatflt}
%%                                                                        %%
%%================================ Інші пакети ===========================%%   
%%                                                                        %%
%%========================== Гіперпосилення (href) =======================%%
%%                                                                        %% 
\usepackage[%colorlinks=true,
	%urlcolor = blue, %Colour for external hyperlinks
	%linkcolor  = malina, %Colour of internal links
	%citecolor  = green, %Colour of citations
	bookmarks = true,
	bookmarksnumbered=true,
	unicode,
	linktoc = all,
	hypertexnames=false,
	pdftoolbar=false,
	pdfpagelayout=TwoPageRight,
	pdfauthor={Ponomarenko S.M. aka sergiokapone},
	pdfdisplaydoctitle=true,
	pdfencoding=auto
	]%
	{hyperref}
		\makeatletter
	\AtBeginDocument{
	\hypersetup{
		pdfinfo={
		Title={\@title},
		}
	}
	}
	\makeatother
%%                                                                        %%	
%%============================== Оформлення списків=======================%%
%%                                                                        %%
\usepackage{enumitem}
\setlist{nolistsep, leftmargin=0cm,itemindent=.5cm}
%%                                                                        %%
%%============================ Заголовок та автори =======================%%
%%                                                                        %%
\title{}
\author{}
\date{}                                   
%%                                                                        %%
%%========================================================================%%

\renewcommand{\baselinestretch}{1.5}
\begin{document}

\begin{center}\bfseries
Мінімальний набір понять для допуску до будь-якого виду занять
\end{center}

\begin{enumerate}
\item Запишіть рівняння Максвелла для випадку електростатики в диференціальній формі.
\item Запишіть рівняння Максвелла для випадку електростатики в інтегральній формі. Який фізичний смисл циркуляції вектора $\vec{E}$?
\item Запишіть рівняння Пуассона та Лапласа і їх розв'язки для острівного тіла.
\item Чому дорівнює електричне поле нескінченно зарядженої площини?
\item Чому дорівнює електричне поле в середині та зовні однорідної зарядженої кулі? Знайдіть потенціал в середині та зовні однорідної зарядженої кулі.
\item Що таке еквіпотенціальна поверхня? Покажіть, що силові лінії перпендикулярні до еквіпотенціальних поверхонь.
\item Запишіть означення дипольного моменту суцільного тіла. Запишіть вираз для потенціалу та поля електричного диполя.
\item Запишіть вираз для сили, що діє на диполь в електричному полі. Яка енергія жорсткого диполя в електричному полі?
\item Що таке поляризованість? Який її зв'язок з електричним полем? Що таке поляризовність та діелектрична проникність?
\item Запишіть граничні умови для поверхні розділу середовищ в електричному полі?
\item Запишіть означення (вираз) для вектора індукції електричного поля.
\item Що таке сторонні сили? Запишіть закони Ома та Джоуля-Ленца в диференціальній формі.
\item Запишіть закон Біо-Савара-Лапласа для об'ємного струму $\vec{j}$.
\item Доведіть, що два паралельних провідника по ким течуть струми взаємодіють. Який характер цієї взаємодії? Від чого це залежить?
\item Запишіть рівняння Максвелла для випадку магнітостатики в диференціальній формі. Як вводиться вектор-потенціал?
\item Як вводиться вектор-потенціал? Запишіть рівняння Пуассона для вектор-потенціалу та його розв'язок для острівного тіла.
\item Запишіть рівняння Максвелла для випадку магнітостатики в інтегральній формі. Який у них фізичний смисл?
\item Запишіть означення магнітного дипольного моменту суцільного тіла. Запишіть вираз для індукції магнітного поля  диполя.
\item Що таке вектор намагніченості? Що таке вектор напруженості магнітного поля? Який їх зв'язок? Що таке магнітна сприйнятливість?
\item Запишіть граничні умови для поверхні розділу середовищ в магнітному полі?
\item Запишіть вираз для сили, яка діє з боку електромагнітного поля на заряджену частинку.
\item Запишіть закон електромагнітної індукції у інтегральній формі.
\item Запишіть вирази для енергії магнітного та електричного полів. Запишіть вираз для вектора Пойнтінга та сформулюйте Теорему Пойнтінга.
\item Що таке змінний квазістаціонарний струм? Які характеристики має цей струм?
\item Що таке активний та реактивний опір елемента електричного кола (конденсатора, котушки, резистора)? Що таке комплексний імпеданс?
\item Явище резонансу напруг та струмів, пояснити їхній сенс та умови виникнення.
\end{enumerate}

\bigskip

\emph{Під відповіддю на питання розуміється не відтворення зазубреного матеріалу, а його розуміння. Для перевірки розуміння, додатково до питання буде задаватись мінізадача.}

\end{document}
