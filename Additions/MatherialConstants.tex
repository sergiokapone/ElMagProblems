% !TeX program = lualatex
% !TeX encoding = utf8
% !TeX spellcheck = uk_UA
% !TeX root =../EMProblems.tex

\newpage
\section{Електричні та магнітні характеристики речовини}

\begin{table}[h!]\centering
	\tabcaption{Відносні діелектричні проникності деяких речовин}
	\label{tab:Dielectric}
	\small
	\begin{tabular}{lc}
		\toprule
		Діелектрик & \epsilon \\ \midrule
		Вода       & 81       \\
		Гас        & 2.0      \\
		Парафін    & 2.0      \\
		Плексиглас & 3.5      \\
		Поліетилен & 2.3      \\
		Слюда      & 7.5      \\
		Спирт      & 26       \\
		Скло       & 6.0      \\
		Ебоніт     & 2.7      \\ \bottomrule
	\end{tabular}
\end{table}

\begin{table}[h!]\centering
	\tabcaption{Магнітна проникність деяких парамагнітних та діамагнітних речовин}
	\label{tab:Dielectric}
	\small
	\begin{tabular}{lclc}
		\toprule
		Парамагнетик          & $\mu$      & Діамагнетик           & $\mu$      \\ \midrule
		Азот (газоподібний)   & $1.000013$ & Водень (газоподібний) & $0.999937$ \\
		Повітря               & $1.000038$ & Графіт                & $0.999895$ \\
		Кисень (газоподібний) & $1.000017$ & Вода                  & $0.999991$ \\
		Кисень (рідкий)       & $1.0034$   & Мідь                  & $0.999912$ \\
		Ебоніт                & $1.000014$ & Скло                  & $0.999987$ \\
		Алюміній              & $1.000023$ & Цинк                  & $0.999991$ \\
		Вольфрам              & $1.000175$ & Золото                & $0.999963$ \\
		Платина               & $1.000253$ & Вісмут                & $0.999824$ \\ \bottomrule
	\end{tabular}
\end{table}
