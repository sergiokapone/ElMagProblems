% !TeX program = lualatex
% !TeX encoding = utf8
% !TeX spellcheck = uk_UA
% !TeX root =../EMProblems.tex
\newpage
\chapter{Система рівнянь Максвелла}
\epigraph{\Annabelle  Невозможно избавиться от ощущения, что эти математические формулы существуют независимо от нас и обладают собственным разумом, что они мудрее нас, мудрее даже тех, кто их открыл, и что мы извлекаем из них больше, чем первоначально было заложено}{Г.~Герц}
%\epigraph{\Annabelle Уж не Боги ли начертали эти знаки?}{Л.~Больцман}

\section{Рівняння Максвелла}

Система рівнянь Максвелла сформульована Дж.~К.~Максвеллом в 60-х роках XIX століття на основі узагальнення емпіричних законів і розвитку ідей про електромагнітні явища. Вони є основою теорії електромагнітного поля, ця система рівнянь дозволяє розв'язувати задачі, пов'язані з відшуканням електричних і магнітних полів, що створюються заданим розподілом електричних зарядів і струмів.

Інтегральна форма:
\begin{align}
	&\oiint\limits_S \Dfield\cdot d\vect{S} = 4\pi\iiint\limits_{V}\rho dV  && \text{\small Теорема Гауса для електричного поля} \label{Int I}\tag{Int I}\\
	&\oiint\limits_S \Bfield\cdot d\vect{S} = 0 && \text{\small Теорема Гауса для магнітного поля} \label{Int II}\tag{Int II}\\
	&\oint\limits_L \Efield\cdot d\vect{r} = - \frac1c \iint\limits_S \frac{\partial\Bfield}{\partial t}\cdot d\vect{S} &&\text{\small Закон електромагнітної індукції} \label{Int III}\tag{Int III}\\
	&\oint\limits_L \Hfield\cdot d\vect{r} =\dfrac{4\pi}{c} \iint\limits_S \vect{j}\cdot d\vect{S} +\frac{1}{c} \iint\limits_S  \frac{\partial\Dfield}{\partial t}\cdot d\vect{S} &&\text{\small Закон Ампера} \label{Int IV}\tag{Int IV}
\end{align}

Диференціальна форма:
\begin{flalign}
	\divg\Dfield &= 4\pi\rho \label{Diff I}\tag{Diff I}\\[0.8em]
	\divg\Bfield &= 0 \label{Diff II}\tag{Diff II}\\
	\rot\Efield &= -\dfrac{1}{c}\dfrac{\partial\Bfield}{\partial t} \label{Diff III}\tag{Diff III}\\
	\rot\Hfield &= \dfrac{4\pi}{c} \vect{j}+\dfrac{1}{c}\dfrac{\partial\Dfield}{\partial t} \label{Diff IV}\tag{Diff IV}
\end{flalign}

%\begin{tabular}{llll}
%	\toprule
%Назва рівняння                         & \specialcell{c}{Математична форма                                                          \\(система СГС)} &  \(\oiint \Dfield\cdot d\vect{S} = 4\pi\rho\) 	&\\ \midrule
%	Теорема Гауса для електричного поля    & \(\divg\Dfield=4\pi\rho \)                                                                & &\\
%	Теорема Гауса для магнітного поля      & \(\divg\Bfield=0\)                                                                       & &\\
%	Закон електромагнітної індукції        & \(\rot\Efield=-\dfrac{1}{c}\dfrac{\partial\Bfield}{\partial t}\)                         & &\\
%	Теорема про циркуляцію магнітного поля & \(\rot\Hfield=\dfrac{4\pi}{c} \vect{j}+\dfrac{1}{c}\dfrac{\partial\Dfield}{\partial t}\) & &\\ \bottomrule
%\end{tabular}

\noindent%
$ \Efield $~-- вектор напруженості електричного поля,\\
$ \Hfield $~-- вектор напруженості магнітного поля,\\
$ \Dfield $~-- вектор індукції електричного поля, визначається формулою:
\begin{equation}
	\Dfield = \Efield + 4\pi\vect{P}
\end{equation}
\begin{equation}
	\Bfield = \Hfield + 4\pi\vect{M}
\end{equation}
$ \vect{P} $~-- вектор поляризації (дипольний момент одиниці об'єму),\\
$ \vect{M} $~-- вектор намагніченості (дипольний магнітний момент одиниці об'єму),\\
$\rho$~-- об'ємна густина вільних електричних зарядів,\\
$\vect{j}$~-- вектор густини електричного струму провідності.

У випадку лінійних ізотропних середовищ:
\begin{align}
	\Dfield & = \epsilon\Efield, \\
	\Bfield & = \mu\Hfield.
\end{align}
$\epsilon$~-- діелектрична проникність середовища,\\
$\mu$~-- магнітна проникність середовища.

\section{Закони збереження}

З рівнянь Максвелла випливають закони збереження:
\begin{itemize}
	\item електричного заряду
	      \begin{equation}
		      \frac{\partial \rho}{\partial t} + \divg\vect{j} = 0
	      \end{equation}
	\item енергії
	      \begin{equation}
		      \frac{\partial w}{\partial t} + \divg\vect{S} = -  \vect{j}\cdot\Efield
	      \end{equation}
\end{itemize}
$ w $~-- густина енергії електромагнітного поля, визначається формулою:
\begin{equation}
	w = \frac{1}{8\pi} (\Efield\cdot\Dfield + \Hfield\cdot\Bfield) 
\end{equation}
\noindent%
$ \vect{S} $~-- вектор Пойнтінга (вектор густини потоку енергії), визначається формулою:
\begin{equation}
	\vect{S} = \frac{c}{4\pi} \Efield\times\Hfield,
\end{equation}
величина
\begin{equation}
	p =\vect{j}\cdot\Efield ,
\end{equation}
є потужністю, що виділяється в одиниці об'єму речовини (закон Джоуля-Ленца).

\section{Граничні умови для полів}

\begin{align}
	E_{1\tau} &= E_{2\tau}, \\
	D_{2n} - D_{1n} &= 4\pi\sigma            \\
	B_{1n} &= B_{2n}, \quad       \\
	H_{2\tau} - H_{1\tau} &= \frac{4\pi}{c}i
\end{align}

\noindent%
$\sigma$~-- поверхнева густина вільних зарядів на границі розділу,\\
$i$~-- поверхневий струм провідності, який тече по границі розділу.

\section{Зв'язок між полями та потенціалами}

\begin{align}
	\Efield &= -\frac{1}{c}\frac{\partial \vect{A}}{\partial t} - \vect{\nabla}\phi, \\
	 \Bfield &= \rot\vect{A}
\end{align}
$\phi$~-- скалярний потенціал,\\
$ \vect{A} $~-- вектор-потенціал.

Потенціали електромагнітного поля визначені не однозначно, якщо замість потенціалів $\vect{A}$ та $\phi$ вибрати інші $\vect{A}'$ та $\phi'$, які пов'язані з вихідними перетвореннями:
\begin{align}
	\vect{A}' & = \vect{A} +\vect{\nabla}f                      \\
	\phi'     & = \phi - \frac1c \frac{\partial f}{\partial t},
\end{align}
де  $f(x,y,z,t)$~-- довільна функція, то характеристики поля $\Efield$ та $\Bfield$ залишаються незмінними.

%\section{Електродинаміка в 4-вимірних позначеннях}

%\section{Лоренцові перетворення полів. Інваріанти електромагнітного поля}
%
%При перетвореннях Лоренца, у випадку відносного руху систем відліку вздовж осі $OX$ (буст):
%
%\begin{align*}
%	dx & = \left( dx' + Vdt\right) \Gamma           \\
%	dy & = dy'                                      \\
%	dz & = dz'                                      \\
%	dt & = \left(dt' + \frac{V}{c^2}dx\right)\Gamma
%\end{align*}
%вектори $\Efield$ та $\Bfield$ перетворюються за законами:
%
%\begin{align}
%	E_x' =  E_x, \\
%	B_x' = B_x, \\
%	E_y' = \left( E_y - \dfrac{V}{c} B_z\right) \Gamma, \\
%	B_y' = \left( B_y + \dfrac{V}{c} E_z\right) \Gamma, \\
%	E_z' = \left( E_z + \dfrac{V}{c} B_y\right) \Gamma, \\
%	B_z' = \left( B_z - \dfrac{V}{c} E_y\right) \Gamma, \\
%\end{align}
%де $\Gamma = \frac{1}{\sqrt{1 - \frac{V^2}{c^2}}}$. 
%
%Для випадку $v \ll c$, формули перетворення полів приймають вигляд:
%\begin{align}
%	\Efield' = \Efield + \left[ \frac{\vect{V}}{c} \times\Bfield\right] , \\ 
%	\Bfield' = \Bfield - \left[ \frac{\vect{V}}{c} \times\Efield\right].                                                            
%\end{align}
%
%\section{Електродинаміка у тензорній формі}%
%
%4-Вектор густини струму:
%\begin{equation}
%	j^i = \rho\frac{dx^i}{dt} = (c\rho, \vect{j}),
%\end{equation}
%де $\rho$~-- густина електричного заряду, $\vect{j}$~-- 3-вектор густини струму.
%
%4-потенціал електромагнітного поля:
%
%\begin{equation}
%	A_i = (\phi, -\vect{A}).
%\end{equation}
%
%Калібровочні перетворення:
%
%\begin{equation}
%	A_k' = A_k - \frac{\partial f}{\partial x^k}
%\end{equation}
%Тензор електромагнітного поля:
%
%\begin{equation}
%	F_{ik} = \frac{\partial A_k}{\partial  x^i} - \frac{\partial A_i}{\partial  x^k},
%\end{equation}
%
%\noindent у формі матриці, тензор $F_{ik}$ має вигляд:
%
%\begin{equation}
%	F_{ik} = \left(
%	\begin{array}{cccc}
%			0    & E_x  & E_y  & E_z  \\
%			-E_x & 0    & -B_z & B_y  \\
%			-E_y & B_z  & 0    & -B_x \\
%			-E_z & -B_y & B_x  & 0
%		\end{array}
%	\right),
%\end{equation}
%де $i,k = 0,1,2,3$ ($i$~-- нумерує рядки, $k$~-- стовпці.)
%
%Із компонент тензора можна утворити інваріанти:
%\begin{equation*}
%	F_{ik}F^{ik} = \inv, \quad e^{iklm}F_{ik}F_{lm} = \inv,
%\end{equation*}
%де $e^{iklm}$~-- повністю антисиметричний тензор, в 3-вигляді, інваріантами перетворень Лоренца виглядають наступним чином:
%\begin{equation*}
%	E^2 - B^2 = \inv, \quad \Efield\cdot\Bfield = \inv.
%\end{equation*}
%
%Рівняння руху частинки в електромагнітному полі:
%\begin{equation}
%	mc\frac{du^i}{ds} = \frac{e}{c}F^{ik}u_k,
%\end{equation}
%де $u_k$~-- 4-швидкість частинки.
%
%Рівняння Максвелла в тензорній формі у вакуумі\footnote{\label{Vacuum_Equations}Запис рівнянь у тензорному вигляді за присутності середовища можна знайти тут \url{https://en.wikipedia.org/wiki/Electromagnetic_tensor}}%
%
%\begin{gather}
%	\frac{\partial F^{ik}}{\partial x^k} =                                                                                -\frac{4\pi}{c}j^i \\
%	\frac{\partial F^{ik}}{\partial x^l} + \frac{\partial F^{kl}}{\partial x^i} +\frac{\partial F^{li}}{\partial x^k}  = 0.
%\end{gather}
%
%Тензор енергії-імпульсу електромагнітного поля (див. зноску~\ref{Vacuum_Equations}):
%
%\begin{equation}
%	T^{ik} = \frac{1}{4\pi}\left( -F^{il}F^k_{ l} + \frac14g^{ik}F_{lm}F^{lm}\right) ,
%\end{equation}
%де $g^{ik}$~-- метричний тензор.