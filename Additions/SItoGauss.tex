% !TeX program = lualatex
% !TeX encoding = utf8
% !TeX spellcheck = uk_UA
% !TeX root =../EMProblems.tex

\newpage
\section{Системи одиниць в електродинаміці}\label{SItoGauss}

\begin{table}[h!]\centering
	\tabcaption{Таблиця переводу виразів і формул із однієї системи одиниць в іншу}
	\label{tab:SItoGaussFormulae}
	\small
	\begin{tabular}{M{7.5cm}ll}
		\toprule
		Величина                                                   & Система SI                                                                   & Система СГС                               \\ \midrule
		Швидкість світла                                           & $(\epsilon_0\mu_0)^{-\nfrac{1}{2}}$                                          & $c$                                       \\
		Напруженість електричного поля (потенціал)                 & $\sqrt{4\pi\epsilon_0}\,\Efield \, (\phi)$                                   & $\Efield \, (\phi)$                       \\
		Електрична індукція                                        & $\sqrt{\frac{4\pi}{\epsilon_0}} \Dfield$                                     & $\Dfield$                                 \\
		Густина заряду (заряд, струм, густина струму, поляризація) & $\dfrac{1}{\sqrt{4\pi\epsilon_0}}\, \rho$ ($q$, $I$, $\vect{j}$, $\vect{P}$) & $\rho$ ($q$, $I$, $\vect{j}$, $\vect{P}$) \\
		Магнітна індукція (вектор-потенціал)                       & $\sqrt{\dfrac{4\pi}{\mu_0}}\,\Bfield\, (\vect{A})$                           & $\Bfield\, (\vect{A})$                    \\
		Напруженість магнітного поля                               & $\sqrt{4\pi \mu_0}\,\Hfield$                                                 & $\Hfield$                                 \\
		Намагніченість                                             & $\sqrt{\dfrac{\mu_0}{4\pi}}\, \vect{M}$                                      & $\vect{M}$                                \\
		Провідність                                                & $\dfrac{\sigma}{4\pi\epsilon_0}$                                             & $\sigma$                                  \\
		%Діелектрична проникність                                   & $\dfrac{\epsilon}{\epsilon_0}$                                                  & $\epsilon$                                    \\
		%Магнітна проникність                                       & $\dfrac{\mu}{\mu_0}$                                                            & $\mu$                                         \\
		Опір                                                       & $4\pi\epsilon_0 R$                                                             & $R$                                       \\
		Ємність                                                    & $\dfrac{C}{4\pi\epsilon_0}$                                                  & $C$                                       \\
		Індуктивність                                              & $\dfrac{4\pi}{\mu_0} L$                                                           & $L$                                       \\ \bottomrule
	\end{tabular}
\end{table}
Табл.~\ref{tab:SItoGaussFormulae} дає схему переводу виразів і рівнянь з системи СГС в систему SI, і навпаки, при незмінності всіх механічних величин. Щоб за допомогою табл.~\ref{tab:SItoGaussFormulae} перетворити будь-яке рівняння, записане в системі одиниць СГС, в рівняння в системі SI, \emph{слід в обох частинах рівняння замінити символи, перелічені в стовпчику <<Система СГС>>, на відповідні символи системи SI}.

Наприклад, якщо ми маємо вираз для магнітної індукції нескінченного соленоїда в системі СГС:
\[
	B = \frac{4\pi}{c} I n.
\]

Для переведення цієї формули в систему SI, замінимо ліву частину $B \to \sqrt{\dfrac{4\pi}{\mu_0}} B$, в правій частині замінимо $\nfrac{1}{c} \to \sqrt{\epsilon_0\mu_0} $, а $I \to \dfrac{1}{\sqrt{4\pi\epsilon_0}} I$. Тоді отримаємо:
\[
	\sqrt{\dfrac{4\pi}{\mu_0}} B = 4\pi\sqrt{\epsilon_0\mu_0} \frac{1}{\sqrt{4\pi\epsilon_0}} I n,
\]
звідки
\[
	B = \mu_0 I n.
\]



\begin{table}[h!]\centering
	\tabcaption{Чисельні коефіцієнти переводу між системами SI та Гауса}
	\label{tab:SItoGauss}
	\small
	\begin{tabular}{M{6.5cm}M{4cm}M{3.5cm}l}
		\toprule
		Величина                       & Система SI    & Система СГС              & \specialcell{c}{Відношення    \\1 SI = X СГС} \\ \midrule
		Довжина                        & метр, м       & сантиметр, см            & $10^{2}$                      \\
		Маса                           & кілограм, кг  & грам, г                  & $10^{3}$                      \\
		Час                            & секунда, с    & секунда, с               & $1$                           \\
		Сила                           & Ньютон, Н     & дина, дина               & $10^{5}$                      \\
		Робота, енергія                & Джоуль, Дж    & ерг                      & $10^{7}$                      \\
		Тиск                           & Паскаль, Па   & дина/см$^2$              & $10$                          \\
		Електричний заряд              & Кулон, Кл     & \specialcell{l}{Франклін                                 \\(статКулон), Фр} & $3\cdot 10^{9}$                           \\
		Густина електричного заряду    & Кл/м$^3$      & Фр/см$^3$                & $3\cdot 10^{3}$               \\
		Сила струму                    & Ампер, А      & Фр/с (статАмпер)         & $3\cdot 10^{9}$               \\
		Густина струму                 & А/м$^2$       & Фр/(с$\cdot$см$^2$)      & $3\cdot 10^{5}$               \\
		Електричний дипольний момент   & Кл$\cdot$м    & Фр$\cdot$см              & $3\cdot 10^{11}$              \\
		Електрична напруга             & Вольт, В      & статВольт, статВ         & $\nfrac{1}{3} \cdot 10^{-2}$  \\
		Напруженість електричного поля & В/м           & статВ/см                 & $\nfrac{1}{3} \cdot 10^{-4}$  \\
		Індукція електричного поля     & Кл/м$^2$      & Фр/см$^2$                & $3 \cdot 10^{5}$              \\
		Електричний опір               & Ом, Ом        & с/см                     & $\nfrac{1}{9} \cdot 10^{-11}$ \\
		Електрична ємність             & Фарад, Ф      & см                       & $9 \cdot 10^{11}$             \\
		Напруженість магнітного поля   & А/м           & Ерстед, Е                & $4\pi\cdot 10^{-3}$           \\
		Магнітна індукція              & Тесла, Тл     & Гаус, Гс                 & $10^{4}$                      \\
		Магнітний дипольний момент     & А$\cdot$м$^2$ & ерг/Гс                   & $10^{3}$                      \\
		Магнітний потік                & Вебер, Вб     & Максвел, Мкс             & $10^{8}$                      \\
		Індуктивність                  & Генрі, Гн     & см                       & $10^{9}$                      \\ \bottomrule
	\end{tabular}
\end{table}

У визначенні розмірності індуктивності в Гаусовій системі існує деяка плутанина. Вона пов'язана з тим, що ряд авторів використовує електромагнітну систему одиниць для введення поняття індуктивності. В цій системі енергія магнітного поля визначається як $\nfrac{L'I^2}{2}$, в гаусовій же системі ця величина дається означенням $\nfrac{LI^2}{2c^2}$. Звідки  видно, що $L' = \nfrac{L}{c^2}$, а отже розмірністю $L'$ є [с$^2$/см].
