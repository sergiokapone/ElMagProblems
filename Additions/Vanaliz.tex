% !TeX program = lualatex
% !TeX encoding = utf8
% !TeX spellcheck = uk_UA
% !TeX root =../EMProblems.tex

%% --------------------------------------------------------
\section{Основні формули векторного аналізу}\label{Vanaliz}
%% --------------------------------------------------------


%% --------------------------------------------------------
\subsection{Диференціальні оператори}
%% --------------------------------------------------------

Оператор <<набла>> $\vect{\nabla}$
\begin{equation}\label{nabla}
	\vect{\nabla} = \mathrm{grad} = \frac{\partial}{\partial \vect{r}}
\end{equation}

Оператор Лапласа (лапласіан) $\Laplasian$
\begin{equation}\label{laplasian}
	\Laplasian = \divg\vect{\nabla} = \vect{\nabla}^2
\end{equation}

Дивергенція вектора $\vect{A}$
\begin{equation}\label{div}
	\mathrm{div}\,\vect{A} = \divg\vect{A}
\end{equation}

Ротор вектора $\vect{A}$
\begin{equation}\label{rot}
	\mathrm{rot}\,\vect{A} = \rot\vect{A}
\end{equation}

%% --------------------------------------------------------
\subsection{Диференціальні операції в різних системах координат}
%% --------------------------------------------------------


%% --------------------------------------------------------
\subsubsection{Декартова система координат}
%% --------------------------------------------------------

\begin{align}\label{cartesian}
	\vect{\nabla}\psi & = \frac{\partial \psi}{\partial x} \vect{i} + \frac{\partial \psi}{\partial y} \vect{j} + \frac{\partial \psi}{\partial z} \vect{k} \\
	\Laplasian\psi    & = \frac{\partial^2 \psi}{\partial x^2} + \frac{\partial^2 \psi}{\partial y^2} + \frac{\partial^2 \psi }{\partial z^2}               \\
	\divg\vect{A}     & = \frac{\partial A_x}{\partial x}  + \frac{\partial A_y}{\partial y} + \frac{\partial A_z}{\partial z}                              \\
	\rot\vect{A}      & = \left( \frac{\partial A_z}{\partial y}  - \frac{\partial A_y}{\partial z}\right)  \vect{i} +
	\left( \frac{\partial A_x}{\partial z}  - \frac{\partial A_z}{\partial x}\right)  \vect{j} +
	\left( \frac{\partial A_y}{\partial x}  - \frac{\partial A_x}{\partial y}\right)  \vect{k}
\end{align}

%% --------------------------------------------------------
\subsubsection{Циліндрична система координат}
%% --------------------------------------------------------

\begin{align}\label{cylindric}
	\vect{\nabla}\psi & = \frac{\partial \psi}{\partial \rho} \vect{e}_{\rho} + \frac{1}{\rho}\frac{\partial \psi}{\partial \phi} \vect{e}_{\phi} + \frac{\partial \psi}{\partial z} \vect{k}                                                                                            \\
	\Laplasian\psi    & = \frac{1}{r} \frac{\partial }{\partial r} \left( r \frac{\partial \psi}{\partial r} \right) + \frac{1}{r^2} \frac{\partial^2 \psi}{\partial \varphi^2} + \frac{\partial^2 \psi}{\partial z^2}                                                                   \\
	\divg\vect{A}     & = \frac{1}{\rho}\frac{\partial \left(\rho A_{\rho }\right)}{\partial \rho }+\frac{1}{\rho }\frac{\partial A_{\varphi } }{\partial \varphi }+\frac{\partial A_{z}}{\partial z}                                                                                    \\
	\rot\vect{A}      & =\left({\frac {1}{\rho }}{\frac {\partial A_{z}}{\partial \varphi }}-{\frac {\partial A_{\varphi }}{\partial z}}\right)\vect{e}_{\rho}+\left({\frac {\partial A_{\rho }}{\partial z}}-{\frac {\partial A_{z}}{\partial \rho }}\right)\vect{e}_{\phi} + \nonumber \\
	                  & +
	{\frac {1}{\rho }}\left({\frac {\partial (\rho A_{\varphi })}{\partial \rho }}-{\frac {\partial A_{\rho }}{\partial \varphi }}\right)\vect{k}
\end{align}

%% --------------------------------------------------------
\subsubsection{Сферична система координат}
%% --------------------------------------------------------

\begin{align}\label{spheric}
	\vect{\nabla}\psi & = \frac{\partial \psi}{\partial r} \vect{e}_{r} + \frac{1}{r}\frac{\partial \psi}{\partial \theta} \vect{e}_{\theta} + \frac{1}{r\sin\theta}\frac{\partial \psi}{\partial \phi} \vect{e}_{\phi}                                                                                                            \\
	\Laplasian\psi    & = \frac{1}{r^2} \frac{\partial}{\partial r} \left( r^2 \frac{\partial \psi}{\partial r} \right) + \frac{1}{r^2 \sin \theta} \frac{\partial \psi}{\partial \theta} \left( \sin \theta \frac{\partial \psi}{\partial \theta} \right) + \frac{1}{r^2\sin^2 \theta} \frac{\partial^2 \psi}{\partial \varphi^2} \\
	\divg\vect{A}     & =\frac{1}{r^2}\frac{\partial \left(r^2A_r\right)}{\partial r}
	+
	\frac{1}{r\sin \theta }\frac{\partial}{\partial \theta }\left(A_{\theta }\sin \theta \right)
	+
	\frac{1}{r\sin\theta}\frac{\partial A_{\varphi}}{\partial \varphi }                                                                                                                                                                                                                                                            \\
	\rot\vect{A}      & = \frac{1}{r\sin \theta }\left(\frac{\partial}{\partial \theta }\left(A_{\varphi }\sin \theta \right)-\frac{\partial A_{\theta }}{\partial \varphi }\right)\vect{e}_{r}
	+
	\frac{1}{r}\left(\frac{1}{\sin \theta }\frac{\partial A_{r}}{\partial \varphi }-\frac{\partial}{\partial r}\left(rA_{\varphi }\right)\right)\vect{e}_{\theta}
	+ \nonumber                                                                                                                                                                                                                                                                                                                    \\
	                  & + \frac{1}{r}\left(\frac{\partial}{\partial r}\left(rA_{\theta }\right)-\frac{\partial A_{r}}{\partial \theta }\right)\vect{e}_{\phi}
\end{align}

%% --------------------------------------------------------
\subsection{Другі похідні}
%% --------------------------------------------------------

\begin{align}
	\mathrm{rot}\,\mathrm{grad}\,\phi    & = \rot(\vect{\nabla}\phi)  = 0                                             \\
	\mathrm{div}\,\mathrm{rot}\,\vect{A} & = \divg(\rot\vect{A})  = 0                                                 \\
	\mathrm{rot}\,\mathrm{rot}\,\vect{A} & = \rot(\rot\vect{A})  = \vect{\nabla}(\divg\vect{A}) - \Laplasian \vect{A}
\end{align}

%% --------------------------------------------------------
\subsection{Похідні від добутків}
%% --------------------------------------------------------

\begin{align}
	\mathrm{grad}\,(\phi \psi)             & = \psi\,\mathrm{grad}\,\phi +\phi\, \mathrm{grad}\,\psi                                                                                                                           \\
	\mathrm{div}\,(\phi \vect{A})          & = \phi\,\mathrm{div}\,\vect{A} + \vect{A}\,\mathrm{grad}\,\phi                                                                                                                    \\
	\mathrm{rot}\,(\phi \vect{A})          & = \phi\,\mathrm{rot}\,\vect{A} + \mathrm{grad}\,\phi \times \vect{A}                                                                                                              \\
	\mathrm{grad}\,(\vect{A}\cdot\vect{B}) & = \vect{B}\times\mathrm{rot}\,\vect{A} + \vect{A}\times\mathrm{rot}\,\vect{B} + \left( \vect{B}\vect{\nabla}\right)\vect{A} + \left( \vect{A}\vect{\nabla}\right)\vect{B}         \\
	\mathrm{div}\,(\vect{A}\times\vect{B}) & = \vect{B}\cdot\mathrm{rot}\,\vect{A} - \vect{A}\cdot\mathrm{rot}\,\vect{B}                                                                                                       \\
	\mathrm{rot}\,(\vect{A}\times\vect{B}) & = \left( \vect{B}\vect{\nabla}\right)\vect{A} - \left( \vect{A}\vect{\nabla}\right)\vect{B} + 	\vect{A}\,\mathrm{div}\,\vect{B} - \vect{B}\,\mathrm{div}\,\vect{A} \label{rotvect} \\
	\frac12\mathrm{grad}\,A^2              & =  \left( \vect{A}\vect{\nabla}\right)\vect{A} + \vect{A}\times\mathrm{rot}\,\vect{A}
\end{align}

%% --------------------------------------------------------
\subsection{Інтегральні характеристики та теореми}
%% --------------------------------------------------------


Потік вектора $\vect{A}$
\begin{equation}\label{flux}
	\Phi_{\vect{A}} = \iint\limits_S \vect{A}\cdot d\vect{S}
\end{equation}

Циркуляція вектора $\vect{A}$
\begin{equation}\label{circulation}
	\Gamma_{\vect{A}} = \oint\limits_L \vect{A}\cdot d\vect{l}
\end{equation}

Теорема Остроградського-Гаусса
\begin{equation}\label{OGTheorem}
	\oiint\limits_S \vect{A}\cdot d\vect{S} = \iiint\limits_V \divg\vect{A} dV
\end{equation}

Теорема Стокса
\begin{equation}\label{Stoksheorem}
	\oint\limits_L \vect{A}\cdot d\vect{l} = \iint\limits_S \rot\vect{A} \cdot d\vect{S}
\end{equation}

Теорема Гріна:
\begin{equation}\label{Grin}
	\iiint\limits_{V}(\varphi \nabla ^{2}\psi -\psi \nabla ^{2}\varphi )\ dV=\iint \limits _{S}(\varphi \vect{\nabla} \psi -\psi \vect{\nabla} \varphi )\cdot d\vect{S}.
\end{equation}
%\subsection{Поліноми Лежандра}
%
%Розкладання в ряд оберненої відстані:
%\begin{equation}\label{multipole}
%	\frac{1}{\left| \vect{R_0} - \vect{r}\right| } = \frac{1}{\sqrt{R_0^2 + r^2 - 2rR_0\cos\chi} } =
%	\begin{cases}
%	 \frac{1}{R_0}\sum\limits_{l=0}^{\infty}\left(\frac{r}{R_0}\right)^l P_l(\cos\chi), \, &r < R_0\\
%	 \frac{1}{r}\sum\limits_{l=0}^{\infty}\left(\frac{R_0}{r}\right)^l P_l(\cos\chi),   \, &r > R_0
%	\end{cases}
%\end{equation}
%де $P_l(\cos\chi)$~--- поліноми Лежандра від аргумента $\cos\chi$, який в свою чергу можна виразити через сферичні координати для вектора $\vect{R_0}(\Theta, \Phi)$ та для $\vect{r}(\theta, \phi)$:
%\[
%	\cos\chi = \cos\Theta\cos\theta + \sin\Theta\sin\theta\cos(\Phi - \phi).
%\]
%
%Перші 4 поліноми Лежандра мають вигляд:
%
%\begin{align*}
%& P_{0}(\cos\chi)  = 1, && P_{1}(\cos\chi)  = \cos\chi, \\
%& P_{2}(\cos\chi)  = \frac {1}{2}(3\cos^2\chi-1), && P_{3}(x)  =\frac {1}{2}(5\cos^2\chi-3\cos\chi)
%\end{align*}

\clearpage

%% --------------------------------------------------------
\section{Деякі інтеграли, що часто використовуються}
%% --------------------------------------------------------

\begin{center}
\begin{tblr}{
  colspec={X[c] X[c]},
  row{1}={font=\large\bfseries},
  rowsep=1em,
  hlines,
  vlines
}
Інтеграл & Значення \\

$\displaystyle \int_0^\infty x^n e^{-x} \, dx$ &
$\displaystyle
\begin{cases}
1, & n = 0 \\[2mm]
\frac{\sqrt{\pi}}{2}, & n = \frac{1}{2} \\[1mm]
1, & n = 1 \\[1mm]
2, & n = 2
\end{cases}$ \\

$\displaystyle \int_0^\infty x^n e^{-x^2} \, dx$ &
$\displaystyle
\begin{cases}
\frac{\sqrt{\pi}}{2}, & n = 0 \\[1mm]
\frac{1}{2}, & n = 1 \\[1mm]
\frac{\sqrt{\pi}}{4}, & n = 2 \\[1mm]
\frac{1}{2}, & n = 3
\end{cases}$ \\

$\displaystyle \int_0^\infty \frac{x^n \, dx}{e^x - 1}$ &
$\displaystyle
\begin{cases}
2.31, & n = \frac{1}{2} \\[1mm]
\frac{\pi^2}{6}, & n = 1 \\[1mm]
2.405, & n = 2 \\[1mm]
\frac{\pi^4}{15}, & n = 3 \\[1mm]
24.9, & n = 4
\end{cases}$ \\

$\displaystyle \int_0^\alpha \frac{x^3 \, dx}{e^x - 1}$ &
$\displaystyle
\begin{cases}
0.225, & \alpha = 1 \\[1mm]
1.18, & \alpha = 2 \\[1mm]
2.56, & \alpha = 3 \\[1mm]
4.91, & \alpha = 5 \\[1mm]
6.43, & \alpha = 10
\end{cases}$ \\
\end{tblr}
\end{center}